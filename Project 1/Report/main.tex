%%%%%%%%%%%%%%% COPYRIGHT ANTOINE HUGOUNET & ETHEL VILLENEUVE
%%%%%%%%%%%%%%%%%%%%%%%%%%%%%%%%%%%%%%%%%%%%%%%%%%%%%%%
%%%%%%%%%%%%%%%%%%%%%%%%%%%%%%%%%%%%%%%%%%%%%%%%%%%%%%%


\documentclass[a4paper, twoside, 11pt]{report}

\usepackage[utf8]{inputenc}
\usepackage[T1]{fontenc}
\usepackage[french]{babel}
\usepackage[top= 120pt, left=80pt, right=80pt]{geometry} %marges
\usepackage{setspace} %interlignage
\usepackage{url}
\usepackage{graphicx}
\usepackage{lmodern} 
\usepackage{array}
\usepackage{csquotes}
\usepackage[numbers,square]{natbib}
\usepackage{soul}
\usepackage{hyperref}
\usepackage{amsthm}
\usepackage{amsmath}
\usepackage{color}
\usepackage[usenames,dvipsnames,svgnames,table]{xcolor}
\usepackage{ragged2e}
\usepackage{adjustbox}
\usepackage{amssymb}
\usepackage{amsmath}
\usepackage{tikz}
\usepackage{dsfont}
\usepackage{amsmath}
\usepackage{algorithm}
\usepackage{algpseudocode}


%%%%%% STYLES DE THÉORÈMES

\newtheoremstyle{theorem}%	Name
  {}%	Space above
  {}%	Space below
  {}%	Body font
  {}%	Indent amount
  {\bfseries}%	Theorem head font
  {.}%	Punctuation after theorem head
  { }%	Space after theorem head, ' ', or \newline
  {}%	Theorem head spec (can be left empty, meaning `normal')

\newtheoremstyle{exemple}%	Name
  {}%	Space above
  {}%	Space below
  {\color{Gray}\itshape}%	Body font
  {}%	Indent amount
  {\color{Gray}\itshape}%	Theorem head font
  {.}%	Punctuation after theorem head
  { }%	Space after theorem head, ' ', or \newline
  {}%	Theorem head spec (can be left empty, meaning `normal')

\newtheoremstyle{remark}%	Name
  {}%	Space above
  {}%	Space below
  {\itshape}%	Body font
  {}%	Indent amount
  {\bfseries}%	Theorem head font
  {.}%	Punctuation after theorem head
  { }%	Space after theorem head, ' ', or \newline
  {}%	Theorem head spec (can be left empty, meaning `normal')
  
%%%%%% DÉCLARATION DES THÉORÈMES

\theoremstyle{theorem}
\newtheorem{theorem}{Théorème}[section]
\newtheorem{lemme}{Lemme}[section]
\newtheorem{proposition}{Proposition}[section]
\newtheorem{definition}{Définition}[section]

\theoremstyle{remark}
\newtheorem{remark}{Remarque}[chapter]

\theoremstyle{exemple}
\newtheorem*{exemple}{Exemple}


%%%%%% COMMANDES QUI SIMPLIFIENT LA VIE

\newcommand{\legende}[1]{
\begin{center}
	\begin{minipage}{12cm}
		\begin{center}
			\textit{\textcolor{WildStrawberry!30}{#1}}
		\end{center}
	\end{minipage}
\end{center}}

\newcommand{\sherlock}[2]{
	\begin{equation}
		\textcolor{WildStrawberry}{#1}
	\end{equation}
	\legende{#2}
}
	
\newcommand{\sherlocked}[1]{
	\begin{equation}
		\textcolor{WildStrawberry}{#1}
	\end{equation}
}

	
\newcommand{\defSherlock}[3]{
	\begin{definition}[\textbf{#1}]
		\sherlock{#2}{#3}
	\end{definition}
}

\newcommand{\defSherlocked}[2]{
	\begin{definition}[\textbf{#1}]
		\sherlocked{#2}
	\end{definition}
}	

\newcommand{\propSherlock}[3]{
	\begin{proposition}[\textbf{#1}]
		\sherlock{#2}{#3}
	\end{proposition}
}

\newcommand{\propSherlocked}[2]{
	\begin{proposition}[\textbf{#1}]
		\sherlocked{#2}
	\end{proposition}
}

\newcommand{\textSherlocked}[1]{
	\begin{center}
		\textcolor{WildStrawberry}{#1}
	\end{center}
}

\newcommand{\N}{\mathbb{N}}
\newcommand{\Z}{\mathbb{Z}}
\newcommand{\R}{\mathbb{R}}
\newcommand{\C}{\mathbb{C}}

%%%%%%%%%%%%%%%%%%%%%%%%%%%%%%%%%%%%%%%%%%%%%%%%%%%%%%%
%%%%%%%%%%%%%%%%%%%%%%%%%%%%%%%%%%%%%%%%%%%%%%%%%%%%%%%
%%%%%%%%%%%%%%%%%%%%%%%%%%%%%%%%%%%%%%%%%%%%%%%%%%%%%%%

\title{FYS3150\\Project 1 - Atom Heart Matrix}
\author{Hugounet, Antoine \& Villeneuve, Ethel}
\date{September 2017 \\University of Oslo \\ \url{https://github.com/kryzar/Alcyonide.git}}



\begin{document}
\maketitle

\renewcommand{\contentsname}{Table of contents}
\tableofcontents
	

\chapter*{A short glimpse to this project}	
	
\paragraph{}This project uses the pretext of a differential equations to study the implementation of Linear Algebra algorithms with modern languages. It aims to make one realize that many of the scientific problems can be solved with simple Linear Algebra tricks which involve matrices, vectors, diagonalization, etc. However, if those algorithm remain rather simple for a human being, they require quite a lot of memory and power for a standard laptop.
	\\Calculating the inverse of a matrix using its adjugate matrix and its determinant is a huge operation for sizes $n>{10}^{3}$. The gaussian elimination appears to be less direct but way more efficient regarding the operation time. It's a progress but the LU decomposition is even better than the gaussian elimination, and its brute force approach does not work for matrices which have any zero diagonal term. It even produces a result (which is obviously wrong) for non-singular matrices! Efficient but restrictive.

\paragraph{}In the end, the results we obtain are approximations of exact mathematical results, due to loss of numerical precision and the fact that computers are built on discrete values. The problematic is then to evaluate carefully our objectives. Stability ? Precision ? Speed ? Memory ? It is the job of the programmer to find its own balance and to code in consequence. The first rule of computational science is that no program is fully reliable, no program fully is impartial.
	
	
	
\chapter{Resolution of the equation $\mathbf{Ax=w}$ with the method of gaussian elimination}

	\footnotesize{\itshape{The files for this chapter are in the folder "General case" on GitHub. You will find the program in C++ and some data from the algorithm.}}

\normalsize
	\paragraph{}The gaussian elimination is an algorithm used for solving systems of linear equations\footnote{It is also called "row reduction".}. This algorithm is composed of two parts : a forward substitution to make the $n$ sized matrix an upper triangular matrix and a backward substitution to solve the equations.
	\\The gaussian elimination is a good example of a easily computerizable algorithm, but which can be subtle to code. For large $n$, it is more efficient than the classical method of matrix inversion which needs a lot of calculations.
	
		\section{Transformation of A into a upper triangular matrix}					
			\caption{Gaussian algorithm}
			\begin{algorithmic}[1]
				\For{$\text{row} = \overline{1,n}$}
					\State $\text{ratio} \gets A[\text{row}, \text{row}-1] / A[\text{row}-1, \text{row}-1]$
					\For{$i = \overline{\text{row},n}$}
						\State $A[i,\text{row}-1] \gets 0$
					\EndFor
					
					\For{$\text{col} = \overline{\text{row}, n}$}
						\State $A[\text{row}, \text{col}] \gets A[\text{row}, \text{col}] - \text{ratio} * A[\text{row} - 1 , \text{col}]$
					\EndFor
					\State $w[\text{row}] \gets w[\text{row}] - \text{ratio} * w[\text{row} - 1]$
					\If $A[\text{row} , \text{row}] = 0$
						\State \text{ERROR. Division by 0!}
					\EndIf
				\EndFor
			\end{algorithmic}
	

\begin{remark}We have directly changed to $0$ the first element of each line and all the elements below. It spares the computer some useless calculations and avoids us numerical approximations which might not end to $0$. \end{remark}
	
\paragraph{}The program begins with a set of tests to set aside the possible errors. This algorithm is quite efficient but is no match at all to Armadillo. Please build and run the main.cpp file to test the efficiency and limits of the algorithm\footnote{We put the algorithm in an external cpp file to make it easier to export use with other problems.}
	
	\section{Comparison with other methods}
	\paragraph{} There are two other methods : the brute force approach of matrix inversion and the LU decomposition.
		\begin{itemize}
			\item{The classical approach for the matrix inversion needs a lot of calculations and will not be efficient enough compared to the two others.}
			\item{The LU decomposition is also a process of gaussian elimination. It consist of separate the matrix A into two matrices : a lower triangular ($L$) and an upper triangular ($U$) matrix. The determinant is the sum of $U$'s diagonal elements. We will use the LU function of \emph{Armadillo} to compare the efficiency of the methods. In the following array, you will see the results regarding the execution time and relative error for our algorithm and the Armadillo function.}
		\end{itemize}
		\paragraph{}
	
	\begin{center}
	\begin{tabular}{|*{5}{c|}}
  \hline
  n  & \multicolumn{3}{|c|}{Operation time (s)}  & relative error \\
	\hline
	 & Our algorithm & LU armadillo & difference between them & \\
  \hline
  $2$ & $9*10^-6$   & $8*10^-6$ & $1*10^-6$ & $-0.903089987$ \\
	\hline
  $3$ & $7*10^-6$ & $1*10^-6$ & $6*10^-6$ & 0.7781512504 \\
  \hline
	$10^5$ & $5.68913$ & $2*10^-6$ & 5.689128 & 6.454015709 \\
	\hline	
\end{tabular}
\end{center}
	
	\paragraph{}We clearly see here that the LU decomposition function of Armadillo is more efficient than our gaussian elimination algorithm. For large n, our algorithm is way less efficient than Armadillo. 
	\paragraph{}
		\begin{center}
		\begin{tabular}{|*{2}{c|}}
  \hline
  \multicolumn{2}{|c|}{Number of FLOPs} \\
	\hline
	General case & LU decomposition\\
	\hline	
	$3n^3 + 2n^2 +2n -5$ & $\frac{2}{3} n^3$ \\
	\hline
\end{tabular}
\end{center}
	
	\paragraph{} The gaussian elimination is still a well-balanced algorithm. It is easy to imagine and to code, its execution-time is fair, for not very very large n, it works for most of cases (see next section). It certainly do not have the best precision but it is balanced between the important criteria of programming. 
	
	\paragraph{}It is the programmer's responsibility to determine precisely what is important for his program. In our case, the resolution of a set of linear equation in an easy and quick way, the criteria are satisfied. But we wouldn't use this algorithm to launch a rocket… 
	
	\section{Limits of the gaussian elimination law}
	\paragraph{} The gaussian elimination has many advantages. But the main drawback is that it works only for regular matrices. For a singular matrix, the algorithm returns a \emph{wrong} result (see the example 3 in the results file).
	\paragraph{} Regular matrices are associated with bijective mapping. There has to be only one result for this equation. The error does not indeed come from the program but from the mathematical algorithm itself. Knowing this, we tried to cover as many cases as possible using the exit() function from C++, but it is fairly impossible to predict all possibilities in a relatively simple way.
	\paragraph{}To make the program work correctly, the diagonal terms of the matrix $\mathbf{A}$ must not be zeros. Indeed, each substitution needs a division by a different non-zero diagonal coefficient.
	\\However, we can fix that by swapping rows but it makes the algorithm more complicated. This goes against the method's peculiar simplicity. Moreover, it may use up memory unnecessarily because of the storage of some matrix rows. Even by doing that, the algorithm will not be perfect, the precision will not be improved. For those cases, we may think of a very optimized inverse or LU algorithm. 
	
	
\chapter{Special case : tridiagonal matrix}

\footnotesize{\itshape{The files for this chapter are in the folder "Tridiagonal case" on GitHub. You will find the program in C++ and some data from the algorithm.}}

\normalsize

	\paragraph{} Here, we attempt to specialize the general algorithm for tridiagonal matrices. The reasoning itself will be the same, but the declaration of the matrix elements will be different. The goal is to solve the one-dimensional Poisson equation using the gaussian elimination. 
	
	\section{The one-dimensional Poisson equation}
	\paragraph{} We can write the Poisson equation as $-u''(x)=f(x)$, with $u''(x)$ the second derivative of $u(x)$ and $u$ the electrostatic potential. The Poisson equation is mainly used in electromagnetism to describe potential field caused by a charge. This is a simple example (because of its simplification $-u''(x)=f(x)$) of a linear second-order differential equation as we often find them in physics. 
	
		\subsection{Special use of Poisson equation}
		\paragraph{} Before trying to solve this equation, let us be more specific on the conditions : we use here the Dirichlet boundary conditions.
		\begin{equation*} 
		-u''(x) = f(x), x \hspace{0.5cm} x\in(0,1), \hspace{0.5cm} u(0) = u(1) = 0
		\end{equation*}
		Then, we approximate $u$ with discretized values $v_{i}$ with grid points $x_{i} = ih , \hspace{0.3cm} i\in(0,n+1) $ and $x_{0} = 0$ and $x_{n+1} = 1$. We define also the step length as $h=1/(n+1)$, and the boundary conditions $v_{0} = v_{n+1} = 0$.
		\\We have
		\begin{align}
		-u''(x) &= f(x)
		\\-\frac{v_{i+1} + v_{i-1} - 2v_{i}}{h^2} &= f_{i} \hspace{0.5cm} \mathrm{for} \hspace{0.1cm} i = \overline {(1,n)}
		\end{align}
		
		\subsection{Rewriting of the equation as a set of linear equations}
		\begin{align}
		&-\frac{v_{i+1} + v_{i-1} - 2v_{i}}{h^2} = f_{i} \hspace{0.5cm} \mathrm{for} \hspace{0.1cm} i = \overline {(1,n)}
		\\ \Rightarrow \hspace{0,1cm} &-(v_{i+1} + v_{i-1} - 2v_{i}) = h^2 f_{i}
		\\ \Rightarrow \hspace{0,1cm} &- v_{i-1} + 2v_{i} - v_{i+1} = h^2 f_{i}
		\end{align}
		So, if we express this equation in matrix form we have
		\begin{equation*}
		\begin{bmatrix}
                           2& -1& 0 &\dots   & \dots &0 \\
                           -1 & 2 & -1 &0 &\dots &\dots \\
                           0&-1 &2 & -1 & 0 & \dots \\
                           & \dots   & \dots &\dots   &\dots & \dots \\
													 & \dots   & \dots &\dots   &\dots & \dots \\
													 & \dots   & \dots &\dots   &\dots & \dots \\
                           0&\dots   &  &-1 &2& -1 \\
                           0&\dots    &  & 0  &-1 & 2 \\
                      \end{bmatrix} . \begin{bmatrix}
                           v_1\\
                           v_2\\
                           \dots \\
                          v_{i-1}  \\
													v_i \\
													v_{i+1} \\
                          \dots \\
                           v_n\\
                      \end{bmatrix}
											= h^2 \begin{bmatrix}
                           f_1\\
                           f_2\\
                           \dots \\
                           f_{i-1} \\
													f_i \\
													f_{i+1} \\
                          \dots \\
                           f_n\\
                      \end{bmatrix}
		\end{equation*} 									
		So we can rewrite the equation as
		\begin{equation*}
		\mathbf{A}\mathbf{v} = \tilde{\mathbf{b}},
		\end{equation*}
		with 
		\begin{equation*}
		\mathbf{A}  = \begin{bmatrix}
                           2& -1& 0 &\dots   & \dots &0 \\
                           -1 & 2 & -1 &0 &\dots &\dots \\
                           0&-1 &2 & -1 & 0 & \dots \\
                           & \dots   & \dots &\dots   &\dots & \dots \\
                           0&\dots   &  &-1 &2& -1 \\
                           0&\dots    &  & 0  &-1 & 2 \\
                      \end{bmatrix} \hspace{0.3cm} \text{and}\hspace{0.3cm}		\tilde{\mathbf{b}} = h^2\mathbf{f}
		\end{equation*} 
		
		\paragraph{} Now we have the Poisson equation written as a set of linear equation. We can use the gaussian elimination to solve it.
		
		\section{Special algorithm}
			\subsection {Specialization of the general algorithm}
		\paragraph{} In the special case program, we did not use a matrix as we did for the general case but three dynamic arrays to set the three diagonals. The vector $\tilde{\mathbf{b}}$ corresponds to $g$ in the program for more simplicity ; also be careful with the index $i$ in the program : $i$ in the program corresponds to $i+1$ in reality, that is because of arrays which must begin with $i=0$. The process is quite the same as the general case for the gaussian elimination.
		
			\subsection{Solution to the Poisson equation}
			\paragraph{} By running the program, we have values for the approximate formula of Poisson equation $-\frac{v_{i+1} + v_{i-1} - 2v_{i}}{h^2} = f_{i}$. Let us compare these values to the closed-form solution of the equation $u(x) = 1 - ( 1 - e^{-10} ) x - e ^{-10x}$. What we have done with that program is to approach the exact value. See below the comparative graph between what we should have and our results. On top of that, main.cpp requires arguments :
				\begin{itemize}
				\item{argv[0] : name of the program} 
				\item{argv[1] : size n}
				\item{argv[2] : coefficient of the first diagonal}
				\item{argv[3] : coefficient of the second diagonal}
				\item{argv[4] : coefficient of the third diagonal}
				\item{argv[5] : path of the file for the results}
				\item{argv[6] : path of the file for the data}
				
			\end{itemize}
				
			
			\begin{figure}[htbp]
					\begin{center}
							% GNUPLOT: LaTeX picture
\setlength{\unitlength}{0.240900pt}
\ifx\plotpoint\undefined\newsavebox{\plotpoint}\fi
\begin{picture}(1500,900)(0,0)
\sbox{\plotpoint}{\rule[-0.200pt]{0.400pt}{0.400pt}}%
\put(110.0,131.0){\rule[-0.200pt]{4.818pt}{0.400pt}}
\put(90,131){\makebox(0,0)[r]{$0$}}
\put(1419.0,131.0){\rule[-0.200pt]{4.818pt}{0.400pt}}
\put(110.0,235.0){\rule[-0.200pt]{4.818pt}{0.400pt}}
\put(90,235){\makebox(0,0)[r]{$0.1$}}
\put(1419.0,235.0){\rule[-0.200pt]{4.818pt}{0.400pt}}
\put(110.0,339.0){\rule[-0.200pt]{4.818pt}{0.400pt}}
\put(90,339){\makebox(0,0)[r]{$0.2$}}
\put(1419.0,339.0){\rule[-0.200pt]{4.818pt}{0.400pt}}
\put(110.0,443.0){\rule[-0.200pt]{4.818pt}{0.400pt}}
\put(90,443){\makebox(0,0)[r]{$0.3$}}
\put(1419.0,443.0){\rule[-0.200pt]{4.818pt}{0.400pt}}
\put(110.0,546.0){\rule[-0.200pt]{4.818pt}{0.400pt}}
\put(90,546){\makebox(0,0)[r]{$0.4$}}
\put(1419.0,546.0){\rule[-0.200pt]{4.818pt}{0.400pt}}
\put(110.0,650.0){\rule[-0.200pt]{4.818pt}{0.400pt}}
\put(90,650){\makebox(0,0)[r]{$0.5$}}
\put(1419.0,650.0){\rule[-0.200pt]{4.818pt}{0.400pt}}
\put(110.0,754.0){\rule[-0.200pt]{4.818pt}{0.400pt}}
\put(90,754){\makebox(0,0)[r]{$0.6$}}
\put(1419.0,754.0){\rule[-0.200pt]{4.818pt}{0.400pt}}
\put(110.0,858.0){\rule[-0.200pt]{4.818pt}{0.400pt}}
\put(90,858){\makebox(0,0)[r]{$0.7$}}
\put(1419.0,858.0){\rule[-0.200pt]{4.818pt}{0.400pt}}
\put(110.0,131.0){\rule[-0.200pt]{0.400pt}{4.818pt}}
\put(110,90){\makebox(0,0){$0$}}
\put(110.0,838.0){\rule[-0.200pt]{0.400pt}{4.818pt}}
\put(243.0,131.0){\rule[-0.200pt]{0.400pt}{4.818pt}}
\put(243,90){\makebox(0,0){$0.1$}}
\put(243.0,838.0){\rule[-0.200pt]{0.400pt}{4.818pt}}
\put(376.0,131.0){\rule[-0.200pt]{0.400pt}{4.818pt}}
\put(376,90){\makebox(0,0){$0.2$}}
\put(376.0,838.0){\rule[-0.200pt]{0.400pt}{4.818pt}}
\put(509.0,131.0){\rule[-0.200pt]{0.400pt}{4.818pt}}
\put(509,90){\makebox(0,0){$0.3$}}
\put(509.0,838.0){\rule[-0.200pt]{0.400pt}{4.818pt}}
\put(642.0,131.0){\rule[-0.200pt]{0.400pt}{4.818pt}}
\put(642,90){\makebox(0,0){$0.4$}}
\put(642.0,838.0){\rule[-0.200pt]{0.400pt}{4.818pt}}
\put(775.0,131.0){\rule[-0.200pt]{0.400pt}{4.818pt}}
\put(775,90){\makebox(0,0){$0.5$}}
\put(775.0,838.0){\rule[-0.200pt]{0.400pt}{4.818pt}}
\put(907.0,131.0){\rule[-0.200pt]{0.400pt}{4.818pt}}
\put(907,90){\makebox(0,0){$0.6$}}
\put(907.0,838.0){\rule[-0.200pt]{0.400pt}{4.818pt}}
\put(1040.0,131.0){\rule[-0.200pt]{0.400pt}{4.818pt}}
\put(1040,90){\makebox(0,0){$0.7$}}
\put(1040.0,838.0){\rule[-0.200pt]{0.400pt}{4.818pt}}
\put(1173.0,131.0){\rule[-0.200pt]{0.400pt}{4.818pt}}
\put(1173,90){\makebox(0,0){$0.8$}}
\put(1173.0,838.0){\rule[-0.200pt]{0.400pt}{4.818pt}}
\put(1306.0,131.0){\rule[-0.200pt]{0.400pt}{4.818pt}}
\put(1306,90){\makebox(0,0){$0.9$}}
\put(1306.0,838.0){\rule[-0.200pt]{0.400pt}{4.818pt}}
\put(1439.0,131.0){\rule[-0.200pt]{0.400pt}{4.818pt}}
\put(1439,90){\makebox(0,0){$1$}}
\put(1439.0,838.0){\rule[-0.200pt]{0.400pt}{4.818pt}}
\put(110.0,131.0){\rule[-0.200pt]{0.400pt}{175.134pt}}
\put(110.0,131.0){\rule[-0.200pt]{320.156pt}{0.400pt}}
\put(1439.0,131.0){\rule[-0.200pt]{0.400pt}{175.134pt}}
\put(110.0,858.0){\rule[-0.200pt]{320.156pt}{0.400pt}}
\put(774,29){\makebox(0,0){xi}}
\put(1279,817){\makebox(0,0)[r]{numerical solution}}
\put(1299.0,817.0){\rule[-0.200pt]{24.090pt}{0.400pt}}
\put(231,622){\usebox{\plotpoint}}
\multiput(231.58,622.00)(0.499,0.599){239}{\rule{0.120pt}{0.579pt}}
\multiput(230.17,622.00)(121.000,143.798){2}{\rule{0.400pt}{0.290pt}}
\multiput(352.00,767.59)(10.797,0.482){9}{\rule{8.100pt}{0.116pt}}
\multiput(352.00,766.17)(103.188,6.000){2}{\rule{4.050pt}{0.400pt}}
\multiput(472.00,771.92)(1.214,-0.498){97}{\rule{1.068pt}{0.120pt}}
\multiput(472.00,772.17)(118.783,-50.000){2}{\rule{0.534pt}{0.400pt}}
\multiput(593.00,721.92)(0.830,-0.499){143}{\rule{0.763pt}{0.120pt}}
\multiput(593.00,722.17)(119.416,-73.000){2}{\rule{0.382pt}{0.400pt}}
\multiput(714.00,648.92)(0.738,-0.499){161}{\rule{0.690pt}{0.120pt}}
\multiput(714.00,649.17)(119.567,-82.000){2}{\rule{0.345pt}{0.400pt}}
\multiput(835.00,566.92)(0.704,-0.499){169}{\rule{0.663pt}{0.120pt}}
\multiput(835.00,567.17)(119.624,-86.000){2}{\rule{0.331pt}{0.400pt}}
\multiput(956.00,480.92)(0.696,-0.499){171}{\rule{0.656pt}{0.120pt}}
\multiput(956.00,481.17)(119.638,-87.000){2}{\rule{0.328pt}{0.400pt}}
\multiput(1077.00,393.92)(0.682,-0.499){173}{\rule{0.645pt}{0.120pt}}
\multiput(1077.00,394.17)(118.660,-88.000){2}{\rule{0.323pt}{0.400pt}}
\multiput(1197.00,305.92)(0.688,-0.499){173}{\rule{0.650pt}{0.120pt}}
\multiput(1197.00,306.17)(119.651,-88.000){2}{\rule{0.325pt}{0.400pt}}
\put(1279,776){\makebox(0,0)[r]{exact solution}}
\multiput(1299,776)(20.756,0.000){5}{\usebox{\plotpoint}}
\put(1399,776){\usebox{\plotpoint}}
\put(231,657){\usebox{\plotpoint}}
\multiput(231,657)(12.772,16.361){10}{\usebox{\plotpoint}}
\multiput(352,812)(20.730,1.036){6}{\usebox{\plotpoint}}
\multiput(472,818)(19.012,-8.327){6}{\usebox{\plotpoint}}
\multiput(593,765)(17.379,-11.347){7}{\usebox{\plotpoint}}
\multiput(714,686)(16.852,-12.117){7}{\usebox{\plotpoint}}
\multiput(835,599)(16.522,-12.562){8}{\usebox{\plotpoint}}
\multiput(956,507)(16.456,-12.648){7}{\usebox{\plotpoint}}
\multiput(1077,414)(16.339,-12.799){7}{\usebox{\plotpoint}}
\multiput(1197,320)(16.325,-12.817){8}{\usebox{\plotpoint}}
\put(1318,225){\usebox{\plotpoint}}
\put(110.0,131.0){\rule[-0.200pt]{0.400pt}{175.134pt}}
\put(110.0,131.0){\rule[-0.200pt]{320.156pt}{0.400pt}}
\put(1439.0,131.0){\rule[-0.200pt]{0.400pt}{175.134pt}}
\put(110.0,858.0){\rule[-0.200pt]{320.156pt}{0.400pt}}
\end{picture}

					\end{center}
					\caption{Comparative graph between numerical and exact results for $n=10$}
			\end{figure}
				
			\begin{figure}[htbp]
					\begin{center}
							% GNUPLOT: LaTeX picture
\setlength{\unitlength}{0.240900pt}
\ifx\plotpoint\undefined\newsavebox{\plotpoint}\fi
\begin{picture}(1500,900)(0,0)
\sbox{\plotpoint}{\rule[-0.200pt]{0.400pt}{0.400pt}}%
\put(110.0,131.0){\rule[-0.200pt]{4.818pt}{0.400pt}}
\put(90,131){\makebox(0,0)[r]{$0$}}
\put(1419.0,131.0){\rule[-0.200pt]{4.818pt}{0.400pt}}
\put(110.0,235.0){\rule[-0.200pt]{4.818pt}{0.400pt}}
\put(90,235){\makebox(0,0)[r]{$0.1$}}
\put(1419.0,235.0){\rule[-0.200pt]{4.818pt}{0.400pt}}
\put(110.0,339.0){\rule[-0.200pt]{4.818pt}{0.400pt}}
\put(90,339){\makebox(0,0)[r]{$0.2$}}
\put(1419.0,339.0){\rule[-0.200pt]{4.818pt}{0.400pt}}
\put(110.0,443.0){\rule[-0.200pt]{4.818pt}{0.400pt}}
\put(90,443){\makebox(0,0)[r]{$0.3$}}
\put(1419.0,443.0){\rule[-0.200pt]{4.818pt}{0.400pt}}
\put(110.0,546.0){\rule[-0.200pt]{4.818pt}{0.400pt}}
\put(90,546){\makebox(0,0)[r]{$0.4$}}
\put(1419.0,546.0){\rule[-0.200pt]{4.818pt}{0.400pt}}
\put(110.0,650.0){\rule[-0.200pt]{4.818pt}{0.400pt}}
\put(90,650){\makebox(0,0)[r]{$0.5$}}
\put(1419.0,650.0){\rule[-0.200pt]{4.818pt}{0.400pt}}
\put(110.0,754.0){\rule[-0.200pt]{4.818pt}{0.400pt}}
\put(90,754){\makebox(0,0)[r]{$0.6$}}
\put(1419.0,754.0){\rule[-0.200pt]{4.818pt}{0.400pt}}
\put(110.0,858.0){\rule[-0.200pt]{4.818pt}{0.400pt}}
\put(90,858){\makebox(0,0)[r]{$0.7$}}
\put(1419.0,858.0){\rule[-0.200pt]{4.818pt}{0.400pt}}
\put(110.0,131.0){\rule[-0.200pt]{0.400pt}{4.818pt}}
\put(110,90){\makebox(0,0){$0$}}
\put(110.0,838.0){\rule[-0.200pt]{0.400pt}{4.818pt}}
\put(243.0,131.0){\rule[-0.200pt]{0.400pt}{4.818pt}}
\put(243,90){\makebox(0,0){$0.1$}}
\put(243.0,838.0){\rule[-0.200pt]{0.400pt}{4.818pt}}
\put(376.0,131.0){\rule[-0.200pt]{0.400pt}{4.818pt}}
\put(376,90){\makebox(0,0){$0.2$}}
\put(376.0,838.0){\rule[-0.200pt]{0.400pt}{4.818pt}}
\put(509.0,131.0){\rule[-0.200pt]{0.400pt}{4.818pt}}
\put(509,90){\makebox(0,0){$0.3$}}
\put(509.0,838.0){\rule[-0.200pt]{0.400pt}{4.818pt}}
\put(642.0,131.0){\rule[-0.200pt]{0.400pt}{4.818pt}}
\put(642,90){\makebox(0,0){$0.4$}}
\put(642.0,838.0){\rule[-0.200pt]{0.400pt}{4.818pt}}
\put(775.0,131.0){\rule[-0.200pt]{0.400pt}{4.818pt}}
\put(775,90){\makebox(0,0){$0.5$}}
\put(775.0,838.0){\rule[-0.200pt]{0.400pt}{4.818pt}}
\put(907.0,131.0){\rule[-0.200pt]{0.400pt}{4.818pt}}
\put(907,90){\makebox(0,0){$0.6$}}
\put(907.0,838.0){\rule[-0.200pt]{0.400pt}{4.818pt}}
\put(1040.0,131.0){\rule[-0.200pt]{0.400pt}{4.818pt}}
\put(1040,90){\makebox(0,0){$0.7$}}
\put(1040.0,838.0){\rule[-0.200pt]{0.400pt}{4.818pt}}
\put(1173.0,131.0){\rule[-0.200pt]{0.400pt}{4.818pt}}
\put(1173,90){\makebox(0,0){$0.8$}}
\put(1173.0,838.0){\rule[-0.200pt]{0.400pt}{4.818pt}}
\put(1306.0,131.0){\rule[-0.200pt]{0.400pt}{4.818pt}}
\put(1306,90){\makebox(0,0){$0.9$}}
\put(1306.0,838.0){\rule[-0.200pt]{0.400pt}{4.818pt}}
\put(1439.0,131.0){\rule[-0.200pt]{0.400pt}{4.818pt}}
\put(1439,90){\makebox(0,0){$1$}}
\put(1439.0,838.0){\rule[-0.200pt]{0.400pt}{4.818pt}}
\put(110.0,131.0){\rule[-0.200pt]{0.400pt}{175.134pt}}
\put(110.0,131.0){\rule[-0.200pt]{320.156pt}{0.400pt}}
\put(1439.0,131.0){\rule[-0.200pt]{0.400pt}{175.134pt}}
\put(110.0,858.0){\rule[-0.200pt]{320.156pt}{0.400pt}}
\put(774,29){\makebox(0,0){xi}}
\put(1279,817){\makebox(0,0)[r]{numerical solution}}
\put(1299.0,817.0){\rule[-0.200pt]{24.090pt}{0.400pt}}
\put(123,219){\usebox{\plotpoint}}
\multiput(123.58,219.00)(0.493,3.074){23}{\rule{0.119pt}{2.500pt}}
\multiput(122.17,219.00)(13.000,72.811){2}{\rule{0.400pt}{1.250pt}}
\multiput(136.58,297.00)(0.493,2.757){23}{\rule{0.119pt}{2.254pt}}
\multiput(135.17,297.00)(13.000,65.322){2}{\rule{0.400pt}{1.127pt}}
\multiput(149.58,367.00)(0.494,2.260){25}{\rule{0.119pt}{1.871pt}}
\multiput(148.17,367.00)(14.000,58.116){2}{\rule{0.400pt}{0.936pt}}
\multiput(163.58,429.00)(0.493,2.201){23}{\rule{0.119pt}{1.823pt}}
\multiput(162.17,429.00)(13.000,52.216){2}{\rule{0.400pt}{0.912pt}}
\multiput(176.58,485.00)(0.493,1.924){23}{\rule{0.119pt}{1.608pt}}
\multiput(175.17,485.00)(13.000,45.663){2}{\rule{0.400pt}{0.804pt}}
\multiput(189.58,534.00)(0.493,1.726){23}{\rule{0.119pt}{1.454pt}}
\multiput(188.17,534.00)(13.000,40.982){2}{\rule{0.400pt}{0.727pt}}
\multiput(202.58,578.00)(0.493,1.527){23}{\rule{0.119pt}{1.300pt}}
\multiput(201.17,578.00)(13.000,36.302){2}{\rule{0.400pt}{0.650pt}}
\multiput(215.58,617.00)(0.493,1.329){23}{\rule{0.119pt}{1.146pt}}
\multiput(214.17,617.00)(13.000,31.621){2}{\rule{0.400pt}{0.573pt}}
\multiput(228.58,651.00)(0.494,1.048){25}{\rule{0.119pt}{0.929pt}}
\multiput(227.17,651.00)(14.000,27.073){2}{\rule{0.400pt}{0.464pt}}
\multiput(242.58,680.00)(0.493,1.012){23}{\rule{0.119pt}{0.900pt}}
\multiput(241.17,680.00)(13.000,24.132){2}{\rule{0.400pt}{0.450pt}}
\multiput(255.58,706.00)(0.493,0.893){23}{\rule{0.119pt}{0.808pt}}
\multiput(254.17,706.00)(13.000,21.324){2}{\rule{0.400pt}{0.404pt}}
\multiput(268.58,729.00)(0.493,0.774){23}{\rule{0.119pt}{0.715pt}}
\multiput(267.17,729.00)(13.000,18.515){2}{\rule{0.400pt}{0.358pt}}
\multiput(281.58,749.00)(0.493,0.616){23}{\rule{0.119pt}{0.592pt}}
\multiput(280.17,749.00)(13.000,14.771){2}{\rule{0.400pt}{0.296pt}}
\multiput(294.58,765.00)(0.493,0.576){23}{\rule{0.119pt}{0.562pt}}
\multiput(293.17,765.00)(13.000,13.834){2}{\rule{0.400pt}{0.281pt}}
\multiput(307.00,780.58)(0.637,0.492){19}{\rule{0.609pt}{0.118pt}}
\multiput(307.00,779.17)(12.736,11.000){2}{\rule{0.305pt}{0.400pt}}
\multiput(321.00,791.58)(0.652,0.491){17}{\rule{0.620pt}{0.118pt}}
\multiput(321.00,790.17)(11.713,10.000){2}{\rule{0.310pt}{0.400pt}}
\multiput(334.00,801.59)(0.824,0.488){13}{\rule{0.750pt}{0.117pt}}
\multiput(334.00,800.17)(11.443,8.000){2}{\rule{0.375pt}{0.400pt}}
\multiput(347.00,809.59)(1.123,0.482){9}{\rule{0.967pt}{0.116pt}}
\multiput(347.00,808.17)(10.994,6.000){2}{\rule{0.483pt}{0.400pt}}
\multiput(360.00,815.59)(1.378,0.477){7}{\rule{1.140pt}{0.115pt}}
\multiput(360.00,814.17)(10.634,5.000){2}{\rule{0.570pt}{0.400pt}}
\multiput(373.00,820.61)(2.695,0.447){3}{\rule{1.833pt}{0.108pt}}
\multiput(373.00,819.17)(9.195,3.000){2}{\rule{0.917pt}{0.400pt}}
\put(386,823.17){\rule{2.700pt}{0.400pt}}
\multiput(386.00,822.17)(7.396,2.000){2}{\rule{1.350pt}{0.400pt}}
\put(399,824.67){\rule{3.373pt}{0.400pt}}
\multiput(399.00,824.17)(7.000,1.000){2}{\rule{1.686pt}{0.400pt}}
\put(426,824.67){\rule{3.132pt}{0.400pt}}
\multiput(426.00,825.17)(6.500,-1.000){2}{\rule{1.566pt}{0.400pt}}
\put(439,823.17){\rule{2.700pt}{0.400pt}}
\multiput(439.00,824.17)(7.396,-2.000){2}{\rule{1.350pt}{0.400pt}}
\multiput(452.00,821.95)(2.695,-0.447){3}{\rule{1.833pt}{0.108pt}}
\multiput(452.00,822.17)(9.195,-3.000){2}{\rule{0.917pt}{0.400pt}}
\multiput(465.00,818.94)(1.797,-0.468){5}{\rule{1.400pt}{0.113pt}}
\multiput(465.00,819.17)(10.094,-4.000){2}{\rule{0.700pt}{0.400pt}}
\multiput(478.00,814.94)(1.943,-0.468){5}{\rule{1.500pt}{0.113pt}}
\multiput(478.00,815.17)(10.887,-4.000){2}{\rule{0.750pt}{0.400pt}}
\multiput(492.00,810.93)(1.378,-0.477){7}{\rule{1.140pt}{0.115pt}}
\multiput(492.00,811.17)(10.634,-5.000){2}{\rule{0.570pt}{0.400pt}}
\multiput(505.00,805.93)(1.378,-0.477){7}{\rule{1.140pt}{0.115pt}}
\multiput(505.00,806.17)(10.634,-5.000){2}{\rule{0.570pt}{0.400pt}}
\multiput(518.00,800.93)(1.123,-0.482){9}{\rule{0.967pt}{0.116pt}}
\multiput(518.00,801.17)(10.994,-6.000){2}{\rule{0.483pt}{0.400pt}}
\multiput(531.00,794.93)(1.123,-0.482){9}{\rule{0.967pt}{0.116pt}}
\multiput(531.00,795.17)(10.994,-6.000){2}{\rule{0.483pt}{0.400pt}}
\multiput(544.00,788.93)(1.123,-0.482){9}{\rule{0.967pt}{0.116pt}}
\multiput(544.00,789.17)(10.994,-6.000){2}{\rule{0.483pt}{0.400pt}}
\multiput(557.00,782.93)(1.026,-0.485){11}{\rule{0.900pt}{0.117pt}}
\multiput(557.00,783.17)(12.132,-7.000){2}{\rule{0.450pt}{0.400pt}}
\multiput(571.00,775.93)(0.824,-0.488){13}{\rule{0.750pt}{0.117pt}}
\multiput(571.00,776.17)(11.443,-8.000){2}{\rule{0.375pt}{0.400pt}}
\multiput(584.00,767.93)(0.950,-0.485){11}{\rule{0.843pt}{0.117pt}}
\multiput(584.00,768.17)(11.251,-7.000){2}{\rule{0.421pt}{0.400pt}}
\multiput(597.00,760.93)(0.824,-0.488){13}{\rule{0.750pt}{0.117pt}}
\multiput(597.00,761.17)(11.443,-8.000){2}{\rule{0.375pt}{0.400pt}}
\multiput(610.00,752.93)(0.824,-0.488){13}{\rule{0.750pt}{0.117pt}}
\multiput(610.00,753.17)(11.443,-8.000){2}{\rule{0.375pt}{0.400pt}}
\multiput(623.00,744.93)(0.824,-0.488){13}{\rule{0.750pt}{0.117pt}}
\multiput(623.00,745.17)(11.443,-8.000){2}{\rule{0.375pt}{0.400pt}}
\multiput(636.00,736.93)(0.824,-0.488){13}{\rule{0.750pt}{0.117pt}}
\multiput(636.00,737.17)(11.443,-8.000){2}{\rule{0.375pt}{0.400pt}}
\multiput(649.00,728.93)(0.786,-0.489){15}{\rule{0.722pt}{0.118pt}}
\multiput(649.00,729.17)(12.501,-9.000){2}{\rule{0.361pt}{0.400pt}}
\multiput(663.00,719.93)(0.728,-0.489){15}{\rule{0.678pt}{0.118pt}}
\multiput(663.00,720.17)(11.593,-9.000){2}{\rule{0.339pt}{0.400pt}}
\multiput(676.00,710.93)(0.728,-0.489){15}{\rule{0.678pt}{0.118pt}}
\multiput(676.00,711.17)(11.593,-9.000){2}{\rule{0.339pt}{0.400pt}}
\multiput(689.00,701.93)(0.728,-0.489){15}{\rule{0.678pt}{0.118pt}}
\multiput(689.00,702.17)(11.593,-9.000){2}{\rule{0.339pt}{0.400pt}}
\multiput(702.00,692.93)(0.728,-0.489){15}{\rule{0.678pt}{0.118pt}}
\multiput(702.00,693.17)(11.593,-9.000){2}{\rule{0.339pt}{0.400pt}}
\multiput(715.00,683.93)(0.728,-0.489){15}{\rule{0.678pt}{0.118pt}}
\multiput(715.00,684.17)(11.593,-9.000){2}{\rule{0.339pt}{0.400pt}}
\multiput(728.00,674.93)(0.786,-0.489){15}{\rule{0.722pt}{0.118pt}}
\multiput(728.00,675.17)(12.501,-9.000){2}{\rule{0.361pt}{0.400pt}}
\multiput(742.00,665.92)(0.652,-0.491){17}{\rule{0.620pt}{0.118pt}}
\multiput(742.00,666.17)(11.713,-10.000){2}{\rule{0.310pt}{0.400pt}}
\multiput(755.00,655.93)(0.728,-0.489){15}{\rule{0.678pt}{0.118pt}}
\multiput(755.00,656.17)(11.593,-9.000){2}{\rule{0.339pt}{0.400pt}}
\multiput(768.00,646.92)(0.652,-0.491){17}{\rule{0.620pt}{0.118pt}}
\multiput(768.00,647.17)(11.713,-10.000){2}{\rule{0.310pt}{0.400pt}}
\multiput(781.00,636.92)(0.652,-0.491){17}{\rule{0.620pt}{0.118pt}}
\multiput(781.00,637.17)(11.713,-10.000){2}{\rule{0.310pt}{0.400pt}}
\multiput(794.00,626.93)(0.728,-0.489){15}{\rule{0.678pt}{0.118pt}}
\multiput(794.00,627.17)(11.593,-9.000){2}{\rule{0.339pt}{0.400pt}}
\multiput(807.00,617.92)(0.704,-0.491){17}{\rule{0.660pt}{0.118pt}}
\multiput(807.00,618.17)(12.630,-10.000){2}{\rule{0.330pt}{0.400pt}}
\multiput(821.00,607.92)(0.652,-0.491){17}{\rule{0.620pt}{0.118pt}}
\multiput(821.00,608.17)(11.713,-10.000){2}{\rule{0.310pt}{0.400pt}}
\multiput(834.00,597.92)(0.652,-0.491){17}{\rule{0.620pt}{0.118pt}}
\multiput(834.00,598.17)(11.713,-10.000){2}{\rule{0.310pt}{0.400pt}}
\multiput(847.00,587.92)(0.652,-0.491){17}{\rule{0.620pt}{0.118pt}}
\multiput(847.00,588.17)(11.713,-10.000){2}{\rule{0.310pt}{0.400pt}}
\multiput(860.00,577.93)(0.728,-0.489){15}{\rule{0.678pt}{0.118pt}}
\multiput(860.00,578.17)(11.593,-9.000){2}{\rule{0.339pt}{0.400pt}}
\multiput(873.00,568.92)(0.652,-0.491){17}{\rule{0.620pt}{0.118pt}}
\multiput(873.00,569.17)(11.713,-10.000){2}{\rule{0.310pt}{0.400pt}}
\multiput(886.00,558.92)(0.704,-0.491){17}{\rule{0.660pt}{0.118pt}}
\multiput(886.00,559.17)(12.630,-10.000){2}{\rule{0.330pt}{0.400pt}}
\multiput(900.00,548.92)(0.652,-0.491){17}{\rule{0.620pt}{0.118pt}}
\multiput(900.00,549.17)(11.713,-10.000){2}{\rule{0.310pt}{0.400pt}}
\multiput(913.00,538.92)(0.590,-0.492){19}{\rule{0.573pt}{0.118pt}}
\multiput(913.00,539.17)(11.811,-11.000){2}{\rule{0.286pt}{0.400pt}}
\multiput(926.00,527.92)(0.652,-0.491){17}{\rule{0.620pt}{0.118pt}}
\multiput(926.00,528.17)(11.713,-10.000){2}{\rule{0.310pt}{0.400pt}}
\multiput(939.00,517.92)(0.652,-0.491){17}{\rule{0.620pt}{0.118pt}}
\multiput(939.00,518.17)(11.713,-10.000){2}{\rule{0.310pt}{0.400pt}}
\multiput(952.00,507.92)(0.652,-0.491){17}{\rule{0.620pt}{0.118pt}}
\multiput(952.00,508.17)(11.713,-10.000){2}{\rule{0.310pt}{0.400pt}}
\multiput(965.00,497.92)(0.652,-0.491){17}{\rule{0.620pt}{0.118pt}}
\multiput(965.00,498.17)(11.713,-10.000){2}{\rule{0.310pt}{0.400pt}}
\multiput(978.00,487.92)(0.704,-0.491){17}{\rule{0.660pt}{0.118pt}}
\multiput(978.00,488.17)(12.630,-10.000){2}{\rule{0.330pt}{0.400pt}}
\multiput(992.00,477.92)(0.652,-0.491){17}{\rule{0.620pt}{0.118pt}}
\multiput(992.00,478.17)(11.713,-10.000){2}{\rule{0.310pt}{0.400pt}}
\multiput(1005.00,467.92)(0.652,-0.491){17}{\rule{0.620pt}{0.118pt}}
\multiput(1005.00,468.17)(11.713,-10.000){2}{\rule{0.310pt}{0.400pt}}
\multiput(1018.00,457.92)(0.652,-0.491){17}{\rule{0.620pt}{0.118pt}}
\multiput(1018.00,458.17)(11.713,-10.000){2}{\rule{0.310pt}{0.400pt}}
\multiput(1031.00,447.92)(0.590,-0.492){19}{\rule{0.573pt}{0.118pt}}
\multiput(1031.00,448.17)(11.811,-11.000){2}{\rule{0.286pt}{0.400pt}}
\multiput(1044.00,436.92)(0.652,-0.491){17}{\rule{0.620pt}{0.118pt}}
\multiput(1044.00,437.17)(11.713,-10.000){2}{\rule{0.310pt}{0.400pt}}
\multiput(1057.00,426.92)(0.704,-0.491){17}{\rule{0.660pt}{0.118pt}}
\multiput(1057.00,427.17)(12.630,-10.000){2}{\rule{0.330pt}{0.400pt}}
\multiput(1071.00,416.92)(0.652,-0.491){17}{\rule{0.620pt}{0.118pt}}
\multiput(1071.00,417.17)(11.713,-10.000){2}{\rule{0.310pt}{0.400pt}}
\multiput(1084.00,406.92)(0.652,-0.491){17}{\rule{0.620pt}{0.118pt}}
\multiput(1084.00,407.17)(11.713,-10.000){2}{\rule{0.310pt}{0.400pt}}
\multiput(1097.00,396.92)(0.590,-0.492){19}{\rule{0.573pt}{0.118pt}}
\multiput(1097.00,397.17)(11.811,-11.000){2}{\rule{0.286pt}{0.400pt}}
\multiput(1110.00,385.92)(0.652,-0.491){17}{\rule{0.620pt}{0.118pt}}
\multiput(1110.00,386.17)(11.713,-10.000){2}{\rule{0.310pt}{0.400pt}}
\multiput(1123.00,375.92)(0.652,-0.491){17}{\rule{0.620pt}{0.118pt}}
\multiput(1123.00,376.17)(11.713,-10.000){2}{\rule{0.310pt}{0.400pt}}
\multiput(1136.00,365.92)(0.704,-0.491){17}{\rule{0.660pt}{0.118pt}}
\multiput(1136.00,366.17)(12.630,-10.000){2}{\rule{0.330pt}{0.400pt}}
\multiput(1150.00,355.92)(0.590,-0.492){19}{\rule{0.573pt}{0.118pt}}
\multiput(1150.00,356.17)(11.811,-11.000){2}{\rule{0.286pt}{0.400pt}}
\multiput(1163.00,344.92)(0.652,-0.491){17}{\rule{0.620pt}{0.118pt}}
\multiput(1163.00,345.17)(11.713,-10.000){2}{\rule{0.310pt}{0.400pt}}
\multiput(1176.00,334.92)(0.652,-0.491){17}{\rule{0.620pt}{0.118pt}}
\multiput(1176.00,335.17)(11.713,-10.000){2}{\rule{0.310pt}{0.400pt}}
\multiput(1189.00,324.92)(0.652,-0.491){17}{\rule{0.620pt}{0.118pt}}
\multiput(1189.00,325.17)(11.713,-10.000){2}{\rule{0.310pt}{0.400pt}}
\multiput(1202.00,314.92)(0.590,-0.492){19}{\rule{0.573pt}{0.118pt}}
\multiput(1202.00,315.17)(11.811,-11.000){2}{\rule{0.286pt}{0.400pt}}
\multiput(1215.00,303.92)(0.652,-0.491){17}{\rule{0.620pt}{0.118pt}}
\multiput(1215.00,304.17)(11.713,-10.000){2}{\rule{0.310pt}{0.400pt}}
\multiput(1228.00,293.92)(0.704,-0.491){17}{\rule{0.660pt}{0.118pt}}
\multiput(1228.00,294.17)(12.630,-10.000){2}{\rule{0.330pt}{0.400pt}}
\multiput(1242.00,283.92)(0.652,-0.491){17}{\rule{0.620pt}{0.118pt}}
\multiput(1242.00,284.17)(11.713,-10.000){2}{\rule{0.310pt}{0.400pt}}
\multiput(1255.00,273.92)(0.590,-0.492){19}{\rule{0.573pt}{0.118pt}}
\multiput(1255.00,274.17)(11.811,-11.000){2}{\rule{0.286pt}{0.400pt}}
\multiput(1268.00,262.92)(0.652,-0.491){17}{\rule{0.620pt}{0.118pt}}
\multiput(1268.00,263.17)(11.713,-10.000){2}{\rule{0.310pt}{0.400pt}}
\multiput(1281.00,252.92)(0.652,-0.491){17}{\rule{0.620pt}{0.118pt}}
\multiput(1281.00,253.17)(11.713,-10.000){2}{\rule{0.310pt}{0.400pt}}
\multiput(1294.00,242.92)(0.652,-0.491){17}{\rule{0.620pt}{0.118pt}}
\multiput(1294.00,243.17)(11.713,-10.000){2}{\rule{0.310pt}{0.400pt}}
\multiput(1307.00,232.92)(0.637,-0.492){19}{\rule{0.609pt}{0.118pt}}
\multiput(1307.00,233.17)(12.736,-11.000){2}{\rule{0.305pt}{0.400pt}}
\multiput(1321.00,221.92)(0.652,-0.491){17}{\rule{0.620pt}{0.118pt}}
\multiput(1321.00,222.17)(11.713,-10.000){2}{\rule{0.310pt}{0.400pt}}
\multiput(1334.00,211.92)(0.652,-0.491){17}{\rule{0.620pt}{0.118pt}}
\multiput(1334.00,212.17)(11.713,-10.000){2}{\rule{0.310pt}{0.400pt}}
\multiput(1347.00,201.92)(0.652,-0.491){17}{\rule{0.620pt}{0.118pt}}
\multiput(1347.00,202.17)(11.713,-10.000){2}{\rule{0.310pt}{0.400pt}}
\multiput(1360.00,191.92)(0.590,-0.492){19}{\rule{0.573pt}{0.118pt}}
\multiput(1360.00,192.17)(11.811,-11.000){2}{\rule{0.286pt}{0.400pt}}
\multiput(1373.00,180.92)(0.652,-0.491){17}{\rule{0.620pt}{0.118pt}}
\multiput(1373.00,181.17)(11.713,-10.000){2}{\rule{0.310pt}{0.400pt}}
\multiput(1386.00,170.92)(0.704,-0.491){17}{\rule{0.660pt}{0.118pt}}
\multiput(1386.00,171.17)(12.630,-10.000){2}{\rule{0.330pt}{0.400pt}}
\multiput(1400.00,160.92)(0.652,-0.491){17}{\rule{0.620pt}{0.118pt}}
\multiput(1400.00,161.17)(11.713,-10.000){2}{\rule{0.310pt}{0.400pt}}
\multiput(1413.00,150.92)(0.590,-0.492){19}{\rule{0.573pt}{0.118pt}}
\multiput(1413.00,151.17)(11.811,-11.000){2}{\rule{0.286pt}{0.400pt}}
\put(413.0,826.0){\rule[-0.200pt]{3.132pt}{0.400pt}}
\put(1279,776){\makebox(0,0)[r]{exact solution}}
\multiput(1299,776)(20.756,0.000){5}{\usebox{\plotpoint}}
\put(1399,776){\usebox{\plotpoint}}
\put(123,219){\usebox{\plotpoint}}
\multiput(123,219)(3.412,20.473){4}{\usebox{\plotpoint}}
\multiput(136,297)(3.790,20.407){4}{\usebox{\plotpoint}}
\multiput(149,367)(4.503,20.261){3}{\usebox{\plotpoint}}
\multiput(163,430)(4.774,20.199){3}{\usebox{\plotpoint}}
\multiput(176,485)(5.223,20.088){2}{\usebox{\plotpoint}}
\multiput(189,535)(6.006,19.867){2}{\usebox{\plotpoint}}
\multiput(202,578)(6.563,19.690){2}{\usebox{\plotpoint}}
\multiput(215,617)(7.413,19.387){2}{\usebox{\plotpoint}}
\multiput(228,651)(8.777,18.808){2}{\usebox{\plotpoint}}
\put(250.77,698.54){\usebox{\plotpoint}}
\put(260.56,716.83){\usebox{\plotpoint}}
\put(271.18,734.65){\usebox{\plotpoint}}
\put(283.04,751.67){\usebox{\plotpoint}}
\put(295.85,767.99){\usebox{\plotpoint}}
\put(310.32,782.84){\usebox{\plotpoint}}
\put(326.30,796.08){\usebox{\plotpoint}}
\put(343.40,807.79){\usebox{\plotpoint}}
\put(362.06,816.79){\usebox{\plotpoint}}
\put(381.81,823.03){\usebox{\plotpoint}}
\put(402.29,826.24){\usebox{\plotpoint}}
\put(422.99,826.23){\usebox{\plotpoint}}
\put(443.64,824.29){\usebox{\plotpoint}}
\put(463.99,820.23){\usebox{\plotpoint}}
\put(484.13,815.25){\usebox{\plotpoint}}
\put(503.73,808.49){\usebox{\plotpoint}}
\put(522.96,800.71){\usebox{\plotpoint}}
\put(541.81,792.01){\usebox{\plotpoint}}
\put(560.20,782.40){\usebox{\plotpoint}}
\put(578.64,772.88){\usebox{\plotpoint}}
\put(596.49,762.31){\usebox{\plotpoint}}
\put(614.60,752.17){\usebox{\plotpoint}}
\put(631.95,740.80){\usebox{\plotpoint}}
\put(649.48,729.69){\usebox{\plotpoint}}
\put(666.99,718.55){\usebox{\plotpoint}}
\put(684.36,707.21){\usebox{\plotpoint}}
\put(701.43,695.40){\usebox{\plotpoint}}
\put(718.37,683.41){\usebox{\plotpoint}}
\put(735.24,671.35){\usebox{\plotpoint}}
\put(752.45,659.76){\usebox{\plotpoint}}
\put(769.03,647.28){\usebox{\plotpoint}}
\put(785.92,635.22){\usebox{\plotpoint}}
\put(802.37,622.56){\usebox{\plotpoint}}
\put(819.13,610.33){\usebox{\plotpoint}}
\put(836.10,598.38){\usebox{\plotpoint}}
\put(852.55,585.73){\usebox{\plotpoint}}
\put(869.00,573.07){\usebox{\plotpoint}}
\put(885.45,560.42){\usebox{\plotpoint}}
\put(902.27,548.25){\usebox{\plotpoint}}
\put(918.72,535.60){\usebox{\plotpoint}}
\put(935.17,522.94){\usebox{\plotpoint}}
\put(951.62,510.29){\usebox{\plotpoint}}
\put(967.96,497.49){\usebox{\plotpoint}}
\put(984.19,484.58){\usebox{\plotpoint}}
\put(1000.84,472.20){\usebox{\plotpoint}}
\put(1017.29,459.54){\usebox{\plotpoint}}
\put(1033.74,446.89){\usebox{\plotpoint}}
\put(1049.97,433.95){\usebox{\plotpoint}}
\put(1066.39,421.29){\usebox{\plotpoint}}
\put(1082.96,408.80){\usebox{\plotpoint}}
\put(1099.42,396.14){\usebox{\plotpoint}}
\put(1115.65,383.22){\usebox{\plotpoint}}
\put(1131.82,370.22){\usebox{\plotpoint}}
\put(1148.60,358.00){\usebox{\plotpoint}}
\put(1165.01,345.30){\usebox{\plotpoint}}
\put(1181.04,332.12){\usebox{\plotpoint}}
\put(1197.49,319.47){\usebox{\plotpoint}}
\put(1213.94,306.81){\usebox{\plotpoint}}
\put(1229.95,293.61){\usebox{\plotpoint}}
\put(1246.71,281.38){\usebox{\plotpoint}}
\put(1263.16,268.72){\usebox{\plotpoint}}
\put(1279.18,255.54){\usebox{\plotpoint}}
\put(1295.57,242.80){\usebox{\plotpoint}}
\put(1311.98,230.09){\usebox{\plotpoint}}
\put(1328.36,217.34){\usebox{\plotpoint}}
\put(1344.81,204.69){\usebox{\plotpoint}}
\put(1361.21,191.97){\usebox{\plotpoint}}
\put(1377.21,178.76){\usebox{\plotpoint}}
\put(1393.87,166.38){\usebox{\plotpoint}}
\put(1410.48,153.94){\usebox{\plotpoint}}
\put(1426,141){\usebox{\plotpoint}}
\put(110.0,131.0){\rule[-0.200pt]{0.400pt}{175.134pt}}
\put(110.0,131.0){\rule[-0.200pt]{320.156pt}{0.400pt}}
\put(1439.0,131.0){\rule[-0.200pt]{0.400pt}{175.134pt}}
\put(110.0,858.0){\rule[-0.200pt]{320.156pt}{0.400pt}}
\end{picture}

					\end{center}
					\caption{Comparative graph between numerical and exact results for $n=100$}
			\end{figure}
			
			\begin{figure}[htbp]
					\begin{center}
							% GNUPLOT: LaTeX picture
\setlength{\unitlength}{0.240900pt}
\ifx\plotpoint\undefined\newsavebox{\plotpoint}\fi
\begin{picture}(1500,900)(0,0)
\sbox{\plotpoint}{\rule[-0.200pt]{0.400pt}{0.400pt}}%
\put(110.0,131.0){\rule[-0.200pt]{4.818pt}{0.400pt}}
\put(90,131){\makebox(0,0)[r]{$0$}}
\put(1419.0,131.0){\rule[-0.200pt]{4.818pt}{0.400pt}}
\put(110.0,235.0){\rule[-0.200pt]{4.818pt}{0.400pt}}
\put(90,235){\makebox(0,0)[r]{$0.1$}}
\put(1419.0,235.0){\rule[-0.200pt]{4.818pt}{0.400pt}}
\put(110.0,339.0){\rule[-0.200pt]{4.818pt}{0.400pt}}
\put(90,339){\makebox(0,0)[r]{$0.2$}}
\put(1419.0,339.0){\rule[-0.200pt]{4.818pt}{0.400pt}}
\put(110.0,443.0){\rule[-0.200pt]{4.818pt}{0.400pt}}
\put(90,443){\makebox(0,0)[r]{$0.3$}}
\put(1419.0,443.0){\rule[-0.200pt]{4.818pt}{0.400pt}}
\put(110.0,546.0){\rule[-0.200pt]{4.818pt}{0.400pt}}
\put(90,546){\makebox(0,0)[r]{$0.4$}}
\put(1419.0,546.0){\rule[-0.200pt]{4.818pt}{0.400pt}}
\put(110.0,650.0){\rule[-0.200pt]{4.818pt}{0.400pt}}
\put(90,650){\makebox(0,0)[r]{$0.5$}}
\put(1419.0,650.0){\rule[-0.200pt]{4.818pt}{0.400pt}}
\put(110.0,754.0){\rule[-0.200pt]{4.818pt}{0.400pt}}
\put(90,754){\makebox(0,0)[r]{$0.6$}}
\put(1419.0,754.0){\rule[-0.200pt]{4.818pt}{0.400pt}}
\put(110.0,858.0){\rule[-0.200pt]{4.818pt}{0.400pt}}
\put(90,858){\makebox(0,0)[r]{$0.7$}}
\put(1419.0,858.0){\rule[-0.200pt]{4.818pt}{0.400pt}}
\put(110.0,131.0){\rule[-0.200pt]{0.400pt}{4.818pt}}
\put(110,90){\makebox(0,0){$0$}}
\put(110.0,838.0){\rule[-0.200pt]{0.400pt}{4.818pt}}
\put(243.0,131.0){\rule[-0.200pt]{0.400pt}{4.818pt}}
\put(243,90){\makebox(0,0){$0.1$}}
\put(243.0,838.0){\rule[-0.200pt]{0.400pt}{4.818pt}}
\put(376.0,131.0){\rule[-0.200pt]{0.400pt}{4.818pt}}
\put(376,90){\makebox(0,0){$0.2$}}
\put(376.0,838.0){\rule[-0.200pt]{0.400pt}{4.818pt}}
\put(509.0,131.0){\rule[-0.200pt]{0.400pt}{4.818pt}}
\put(509,90){\makebox(0,0){$0.3$}}
\put(509.0,838.0){\rule[-0.200pt]{0.400pt}{4.818pt}}
\put(642.0,131.0){\rule[-0.200pt]{0.400pt}{4.818pt}}
\put(642,90){\makebox(0,0){$0.4$}}
\put(642.0,838.0){\rule[-0.200pt]{0.400pt}{4.818pt}}
\put(775.0,131.0){\rule[-0.200pt]{0.400pt}{4.818pt}}
\put(775,90){\makebox(0,0){$0.5$}}
\put(775.0,838.0){\rule[-0.200pt]{0.400pt}{4.818pt}}
\put(907.0,131.0){\rule[-0.200pt]{0.400pt}{4.818pt}}
\put(907,90){\makebox(0,0){$0.6$}}
\put(907.0,838.0){\rule[-0.200pt]{0.400pt}{4.818pt}}
\put(1040.0,131.0){\rule[-0.200pt]{0.400pt}{4.818pt}}
\put(1040,90){\makebox(0,0){$0.7$}}
\put(1040.0,838.0){\rule[-0.200pt]{0.400pt}{4.818pt}}
\put(1173.0,131.0){\rule[-0.200pt]{0.400pt}{4.818pt}}
\put(1173,90){\makebox(0,0){$0.8$}}
\put(1173.0,838.0){\rule[-0.200pt]{0.400pt}{4.818pt}}
\put(1306.0,131.0){\rule[-0.200pt]{0.400pt}{4.818pt}}
\put(1306,90){\makebox(0,0){$0.9$}}
\put(1306.0,838.0){\rule[-0.200pt]{0.400pt}{4.818pt}}
\put(1439.0,131.0){\rule[-0.200pt]{0.400pt}{4.818pt}}
\put(1439,90){\makebox(0,0){$1$}}
\put(1439.0,838.0){\rule[-0.200pt]{0.400pt}{4.818pt}}
\put(110.0,131.0){\rule[-0.200pt]{0.400pt}{175.134pt}}
\put(110.0,131.0){\rule[-0.200pt]{320.156pt}{0.400pt}}
\put(1439.0,131.0){\rule[-0.200pt]{0.400pt}{175.134pt}}
\put(110.0,858.0){\rule[-0.200pt]{320.156pt}{0.400pt}}
\put(774,29){\makebox(0,0){xi}}
\put(1279,817){\makebox(0,0)[r]{numerical solution}}
\put(1299.0,817.0){\rule[-0.200pt]{24.090pt}{0.400pt}}
\put(111,140){\usebox{\plotpoint}}
\put(111.17,140){\rule{0.400pt}{1.900pt}}
\multiput(110.17,140.00)(2.000,5.056){2}{\rule{0.400pt}{0.950pt}}
\put(112.67,149){\rule{0.400pt}{2.409pt}}
\multiput(112.17,149.00)(1.000,5.000){2}{\rule{0.400pt}{1.204pt}}
\put(113.67,159){\rule{0.400pt}{2.168pt}}
\multiput(113.17,159.00)(1.000,4.500){2}{\rule{0.400pt}{1.084pt}}
\put(115.17,168){\rule{0.400pt}{1.700pt}}
\multiput(114.17,168.00)(2.000,4.472){2}{\rule{0.400pt}{0.850pt}}
\put(116.67,176){\rule{0.400pt}{2.168pt}}
\multiput(116.17,176.00)(1.000,4.500){2}{\rule{0.400pt}{1.084pt}}
\put(117.67,185){\rule{0.400pt}{2.168pt}}
\multiput(117.17,185.00)(1.000,4.500){2}{\rule{0.400pt}{1.084pt}}
\put(119.17,194){\rule{0.400pt}{1.700pt}}
\multiput(118.17,194.00)(2.000,4.472){2}{\rule{0.400pt}{0.850pt}}
\put(120.67,202){\rule{0.400pt}{2.168pt}}
\multiput(120.17,202.00)(1.000,4.500){2}{\rule{0.400pt}{1.084pt}}
\put(121.67,211){\rule{0.400pt}{1.927pt}}
\multiput(121.17,211.00)(1.000,4.000){2}{\rule{0.400pt}{0.964pt}}
\put(123.17,219){\rule{0.400pt}{1.900pt}}
\multiput(122.17,219.00)(2.000,5.056){2}{\rule{0.400pt}{0.950pt}}
\put(124.67,228){\rule{0.400pt}{1.927pt}}
\multiput(124.17,228.00)(1.000,4.000){2}{\rule{0.400pt}{0.964pt}}
\put(125.67,236){\rule{0.400pt}{1.927pt}}
\multiput(125.17,236.00)(1.000,4.000){2}{\rule{0.400pt}{0.964pt}}
\put(127.17,244){\rule{0.400pt}{1.700pt}}
\multiput(126.17,244.00)(2.000,4.472){2}{\rule{0.400pt}{0.850pt}}
\put(128.67,252){\rule{0.400pt}{1.927pt}}
\multiput(128.17,252.00)(1.000,4.000){2}{\rule{0.400pt}{0.964pt}}
\put(129.67,260){\rule{0.400pt}{1.927pt}}
\multiput(129.17,260.00)(1.000,4.000){2}{\rule{0.400pt}{0.964pt}}
\put(131.17,268){\rule{0.400pt}{1.700pt}}
\multiput(130.17,268.00)(2.000,4.472){2}{\rule{0.400pt}{0.850pt}}
\put(132.67,276){\rule{0.400pt}{1.686pt}}
\multiput(132.17,276.00)(1.000,3.500){2}{\rule{0.400pt}{0.843pt}}
\put(133.67,283){\rule{0.400pt}{1.927pt}}
\multiput(133.17,283.00)(1.000,4.000){2}{\rule{0.400pt}{0.964pt}}
\put(135.17,291){\rule{0.400pt}{1.500pt}}
\multiput(134.17,291.00)(2.000,3.887){2}{\rule{0.400pt}{0.750pt}}
\put(136.67,298){\rule{0.400pt}{1.927pt}}
\multiput(136.17,298.00)(1.000,4.000){2}{\rule{0.400pt}{0.964pt}}
\put(137.67,306){\rule{0.400pt}{1.686pt}}
\multiput(137.17,306.00)(1.000,3.500){2}{\rule{0.400pt}{0.843pt}}
\put(139.17,313){\rule{0.400pt}{1.500pt}}
\multiput(138.17,313.00)(2.000,3.887){2}{\rule{0.400pt}{0.750pt}}
\put(140.67,320){\rule{0.400pt}{1.927pt}}
\multiput(140.17,320.00)(1.000,4.000){2}{\rule{0.400pt}{0.964pt}}
\put(141.67,328){\rule{0.400pt}{1.686pt}}
\multiput(141.17,328.00)(1.000,3.500){2}{\rule{0.400pt}{0.843pt}}
\put(143.17,335){\rule{0.400pt}{1.500pt}}
\multiput(142.17,335.00)(2.000,3.887){2}{\rule{0.400pt}{0.750pt}}
\put(144.67,342){\rule{0.400pt}{1.686pt}}
\multiput(144.17,342.00)(1.000,3.500){2}{\rule{0.400pt}{0.843pt}}
\put(145.67,349){\rule{0.400pt}{1.445pt}}
\multiput(145.17,349.00)(1.000,3.000){2}{\rule{0.400pt}{0.723pt}}
\put(147.17,355){\rule{0.400pt}{1.500pt}}
\multiput(146.17,355.00)(2.000,3.887){2}{\rule{0.400pt}{0.750pt}}
\put(148.67,362){\rule{0.400pt}{1.686pt}}
\multiput(148.17,362.00)(1.000,3.500){2}{\rule{0.400pt}{0.843pt}}
\put(149.67,369){\rule{0.400pt}{1.445pt}}
\multiput(149.17,369.00)(1.000,3.000){2}{\rule{0.400pt}{0.723pt}}
\put(150.67,375){\rule{0.400pt}{1.686pt}}
\multiput(150.17,375.00)(1.000,3.500){2}{\rule{0.400pt}{0.843pt}}
\put(152.17,382){\rule{0.400pt}{1.300pt}}
\multiput(151.17,382.00)(2.000,3.302){2}{\rule{0.400pt}{0.650pt}}
\put(153.67,388){\rule{0.400pt}{1.686pt}}
\multiput(153.17,388.00)(1.000,3.500){2}{\rule{0.400pt}{0.843pt}}
\put(154.67,395){\rule{0.400pt}{1.445pt}}
\multiput(154.17,395.00)(1.000,3.000){2}{\rule{0.400pt}{0.723pt}}
\put(156.17,401){\rule{0.400pt}{1.300pt}}
\multiput(155.17,401.00)(2.000,3.302){2}{\rule{0.400pt}{0.650pt}}
\put(157.67,407){\rule{0.400pt}{1.686pt}}
\multiput(157.17,407.00)(1.000,3.500){2}{\rule{0.400pt}{0.843pt}}
\put(158.67,414){\rule{0.400pt}{1.445pt}}
\multiput(158.17,414.00)(1.000,3.000){2}{\rule{0.400pt}{0.723pt}}
\put(160.17,420){\rule{0.400pt}{1.300pt}}
\multiput(159.17,420.00)(2.000,3.302){2}{\rule{0.400pt}{0.650pt}}
\put(161.67,426){\rule{0.400pt}{1.445pt}}
\multiput(161.17,426.00)(1.000,3.000){2}{\rule{0.400pt}{0.723pt}}
\put(162.67,432){\rule{0.400pt}{1.445pt}}
\multiput(162.17,432.00)(1.000,3.000){2}{\rule{0.400pt}{0.723pt}}
\put(164.17,438){\rule{0.400pt}{1.100pt}}
\multiput(163.17,438.00)(2.000,2.717){2}{\rule{0.400pt}{0.550pt}}
\put(165.67,443){\rule{0.400pt}{1.445pt}}
\multiput(165.17,443.00)(1.000,3.000){2}{\rule{0.400pt}{0.723pt}}
\put(166.67,449){\rule{0.400pt}{1.445pt}}
\multiput(166.17,449.00)(1.000,3.000){2}{\rule{0.400pt}{0.723pt}}
\put(168.17,455){\rule{0.400pt}{1.100pt}}
\multiput(167.17,455.00)(2.000,2.717){2}{\rule{0.400pt}{0.550pt}}
\put(169.67,460){\rule{0.400pt}{1.445pt}}
\multiput(169.17,460.00)(1.000,3.000){2}{\rule{0.400pt}{0.723pt}}
\put(170.67,466){\rule{0.400pt}{1.204pt}}
\multiput(170.17,466.00)(1.000,2.500){2}{\rule{0.400pt}{0.602pt}}
\put(172.17,471){\rule{0.400pt}{1.300pt}}
\multiput(171.17,471.00)(2.000,3.302){2}{\rule{0.400pt}{0.650pt}}
\put(173.67,477){\rule{0.400pt}{1.204pt}}
\multiput(173.17,477.00)(1.000,2.500){2}{\rule{0.400pt}{0.602pt}}
\put(174.67,482){\rule{0.400pt}{1.204pt}}
\multiput(174.17,482.00)(1.000,2.500){2}{\rule{0.400pt}{0.602pt}}
\put(176.17,487){\rule{0.400pt}{1.300pt}}
\multiput(175.17,487.00)(2.000,3.302){2}{\rule{0.400pt}{0.650pt}}
\put(177.67,493){\rule{0.400pt}{1.204pt}}
\multiput(177.17,493.00)(1.000,2.500){2}{\rule{0.400pt}{0.602pt}}
\put(178.67,498){\rule{0.400pt}{1.204pt}}
\multiput(178.17,498.00)(1.000,2.500){2}{\rule{0.400pt}{0.602pt}}
\put(180.17,503){\rule{0.400pt}{1.100pt}}
\multiput(179.17,503.00)(2.000,2.717){2}{\rule{0.400pt}{0.550pt}}
\put(181.67,508){\rule{0.400pt}{1.204pt}}
\multiput(181.17,508.00)(1.000,2.500){2}{\rule{0.400pt}{0.602pt}}
\put(182.67,513){\rule{0.400pt}{1.204pt}}
\multiput(182.17,513.00)(1.000,2.500){2}{\rule{0.400pt}{0.602pt}}
\put(184.17,518){\rule{0.400pt}{1.100pt}}
\multiput(183.17,518.00)(2.000,2.717){2}{\rule{0.400pt}{0.550pt}}
\put(185.67,523){\rule{0.400pt}{1.204pt}}
\multiput(185.17,523.00)(1.000,2.500){2}{\rule{0.400pt}{0.602pt}}
\put(186.67,528){\rule{0.400pt}{0.964pt}}
\multiput(186.17,528.00)(1.000,2.000){2}{\rule{0.400pt}{0.482pt}}
\put(188.17,532){\rule{0.400pt}{1.100pt}}
\multiput(187.17,532.00)(2.000,2.717){2}{\rule{0.400pt}{0.550pt}}
\put(189.67,537){\rule{0.400pt}{1.204pt}}
\multiput(189.17,537.00)(1.000,2.500){2}{\rule{0.400pt}{0.602pt}}
\put(190.67,542){\rule{0.400pt}{0.964pt}}
\multiput(190.17,542.00)(1.000,2.000){2}{\rule{0.400pt}{0.482pt}}
\put(192.17,546){\rule{0.400pt}{1.100pt}}
\multiput(191.17,546.00)(2.000,2.717){2}{\rule{0.400pt}{0.550pt}}
\put(193.67,551){\rule{0.400pt}{0.964pt}}
\multiput(193.17,551.00)(1.000,2.000){2}{\rule{0.400pt}{0.482pt}}
\put(194.67,555){\rule{0.400pt}{1.204pt}}
\multiput(194.17,555.00)(1.000,2.500){2}{\rule{0.400pt}{0.602pt}}
\put(196.17,560){\rule{0.400pt}{0.900pt}}
\multiput(195.17,560.00)(2.000,2.132){2}{\rule{0.400pt}{0.450pt}}
\put(197.67,564){\rule{0.400pt}{0.964pt}}
\multiput(197.17,564.00)(1.000,2.000){2}{\rule{0.400pt}{0.482pt}}
\put(198.67,568){\rule{0.400pt}{1.204pt}}
\multiput(198.17,568.00)(1.000,2.500){2}{\rule{0.400pt}{0.602pt}}
\put(200.17,573){\rule{0.400pt}{0.900pt}}
\multiput(199.17,573.00)(2.000,2.132){2}{\rule{0.400pt}{0.450pt}}
\put(201.67,577){\rule{0.400pt}{0.964pt}}
\multiput(201.17,577.00)(1.000,2.000){2}{\rule{0.400pt}{0.482pt}}
\put(202.67,581){\rule{0.400pt}{0.964pt}}
\multiput(202.17,581.00)(1.000,2.000){2}{\rule{0.400pt}{0.482pt}}
\put(204.17,585){\rule{0.400pt}{0.900pt}}
\multiput(203.17,585.00)(2.000,2.132){2}{\rule{0.400pt}{0.450pt}}
\put(205.67,589){\rule{0.400pt}{0.964pt}}
\multiput(205.17,589.00)(1.000,2.000){2}{\rule{0.400pt}{0.482pt}}
\put(206.67,593){\rule{0.400pt}{0.964pt}}
\multiput(206.17,593.00)(1.000,2.000){2}{\rule{0.400pt}{0.482pt}}
\put(208.17,597){\rule{0.400pt}{0.900pt}}
\multiput(207.17,597.00)(2.000,2.132){2}{\rule{0.400pt}{0.450pt}}
\put(209.67,601){\rule{0.400pt}{0.964pt}}
\multiput(209.17,601.00)(1.000,2.000){2}{\rule{0.400pt}{0.482pt}}
\put(210.67,605){\rule{0.400pt}{0.723pt}}
\multiput(210.17,605.00)(1.000,1.500){2}{\rule{0.400pt}{0.361pt}}
\put(212.17,608){\rule{0.400pt}{0.900pt}}
\multiput(211.17,608.00)(2.000,2.132){2}{\rule{0.400pt}{0.450pt}}
\put(213.67,612){\rule{0.400pt}{0.964pt}}
\multiput(213.17,612.00)(1.000,2.000){2}{\rule{0.400pt}{0.482pt}}
\put(214.67,616){\rule{0.400pt}{0.964pt}}
\multiput(214.17,616.00)(1.000,2.000){2}{\rule{0.400pt}{0.482pt}}
\put(216.17,620){\rule{0.400pt}{0.700pt}}
\multiput(215.17,620.00)(2.000,1.547){2}{\rule{0.400pt}{0.350pt}}
\put(217.67,623){\rule{0.400pt}{0.964pt}}
\multiput(217.17,623.00)(1.000,2.000){2}{\rule{0.400pt}{0.482pt}}
\put(218.67,627){\rule{0.400pt}{0.723pt}}
\multiput(218.17,627.00)(1.000,1.500){2}{\rule{0.400pt}{0.361pt}}
\put(220.17,630){\rule{0.400pt}{0.900pt}}
\multiput(219.17,630.00)(2.000,2.132){2}{\rule{0.400pt}{0.450pt}}
\put(221.67,634){\rule{0.400pt}{0.723pt}}
\multiput(221.17,634.00)(1.000,1.500){2}{\rule{0.400pt}{0.361pt}}
\put(222.67,637){\rule{0.400pt}{0.723pt}}
\multiput(222.17,637.00)(1.000,1.500){2}{\rule{0.400pt}{0.361pt}}
\put(224.17,640){\rule{0.400pt}{0.900pt}}
\multiput(223.17,640.00)(2.000,2.132){2}{\rule{0.400pt}{0.450pt}}
\put(225.67,644){\rule{0.400pt}{0.723pt}}
\multiput(225.17,644.00)(1.000,1.500){2}{\rule{0.400pt}{0.361pt}}
\put(226.67,647){\rule{0.400pt}{0.723pt}}
\multiput(226.17,647.00)(1.000,1.500){2}{\rule{0.400pt}{0.361pt}}
\put(227.67,650){\rule{0.400pt}{0.964pt}}
\multiput(227.17,650.00)(1.000,2.000){2}{\rule{0.400pt}{0.482pt}}
\put(229.17,654){\rule{0.400pt}{0.700pt}}
\multiput(228.17,654.00)(2.000,1.547){2}{\rule{0.400pt}{0.350pt}}
\put(230.67,657){\rule{0.400pt}{0.723pt}}
\multiput(230.17,657.00)(1.000,1.500){2}{\rule{0.400pt}{0.361pt}}
\put(231.67,660){\rule{0.400pt}{0.723pt}}
\multiput(231.17,660.00)(1.000,1.500){2}{\rule{0.400pt}{0.361pt}}
\put(233.17,663){\rule{0.400pt}{0.700pt}}
\multiput(232.17,663.00)(2.000,1.547){2}{\rule{0.400pt}{0.350pt}}
\put(234.67,666){\rule{0.400pt}{0.723pt}}
\multiput(234.17,666.00)(1.000,1.500){2}{\rule{0.400pt}{0.361pt}}
\put(235.67,669){\rule{0.400pt}{0.723pt}}
\multiput(235.17,669.00)(1.000,1.500){2}{\rule{0.400pt}{0.361pt}}
\put(237.17,672){\rule{0.400pt}{0.700pt}}
\multiput(236.17,672.00)(2.000,1.547){2}{\rule{0.400pt}{0.350pt}}
\put(238.67,675){\rule{0.400pt}{0.723pt}}
\multiput(238.17,675.00)(1.000,1.500){2}{\rule{0.400pt}{0.361pt}}
\put(239.67,678){\rule{0.400pt}{0.723pt}}
\multiput(239.17,678.00)(1.000,1.500){2}{\rule{0.400pt}{0.361pt}}
\put(241,681.17){\rule{0.482pt}{0.400pt}}
\multiput(241.00,680.17)(1.000,2.000){2}{\rule{0.241pt}{0.400pt}}
\put(242.67,683){\rule{0.400pt}{0.723pt}}
\multiput(242.17,683.00)(1.000,1.500){2}{\rule{0.400pt}{0.361pt}}
\put(243.67,686){\rule{0.400pt}{0.723pt}}
\multiput(243.17,686.00)(1.000,1.500){2}{\rule{0.400pt}{0.361pt}}
\put(245.17,689){\rule{0.400pt}{0.700pt}}
\multiput(244.17,689.00)(2.000,1.547){2}{\rule{0.400pt}{0.350pt}}
\put(246.67,692){\rule{0.400pt}{0.482pt}}
\multiput(246.17,692.00)(1.000,1.000){2}{\rule{0.400pt}{0.241pt}}
\put(247.67,694){\rule{0.400pt}{0.723pt}}
\multiput(247.17,694.00)(1.000,1.500){2}{\rule{0.400pt}{0.361pt}}
\put(249,697.17){\rule{0.482pt}{0.400pt}}
\multiput(249.00,696.17)(1.000,2.000){2}{\rule{0.241pt}{0.400pt}}
\put(250.67,699){\rule{0.400pt}{0.723pt}}
\multiput(250.17,699.00)(1.000,1.500){2}{\rule{0.400pt}{0.361pt}}
\put(251.67,702){\rule{0.400pt}{0.482pt}}
\multiput(251.17,702.00)(1.000,1.000){2}{\rule{0.400pt}{0.241pt}}
\put(253.17,704){\rule{0.400pt}{0.700pt}}
\multiput(252.17,704.00)(2.000,1.547){2}{\rule{0.400pt}{0.350pt}}
\put(254.67,707){\rule{0.400pt}{0.482pt}}
\multiput(254.17,707.00)(1.000,1.000){2}{\rule{0.400pt}{0.241pt}}
\put(255.67,709){\rule{0.400pt}{0.723pt}}
\multiput(255.17,709.00)(1.000,1.500){2}{\rule{0.400pt}{0.361pt}}
\put(257,712.17){\rule{0.482pt}{0.400pt}}
\multiput(257.00,711.17)(1.000,2.000){2}{\rule{0.241pt}{0.400pt}}
\put(258.67,714){\rule{0.400pt}{0.482pt}}
\multiput(258.17,714.00)(1.000,1.000){2}{\rule{0.400pt}{0.241pt}}
\put(259.67,716){\rule{0.400pt}{0.723pt}}
\multiput(259.17,716.00)(1.000,1.500){2}{\rule{0.400pt}{0.361pt}}
\put(261,719.17){\rule{0.482pt}{0.400pt}}
\multiput(261.00,718.17)(1.000,2.000){2}{\rule{0.241pt}{0.400pt}}
\put(262.67,721){\rule{0.400pt}{0.482pt}}
\multiput(262.17,721.00)(1.000,1.000){2}{\rule{0.400pt}{0.241pt}}
\put(263.67,723){\rule{0.400pt}{0.482pt}}
\multiput(263.17,723.00)(1.000,1.000){2}{\rule{0.400pt}{0.241pt}}
\put(265.17,725){\rule{0.400pt}{0.700pt}}
\multiput(264.17,725.00)(2.000,1.547){2}{\rule{0.400pt}{0.350pt}}
\put(266.67,728){\rule{0.400pt}{0.482pt}}
\multiput(266.17,728.00)(1.000,1.000){2}{\rule{0.400pt}{0.241pt}}
\put(267.67,730){\rule{0.400pt}{0.482pt}}
\multiput(267.17,730.00)(1.000,1.000){2}{\rule{0.400pt}{0.241pt}}
\put(269,732.17){\rule{0.482pt}{0.400pt}}
\multiput(269.00,731.17)(1.000,2.000){2}{\rule{0.241pt}{0.400pt}}
\put(270.67,734){\rule{0.400pt}{0.482pt}}
\multiput(270.17,734.00)(1.000,1.000){2}{\rule{0.400pt}{0.241pt}}
\put(271.67,736){\rule{0.400pt}{0.482pt}}
\multiput(271.17,736.00)(1.000,1.000){2}{\rule{0.400pt}{0.241pt}}
\put(273,738.17){\rule{0.482pt}{0.400pt}}
\multiput(273.00,737.17)(1.000,2.000){2}{\rule{0.241pt}{0.400pt}}
\put(274.67,740){\rule{0.400pt}{0.482pt}}
\multiput(274.17,740.00)(1.000,1.000){2}{\rule{0.400pt}{0.241pt}}
\put(275.67,742){\rule{0.400pt}{0.482pt}}
\multiput(275.17,742.00)(1.000,1.000){2}{\rule{0.400pt}{0.241pt}}
\put(277,744.17){\rule{0.482pt}{0.400pt}}
\multiput(277.00,743.17)(1.000,2.000){2}{\rule{0.241pt}{0.400pt}}
\put(278.67,746){\rule{0.400pt}{0.482pt}}
\multiput(278.17,746.00)(1.000,1.000){2}{\rule{0.400pt}{0.241pt}}
\put(280,747.67){\rule{0.241pt}{0.400pt}}
\multiput(280.00,747.17)(0.500,1.000){2}{\rule{0.120pt}{0.400pt}}
\put(281,749.17){\rule{0.482pt}{0.400pt}}
\multiput(281.00,748.17)(1.000,2.000){2}{\rule{0.241pt}{0.400pt}}
\put(282.67,751){\rule{0.400pt}{0.482pt}}
\multiput(282.17,751.00)(1.000,1.000){2}{\rule{0.400pt}{0.241pt}}
\put(283.67,753){\rule{0.400pt}{0.482pt}}
\multiput(283.17,753.00)(1.000,1.000){2}{\rule{0.400pt}{0.241pt}}
\put(285,755.17){\rule{0.482pt}{0.400pt}}
\multiput(285.00,754.17)(1.000,2.000){2}{\rule{0.241pt}{0.400pt}}
\put(287,756.67){\rule{0.241pt}{0.400pt}}
\multiput(287.00,756.17)(0.500,1.000){2}{\rule{0.120pt}{0.400pt}}
\put(287.67,758){\rule{0.400pt}{0.482pt}}
\multiput(287.17,758.00)(1.000,1.000){2}{\rule{0.400pt}{0.241pt}}
\put(289,760.17){\rule{0.482pt}{0.400pt}}
\multiput(289.00,759.17)(1.000,2.000){2}{\rule{0.241pt}{0.400pt}}
\put(291,761.67){\rule{0.241pt}{0.400pt}}
\multiput(291.00,761.17)(0.500,1.000){2}{\rule{0.120pt}{0.400pt}}
\put(291.67,763){\rule{0.400pt}{0.482pt}}
\multiput(291.17,763.00)(1.000,1.000){2}{\rule{0.400pt}{0.241pt}}
\put(293,764.67){\rule{0.482pt}{0.400pt}}
\multiput(293.00,764.17)(1.000,1.000){2}{\rule{0.241pt}{0.400pt}}
\put(294.67,766){\rule{0.400pt}{0.482pt}}
\multiput(294.17,766.00)(1.000,1.000){2}{\rule{0.400pt}{0.241pt}}
\put(296,767.67){\rule{0.241pt}{0.400pt}}
\multiput(296.00,767.17)(0.500,1.000){2}{\rule{0.120pt}{0.400pt}}
\put(297,769.17){\rule{0.482pt}{0.400pt}}
\multiput(297.00,768.17)(1.000,2.000){2}{\rule{0.241pt}{0.400pt}}
\put(299,770.67){\rule{0.241pt}{0.400pt}}
\multiput(299.00,770.17)(0.500,1.000){2}{\rule{0.120pt}{0.400pt}}
\put(299.67,772){\rule{0.400pt}{0.482pt}}
\multiput(299.17,772.00)(1.000,1.000){2}{\rule{0.400pt}{0.241pt}}
\put(301,773.67){\rule{0.482pt}{0.400pt}}
\multiput(301.00,773.17)(1.000,1.000){2}{\rule{0.241pt}{0.400pt}}
\put(302.67,775){\rule{0.400pt}{0.482pt}}
\multiput(302.17,775.00)(1.000,1.000){2}{\rule{0.400pt}{0.241pt}}
\put(304,776.67){\rule{0.241pt}{0.400pt}}
\multiput(304.00,776.17)(0.500,1.000){2}{\rule{0.120pt}{0.400pt}}
\put(305,777.67){\rule{0.241pt}{0.400pt}}
\multiput(305.00,777.17)(0.500,1.000){2}{\rule{0.120pt}{0.400pt}}
\put(306,779.17){\rule{0.482pt}{0.400pt}}
\multiput(306.00,778.17)(1.000,2.000){2}{\rule{0.241pt}{0.400pt}}
\put(308,780.67){\rule{0.241pt}{0.400pt}}
\multiput(308.00,780.17)(0.500,1.000){2}{\rule{0.120pt}{0.400pt}}
\put(309,781.67){\rule{0.241pt}{0.400pt}}
\multiput(309.00,781.17)(0.500,1.000){2}{\rule{0.120pt}{0.400pt}}
\put(310,782.67){\rule{0.482pt}{0.400pt}}
\multiput(310.00,782.17)(1.000,1.000){2}{\rule{0.241pt}{0.400pt}}
\put(311.67,784){\rule{0.400pt}{0.482pt}}
\multiput(311.17,784.00)(1.000,1.000){2}{\rule{0.400pt}{0.241pt}}
\put(313,785.67){\rule{0.241pt}{0.400pt}}
\multiput(313.00,785.17)(0.500,1.000){2}{\rule{0.120pt}{0.400pt}}
\put(314,786.67){\rule{0.482pt}{0.400pt}}
\multiput(314.00,786.17)(1.000,1.000){2}{\rule{0.241pt}{0.400pt}}
\put(316,787.67){\rule{0.241pt}{0.400pt}}
\multiput(316.00,787.17)(0.500,1.000){2}{\rule{0.120pt}{0.400pt}}
\put(317,788.67){\rule{0.241pt}{0.400pt}}
\multiput(317.00,788.17)(0.500,1.000){2}{\rule{0.120pt}{0.400pt}}
\put(318,789.67){\rule{0.482pt}{0.400pt}}
\multiput(318.00,789.17)(1.000,1.000){2}{\rule{0.241pt}{0.400pt}}
\put(320,790.67){\rule{0.241pt}{0.400pt}}
\multiput(320.00,790.17)(0.500,1.000){2}{\rule{0.120pt}{0.400pt}}
\put(320.67,792){\rule{0.400pt}{0.482pt}}
\multiput(320.17,792.00)(1.000,1.000){2}{\rule{0.400pt}{0.241pt}}
\put(322,793.67){\rule{0.482pt}{0.400pt}}
\multiput(322.00,793.17)(1.000,1.000){2}{\rule{0.241pt}{0.400pt}}
\put(324,794.67){\rule{0.241pt}{0.400pt}}
\multiput(324.00,794.17)(0.500,1.000){2}{\rule{0.120pt}{0.400pt}}
\put(325,795.67){\rule{0.241pt}{0.400pt}}
\multiput(325.00,795.17)(0.500,1.000){2}{\rule{0.120pt}{0.400pt}}
\put(326,796.67){\rule{0.482pt}{0.400pt}}
\multiput(326.00,796.17)(1.000,1.000){2}{\rule{0.241pt}{0.400pt}}
\put(328,797.67){\rule{0.241pt}{0.400pt}}
\multiput(328.00,797.17)(0.500,1.000){2}{\rule{0.120pt}{0.400pt}}
\put(329,798.67){\rule{0.241pt}{0.400pt}}
\multiput(329.00,798.17)(0.500,1.000){2}{\rule{0.120pt}{0.400pt}}
\put(332,799.67){\rule{0.241pt}{0.400pt}}
\multiput(332.00,799.17)(0.500,1.000){2}{\rule{0.120pt}{0.400pt}}
\put(333,800.67){\rule{0.241pt}{0.400pt}}
\multiput(333.00,800.17)(0.500,1.000){2}{\rule{0.120pt}{0.400pt}}
\put(334,801.67){\rule{0.482pt}{0.400pt}}
\multiput(334.00,801.17)(1.000,1.000){2}{\rule{0.241pt}{0.400pt}}
\put(336,802.67){\rule{0.241pt}{0.400pt}}
\multiput(336.00,802.17)(0.500,1.000){2}{\rule{0.120pt}{0.400pt}}
\put(337,803.67){\rule{0.241pt}{0.400pt}}
\multiput(337.00,803.17)(0.500,1.000){2}{\rule{0.120pt}{0.400pt}}
\put(338,804.67){\rule{0.482pt}{0.400pt}}
\multiput(338.00,804.17)(1.000,1.000){2}{\rule{0.241pt}{0.400pt}}
\put(330.0,800.0){\rule[-0.200pt]{0.482pt}{0.400pt}}
\put(341,805.67){\rule{0.241pt}{0.400pt}}
\multiput(341.00,805.17)(0.500,1.000){2}{\rule{0.120pt}{0.400pt}}
\put(342,806.67){\rule{0.482pt}{0.400pt}}
\multiput(342.00,806.17)(1.000,1.000){2}{\rule{0.241pt}{0.400pt}}
\put(344,807.67){\rule{0.241pt}{0.400pt}}
\multiput(344.00,807.17)(0.500,1.000){2}{\rule{0.120pt}{0.400pt}}
\put(340.0,806.0){\usebox{\plotpoint}}
\put(346,808.67){\rule{0.482pt}{0.400pt}}
\multiput(346.00,808.17)(1.000,1.000){2}{\rule{0.241pt}{0.400pt}}
\put(348,809.67){\rule{0.241pt}{0.400pt}}
\multiput(348.00,809.17)(0.500,1.000){2}{\rule{0.120pt}{0.400pt}}
\put(349,810.67){\rule{0.241pt}{0.400pt}}
\multiput(349.00,810.17)(0.500,1.000){2}{\rule{0.120pt}{0.400pt}}
\put(345.0,809.0){\usebox{\plotpoint}}
\put(352,811.67){\rule{0.241pt}{0.400pt}}
\multiput(352.00,811.17)(0.500,1.000){2}{\rule{0.120pt}{0.400pt}}
\put(350.0,812.0){\rule[-0.200pt]{0.482pt}{0.400pt}}
\put(354,812.67){\rule{0.482pt}{0.400pt}}
\multiput(354.00,812.17)(1.000,1.000){2}{\rule{0.241pt}{0.400pt}}
\put(356,813.67){\rule{0.241pt}{0.400pt}}
\multiput(356.00,813.17)(0.500,1.000){2}{\rule{0.120pt}{0.400pt}}
\put(353.0,813.0){\usebox{\plotpoint}}
\put(358,814.67){\rule{0.482pt}{0.400pt}}
\multiput(358.00,814.17)(1.000,1.000){2}{\rule{0.241pt}{0.400pt}}
\put(357.0,815.0){\usebox{\plotpoint}}
\put(361,815.67){\rule{0.241pt}{0.400pt}}
\multiput(361.00,815.17)(0.500,1.000){2}{\rule{0.120pt}{0.400pt}}
\put(360.0,816.0){\usebox{\plotpoint}}
\put(364,816.67){\rule{0.241pt}{0.400pt}}
\multiput(364.00,816.17)(0.500,1.000){2}{\rule{0.120pt}{0.400pt}}
\put(362.0,817.0){\rule[-0.200pt]{0.482pt}{0.400pt}}
\put(366,817.67){\rule{0.482pt}{0.400pt}}
\multiput(366.00,817.17)(1.000,1.000){2}{\rule{0.241pt}{0.400pt}}
\put(365.0,818.0){\usebox{\plotpoint}}
\put(369,818.67){\rule{0.241pt}{0.400pt}}
\multiput(369.00,818.17)(0.500,1.000){2}{\rule{0.120pt}{0.400pt}}
\put(368.0,819.0){\usebox{\plotpoint}}
\put(373,819.67){\rule{0.241pt}{0.400pt}}
\multiput(373.00,819.17)(0.500,1.000){2}{\rule{0.120pt}{0.400pt}}
\put(370.0,820.0){\rule[-0.200pt]{0.723pt}{0.400pt}}
\put(376,820.67){\rule{0.241pt}{0.400pt}}
\multiput(376.00,820.17)(0.500,1.000){2}{\rule{0.120pt}{0.400pt}}
\put(374.0,821.0){\rule[-0.200pt]{0.482pt}{0.400pt}}
\put(380,821.67){\rule{0.241pt}{0.400pt}}
\multiput(380.00,821.17)(0.500,1.000){2}{\rule{0.120pt}{0.400pt}}
\put(377.0,822.0){\rule[-0.200pt]{0.723pt}{0.400pt}}
\put(385,822.67){\rule{0.241pt}{0.400pt}}
\multiput(385.00,822.17)(0.500,1.000){2}{\rule{0.120pt}{0.400pt}}
\put(381.0,823.0){\rule[-0.200pt]{0.964pt}{0.400pt}}
\put(390,823.67){\rule{0.241pt}{0.400pt}}
\multiput(390.00,823.17)(0.500,1.000){2}{\rule{0.120pt}{0.400pt}}
\put(386.0,824.0){\rule[-0.200pt]{0.964pt}{0.400pt}}
\put(397,824.67){\rule{0.241pt}{0.400pt}}
\multiput(397.00,824.17)(0.500,1.000){2}{\rule{0.120pt}{0.400pt}}
\put(391.0,825.0){\rule[-0.200pt]{1.445pt}{0.400pt}}
\put(410,825.67){\rule{0.241pt}{0.400pt}}
\multiput(410.00,825.17)(0.500,1.000){2}{\rule{0.120pt}{0.400pt}}
\put(398.0,826.0){\rule[-0.200pt]{2.891pt}{0.400pt}}
\put(421,825.67){\rule{0.241pt}{0.400pt}}
\multiput(421.00,826.17)(0.500,-1.000){2}{\rule{0.120pt}{0.400pt}}
\put(411.0,827.0){\rule[-0.200pt]{2.409pt}{0.400pt}}
\put(435,824.67){\rule{0.482pt}{0.400pt}}
\multiput(435.00,825.17)(1.000,-1.000){2}{\rule{0.241pt}{0.400pt}}
\put(422.0,826.0){\rule[-0.200pt]{3.132pt}{0.400pt}}
\put(443,823.67){\rule{0.482pt}{0.400pt}}
\multiput(443.00,824.17)(1.000,-1.000){2}{\rule{0.241pt}{0.400pt}}
\put(437.0,825.0){\rule[-0.200pt]{1.445pt}{0.400pt}}
\put(449,822.67){\rule{0.241pt}{0.400pt}}
\multiput(449.00,823.17)(0.500,-1.000){2}{\rule{0.120pt}{0.400pt}}
\put(445.0,824.0){\rule[-0.200pt]{0.964pt}{0.400pt}}
\put(454,821.67){\rule{0.241pt}{0.400pt}}
\multiput(454.00,822.17)(0.500,-1.000){2}{\rule{0.120pt}{0.400pt}}
\put(450.0,823.0){\rule[-0.200pt]{0.964pt}{0.400pt}}
\put(459,820.67){\rule{0.482pt}{0.400pt}}
\multiput(459.00,821.17)(1.000,-1.000){2}{\rule{0.241pt}{0.400pt}}
\put(455.0,822.0){\rule[-0.200pt]{0.964pt}{0.400pt}}
\put(463,819.67){\rule{0.241pt}{0.400pt}}
\multiput(463.00,820.17)(0.500,-1.000){2}{\rule{0.120pt}{0.400pt}}
\put(461.0,821.0){\rule[-0.200pt]{0.482pt}{0.400pt}}
\put(467,818.67){\rule{0.241pt}{0.400pt}}
\multiput(467.00,819.17)(0.500,-1.000){2}{\rule{0.120pt}{0.400pt}}
\put(464.0,820.0){\rule[-0.200pt]{0.723pt}{0.400pt}}
\put(471,817.67){\rule{0.241pt}{0.400pt}}
\multiput(471.00,818.17)(0.500,-1.000){2}{\rule{0.120pt}{0.400pt}}
\put(468.0,819.0){\rule[-0.200pt]{0.723pt}{0.400pt}}
\put(475,816.67){\rule{0.241pt}{0.400pt}}
\multiput(475.00,817.17)(0.500,-1.000){2}{\rule{0.120pt}{0.400pt}}
\put(472.0,818.0){\rule[-0.200pt]{0.723pt}{0.400pt}}
\put(479,815.67){\rule{0.241pt}{0.400pt}}
\multiput(479.00,816.17)(0.500,-1.000){2}{\rule{0.120pt}{0.400pt}}
\put(476.0,817.0){\rule[-0.200pt]{0.723pt}{0.400pt}}
\put(482,814.67){\rule{0.241pt}{0.400pt}}
\multiput(482.00,815.17)(0.500,-1.000){2}{\rule{0.120pt}{0.400pt}}
\put(480.0,816.0){\rule[-0.200pt]{0.482pt}{0.400pt}}
\put(484,813.67){\rule{0.482pt}{0.400pt}}
\multiput(484.00,814.17)(1.000,-1.000){2}{\rule{0.241pt}{0.400pt}}
\put(483.0,815.0){\usebox{\plotpoint}}
\put(488,812.67){\rule{0.482pt}{0.400pt}}
\multiput(488.00,813.17)(1.000,-1.000){2}{\rule{0.241pt}{0.400pt}}
\put(486.0,814.0){\rule[-0.200pt]{0.482pt}{0.400pt}}
\put(491,811.67){\rule{0.241pt}{0.400pt}}
\multiput(491.00,812.17)(0.500,-1.000){2}{\rule{0.120pt}{0.400pt}}
\put(490.0,813.0){\usebox{\plotpoint}}
\put(494,810.67){\rule{0.241pt}{0.400pt}}
\multiput(494.00,811.17)(0.500,-1.000){2}{\rule{0.120pt}{0.400pt}}
\put(492.0,812.0){\rule[-0.200pt]{0.482pt}{0.400pt}}
\put(496,809.67){\rule{0.482pt}{0.400pt}}
\multiput(496.00,810.17)(1.000,-1.000){2}{\rule{0.241pt}{0.400pt}}
\put(495.0,811.0){\usebox{\plotpoint}}
\put(499,808.67){\rule{0.241pt}{0.400pt}}
\multiput(499.00,809.17)(0.500,-1.000){2}{\rule{0.120pt}{0.400pt}}
\put(498.0,810.0){\usebox{\plotpoint}}
\put(502,807.67){\rule{0.241pt}{0.400pt}}
\multiput(502.00,808.17)(0.500,-1.000){2}{\rule{0.120pt}{0.400pt}}
\put(500.0,809.0){\rule[-0.200pt]{0.482pt}{0.400pt}}
\put(504,806.67){\rule{0.482pt}{0.400pt}}
\multiput(504.00,807.17)(1.000,-1.000){2}{\rule{0.241pt}{0.400pt}}
\put(503.0,808.0){\usebox{\plotpoint}}
\put(507,805.67){\rule{0.241pt}{0.400pt}}
\multiput(507.00,806.17)(0.500,-1.000){2}{\rule{0.120pt}{0.400pt}}
\put(506.0,807.0){\usebox{\plotpoint}}
\put(510,804.67){\rule{0.241pt}{0.400pt}}
\multiput(510.00,805.17)(0.500,-1.000){2}{\rule{0.120pt}{0.400pt}}
\put(508.0,806.0){\rule[-0.200pt]{0.482pt}{0.400pt}}
\put(512,803.67){\rule{0.482pt}{0.400pt}}
\multiput(512.00,804.17)(1.000,-1.000){2}{\rule{0.241pt}{0.400pt}}
\put(511.0,805.0){\usebox{\plotpoint}}
\put(515,802.67){\rule{0.241pt}{0.400pt}}
\multiput(515.00,803.17)(0.500,-1.000){2}{\rule{0.120pt}{0.400pt}}
\put(514.0,804.0){\usebox{\plotpoint}}
\put(518,801.67){\rule{0.241pt}{0.400pt}}
\multiput(518.00,802.17)(0.500,-1.000){2}{\rule{0.120pt}{0.400pt}}
\put(516.0,803.0){\rule[-0.200pt]{0.482pt}{0.400pt}}
\put(520,800.67){\rule{0.482pt}{0.400pt}}
\multiput(520.00,801.17)(1.000,-1.000){2}{\rule{0.241pt}{0.400pt}}
\put(522,799.67){\rule{0.241pt}{0.400pt}}
\multiput(522.00,800.17)(0.500,-1.000){2}{\rule{0.120pt}{0.400pt}}
\put(519.0,802.0){\usebox{\plotpoint}}
\put(524,798.67){\rule{0.482pt}{0.400pt}}
\multiput(524.00,799.17)(1.000,-1.000){2}{\rule{0.241pt}{0.400pt}}
\put(523.0,800.0){\usebox{\plotpoint}}
\put(527,797.67){\rule{0.241pt}{0.400pt}}
\multiput(527.00,798.17)(0.500,-1.000){2}{\rule{0.120pt}{0.400pt}}
\put(526.0,799.0){\usebox{\plotpoint}}
\put(530,796.67){\rule{0.241pt}{0.400pt}}
\multiput(530.00,797.17)(0.500,-1.000){2}{\rule{0.120pt}{0.400pt}}
\put(531,795.67){\rule{0.241pt}{0.400pt}}
\multiput(531.00,796.17)(0.500,-1.000){2}{\rule{0.120pt}{0.400pt}}
\put(528.0,798.0){\rule[-0.200pt]{0.482pt}{0.400pt}}
\put(534,794.67){\rule{0.241pt}{0.400pt}}
\multiput(534.00,795.17)(0.500,-1.000){2}{\rule{0.120pt}{0.400pt}}
\put(535,793.67){\rule{0.241pt}{0.400pt}}
\multiput(535.00,794.17)(0.500,-1.000){2}{\rule{0.120pt}{0.400pt}}
\put(532.0,796.0){\rule[-0.200pt]{0.482pt}{0.400pt}}
\put(538,792.67){\rule{0.241pt}{0.400pt}}
\multiput(538.00,793.17)(0.500,-1.000){2}{\rule{0.120pt}{0.400pt}}
\put(536.0,794.0){\rule[-0.200pt]{0.482pt}{0.400pt}}
\put(540,791.67){\rule{0.241pt}{0.400pt}}
\multiput(540.00,792.17)(0.500,-1.000){2}{\rule{0.120pt}{0.400pt}}
\put(541,790.67){\rule{0.482pt}{0.400pt}}
\multiput(541.00,791.17)(1.000,-1.000){2}{\rule{0.241pt}{0.400pt}}
\put(539.0,793.0){\usebox{\plotpoint}}
\put(544,789.67){\rule{0.241pt}{0.400pt}}
\multiput(544.00,790.17)(0.500,-1.000){2}{\rule{0.120pt}{0.400pt}}
\put(545,788.67){\rule{0.482pt}{0.400pt}}
\multiput(545.00,789.17)(1.000,-1.000){2}{\rule{0.241pt}{0.400pt}}
\put(543.0,791.0){\usebox{\plotpoint}}
\put(548,787.67){\rule{0.241pt}{0.400pt}}
\multiput(548.00,788.17)(0.500,-1.000){2}{\rule{0.120pt}{0.400pt}}
\put(549,786.67){\rule{0.482pt}{0.400pt}}
\multiput(549.00,787.17)(1.000,-1.000){2}{\rule{0.241pt}{0.400pt}}
\put(547.0,789.0){\usebox{\plotpoint}}
\put(552,785.67){\rule{0.241pt}{0.400pt}}
\multiput(552.00,786.17)(0.500,-1.000){2}{\rule{0.120pt}{0.400pt}}
\put(553,784.67){\rule{0.482pt}{0.400pt}}
\multiput(553.00,785.17)(1.000,-1.000){2}{\rule{0.241pt}{0.400pt}}
\put(551.0,787.0){\usebox{\plotpoint}}
\put(556,783.67){\rule{0.241pt}{0.400pt}}
\multiput(556.00,784.17)(0.500,-1.000){2}{\rule{0.120pt}{0.400pt}}
\put(557,782.67){\rule{0.482pt}{0.400pt}}
\multiput(557.00,783.17)(1.000,-1.000){2}{\rule{0.241pt}{0.400pt}}
\put(555.0,785.0){\usebox{\plotpoint}}
\put(560,781.67){\rule{0.241pt}{0.400pt}}
\multiput(560.00,782.17)(0.500,-1.000){2}{\rule{0.120pt}{0.400pt}}
\put(561,780.67){\rule{0.482pt}{0.400pt}}
\multiput(561.00,781.17)(1.000,-1.000){2}{\rule{0.241pt}{0.400pt}}
\put(559.0,783.0){\usebox{\plotpoint}}
\put(564,779.67){\rule{0.241pt}{0.400pt}}
\multiput(564.00,780.17)(0.500,-1.000){2}{\rule{0.120pt}{0.400pt}}
\put(565,778.67){\rule{0.482pt}{0.400pt}}
\multiput(565.00,779.17)(1.000,-1.000){2}{\rule{0.241pt}{0.400pt}}
\put(563.0,781.0){\usebox{\plotpoint}}
\put(568,777.67){\rule{0.241pt}{0.400pt}}
\multiput(568.00,778.17)(0.500,-1.000){2}{\rule{0.120pt}{0.400pt}}
\put(569,776.67){\rule{0.482pt}{0.400pt}}
\multiput(569.00,777.17)(1.000,-1.000){2}{\rule{0.241pt}{0.400pt}}
\put(571,775.67){\rule{0.241pt}{0.400pt}}
\multiput(571.00,776.17)(0.500,-1.000){2}{\rule{0.120pt}{0.400pt}}
\put(567.0,779.0){\usebox{\plotpoint}}
\put(573,774.67){\rule{0.482pt}{0.400pt}}
\multiput(573.00,775.17)(1.000,-1.000){2}{\rule{0.241pt}{0.400pt}}
\put(575,773.67){\rule{0.241pt}{0.400pt}}
\multiput(575.00,774.17)(0.500,-1.000){2}{\rule{0.120pt}{0.400pt}}
\put(572.0,776.0){\usebox{\plotpoint}}
\put(577,772.67){\rule{0.482pt}{0.400pt}}
\multiput(577.00,773.17)(1.000,-1.000){2}{\rule{0.241pt}{0.400pt}}
\put(579,771.67){\rule{0.241pt}{0.400pt}}
\multiput(579.00,772.17)(0.500,-1.000){2}{\rule{0.120pt}{0.400pt}}
\put(580,770.67){\rule{0.241pt}{0.400pt}}
\multiput(580.00,771.17)(0.500,-1.000){2}{\rule{0.120pt}{0.400pt}}
\put(576.0,774.0){\usebox{\plotpoint}}
\put(583,769.67){\rule{0.241pt}{0.400pt}}
\multiput(583.00,770.17)(0.500,-1.000){2}{\rule{0.120pt}{0.400pt}}
\put(584,768.67){\rule{0.241pt}{0.400pt}}
\multiput(584.00,769.17)(0.500,-1.000){2}{\rule{0.120pt}{0.400pt}}
\put(585,767.67){\rule{0.482pt}{0.400pt}}
\multiput(585.00,768.17)(1.000,-1.000){2}{\rule{0.241pt}{0.400pt}}
\put(581.0,771.0){\rule[-0.200pt]{0.482pt}{0.400pt}}
\put(588,766.67){\rule{0.241pt}{0.400pt}}
\multiput(588.00,767.17)(0.500,-1.000){2}{\rule{0.120pt}{0.400pt}}
\put(589,765.67){\rule{0.482pt}{0.400pt}}
\multiput(589.00,766.17)(1.000,-1.000){2}{\rule{0.241pt}{0.400pt}}
\put(591,764.67){\rule{0.241pt}{0.400pt}}
\multiput(591.00,765.17)(0.500,-1.000){2}{\rule{0.120pt}{0.400pt}}
\put(587.0,768.0){\usebox{\plotpoint}}
\put(593,763.67){\rule{0.482pt}{0.400pt}}
\multiput(593.00,764.17)(1.000,-1.000){2}{\rule{0.241pt}{0.400pt}}
\put(595,762.67){\rule{0.241pt}{0.400pt}}
\multiput(595.00,763.17)(0.500,-1.000){2}{\rule{0.120pt}{0.400pt}}
\put(596,761.67){\rule{0.241pt}{0.400pt}}
\multiput(596.00,762.17)(0.500,-1.000){2}{\rule{0.120pt}{0.400pt}}
\put(597,760.67){\rule{0.482pt}{0.400pt}}
\multiput(597.00,761.17)(1.000,-1.000){2}{\rule{0.241pt}{0.400pt}}
\put(592.0,765.0){\usebox{\plotpoint}}
\put(600,759.67){\rule{0.241pt}{0.400pt}}
\multiput(600.00,760.17)(0.500,-1.000){2}{\rule{0.120pt}{0.400pt}}
\put(601,758.67){\rule{0.482pt}{0.400pt}}
\multiput(601.00,759.17)(1.000,-1.000){2}{\rule{0.241pt}{0.400pt}}
\put(603,757.67){\rule{0.241pt}{0.400pt}}
\multiput(603.00,758.17)(0.500,-1.000){2}{\rule{0.120pt}{0.400pt}}
\put(599.0,761.0){\usebox{\plotpoint}}
\put(605,756.67){\rule{0.482pt}{0.400pt}}
\multiput(605.00,757.17)(1.000,-1.000){2}{\rule{0.241pt}{0.400pt}}
\put(607,755.67){\rule{0.241pt}{0.400pt}}
\multiput(607.00,756.17)(0.500,-1.000){2}{\rule{0.120pt}{0.400pt}}
\put(608,754.67){\rule{0.241pt}{0.400pt}}
\multiput(608.00,755.17)(0.500,-1.000){2}{\rule{0.120pt}{0.400pt}}
\put(609,753.67){\rule{0.482pt}{0.400pt}}
\multiput(609.00,754.17)(1.000,-1.000){2}{\rule{0.241pt}{0.400pt}}
\put(604.0,758.0){\usebox{\plotpoint}}
\put(612,752.67){\rule{0.241pt}{0.400pt}}
\multiput(612.00,753.17)(0.500,-1.000){2}{\rule{0.120pt}{0.400pt}}
\put(613,751.67){\rule{0.482pt}{0.400pt}}
\multiput(613.00,752.17)(1.000,-1.000){2}{\rule{0.241pt}{0.400pt}}
\put(615,750.67){\rule{0.241pt}{0.400pt}}
\multiput(615.00,751.17)(0.500,-1.000){2}{\rule{0.120pt}{0.400pt}}
\put(616,749.67){\rule{0.241pt}{0.400pt}}
\multiput(616.00,750.17)(0.500,-1.000){2}{\rule{0.120pt}{0.400pt}}
\put(611.0,754.0){\usebox{\plotpoint}}
\put(618,748.67){\rule{0.482pt}{0.400pt}}
\multiput(618.00,749.17)(1.000,-1.000){2}{\rule{0.241pt}{0.400pt}}
\put(620,747.67){\rule{0.241pt}{0.400pt}}
\multiput(620.00,748.17)(0.500,-1.000){2}{\rule{0.120pt}{0.400pt}}
\put(621,746.67){\rule{0.241pt}{0.400pt}}
\multiput(621.00,747.17)(0.500,-1.000){2}{\rule{0.120pt}{0.400pt}}
\put(622,745.67){\rule{0.482pt}{0.400pt}}
\multiput(622.00,746.17)(1.000,-1.000){2}{\rule{0.241pt}{0.400pt}}
\put(624,744.67){\rule{0.241pt}{0.400pt}}
\multiput(624.00,745.17)(0.500,-1.000){2}{\rule{0.120pt}{0.400pt}}
\put(617.0,750.0){\usebox{\plotpoint}}
\put(626,743.67){\rule{0.482pt}{0.400pt}}
\multiput(626.00,744.17)(1.000,-1.000){2}{\rule{0.241pt}{0.400pt}}
\put(628,742.67){\rule{0.241pt}{0.400pt}}
\multiput(628.00,743.17)(0.500,-1.000){2}{\rule{0.120pt}{0.400pt}}
\put(629,741.67){\rule{0.241pt}{0.400pt}}
\multiput(629.00,742.17)(0.500,-1.000){2}{\rule{0.120pt}{0.400pt}}
\put(630,740.67){\rule{0.482pt}{0.400pt}}
\multiput(630.00,741.17)(1.000,-1.000){2}{\rule{0.241pt}{0.400pt}}
\put(625.0,745.0){\usebox{\plotpoint}}
\put(633,739.67){\rule{0.241pt}{0.400pt}}
\multiput(633.00,740.17)(0.500,-1.000){2}{\rule{0.120pt}{0.400pt}}
\put(634,738.67){\rule{0.482pt}{0.400pt}}
\multiput(634.00,739.17)(1.000,-1.000){2}{\rule{0.241pt}{0.400pt}}
\put(636,737.67){\rule{0.241pt}{0.400pt}}
\multiput(636.00,738.17)(0.500,-1.000){2}{\rule{0.120pt}{0.400pt}}
\put(637,736.67){\rule{0.241pt}{0.400pt}}
\multiput(637.00,737.17)(0.500,-1.000){2}{\rule{0.120pt}{0.400pt}}
\put(638,735.67){\rule{0.482pt}{0.400pt}}
\multiput(638.00,736.17)(1.000,-1.000){2}{\rule{0.241pt}{0.400pt}}
\put(640,734.67){\rule{0.241pt}{0.400pt}}
\multiput(640.00,735.17)(0.500,-1.000){2}{\rule{0.120pt}{0.400pt}}
\put(632.0,741.0){\usebox{\plotpoint}}
\put(642,733.67){\rule{0.482pt}{0.400pt}}
\multiput(642.00,734.17)(1.000,-1.000){2}{\rule{0.241pt}{0.400pt}}
\put(644,732.67){\rule{0.241pt}{0.400pt}}
\multiput(644.00,733.17)(0.500,-1.000){2}{\rule{0.120pt}{0.400pt}}
\put(645,731.67){\rule{0.241pt}{0.400pt}}
\multiput(645.00,732.17)(0.500,-1.000){2}{\rule{0.120pt}{0.400pt}}
\put(646,730.67){\rule{0.482pt}{0.400pt}}
\multiput(646.00,731.17)(1.000,-1.000){2}{\rule{0.241pt}{0.400pt}}
\put(648,729.67){\rule{0.241pt}{0.400pt}}
\multiput(648.00,730.17)(0.500,-1.000){2}{\rule{0.120pt}{0.400pt}}
\put(641.0,735.0){\usebox{\plotpoint}}
\put(650,728.67){\rule{0.482pt}{0.400pt}}
\multiput(650.00,729.17)(1.000,-1.000){2}{\rule{0.241pt}{0.400pt}}
\put(652,727.67){\rule{0.241pt}{0.400pt}}
\multiput(652.00,728.17)(0.500,-1.000){2}{\rule{0.120pt}{0.400pt}}
\put(653,726.67){\rule{0.241pt}{0.400pt}}
\multiput(653.00,727.17)(0.500,-1.000){2}{\rule{0.120pt}{0.400pt}}
\put(654,725.67){\rule{0.482pt}{0.400pt}}
\multiput(654.00,726.17)(1.000,-1.000){2}{\rule{0.241pt}{0.400pt}}
\put(656,724.67){\rule{0.241pt}{0.400pt}}
\multiput(656.00,725.17)(0.500,-1.000){2}{\rule{0.120pt}{0.400pt}}
\put(657,723.67){\rule{0.241pt}{0.400pt}}
\multiput(657.00,724.17)(0.500,-1.000){2}{\rule{0.120pt}{0.400pt}}
\put(658,722.67){\rule{0.482pt}{0.400pt}}
\multiput(658.00,723.17)(1.000,-1.000){2}{\rule{0.241pt}{0.400pt}}
\put(649.0,730.0){\usebox{\plotpoint}}
\put(661,721.67){\rule{0.241pt}{0.400pt}}
\multiput(661.00,722.17)(0.500,-1.000){2}{\rule{0.120pt}{0.400pt}}
\put(662,720.67){\rule{0.482pt}{0.400pt}}
\multiput(662.00,721.17)(1.000,-1.000){2}{\rule{0.241pt}{0.400pt}}
\put(664,719.67){\rule{0.241pt}{0.400pt}}
\multiput(664.00,720.17)(0.500,-1.000){2}{\rule{0.120pt}{0.400pt}}
\put(665,718.67){\rule{0.241pt}{0.400pt}}
\multiput(665.00,719.17)(0.500,-1.000){2}{\rule{0.120pt}{0.400pt}}
\put(666,717.67){\rule{0.482pt}{0.400pt}}
\multiput(666.00,718.17)(1.000,-1.000){2}{\rule{0.241pt}{0.400pt}}
\put(668,716.67){\rule{0.241pt}{0.400pt}}
\multiput(668.00,717.17)(0.500,-1.000){2}{\rule{0.120pt}{0.400pt}}
\put(669,715.67){\rule{0.241pt}{0.400pt}}
\multiput(669.00,716.17)(0.500,-1.000){2}{\rule{0.120pt}{0.400pt}}
\put(660.0,723.0){\usebox{\plotpoint}}
\put(672,714.67){\rule{0.241pt}{0.400pt}}
\multiput(672.00,715.17)(0.500,-1.000){2}{\rule{0.120pt}{0.400pt}}
\put(673,713.67){\rule{0.241pt}{0.400pt}}
\multiput(673.00,714.17)(0.500,-1.000){2}{\rule{0.120pt}{0.400pt}}
\put(674,712.67){\rule{0.482pt}{0.400pt}}
\multiput(674.00,713.17)(1.000,-1.000){2}{\rule{0.241pt}{0.400pt}}
\put(676,711.67){\rule{0.241pt}{0.400pt}}
\multiput(676.00,712.17)(0.500,-1.000){2}{\rule{0.120pt}{0.400pt}}
\put(677,710.67){\rule{0.241pt}{0.400pt}}
\multiput(677.00,711.17)(0.500,-1.000){2}{\rule{0.120pt}{0.400pt}}
\put(678,709.67){\rule{0.482pt}{0.400pt}}
\multiput(678.00,710.17)(1.000,-1.000){2}{\rule{0.241pt}{0.400pt}}
\put(680,708.67){\rule{0.241pt}{0.400pt}}
\multiput(680.00,709.17)(0.500,-1.000){2}{\rule{0.120pt}{0.400pt}}
\put(681,707.67){\rule{0.241pt}{0.400pt}}
\multiput(681.00,708.17)(0.500,-1.000){2}{\rule{0.120pt}{0.400pt}}
\put(682,706.67){\rule{0.482pt}{0.400pt}}
\multiput(682.00,707.17)(1.000,-1.000){2}{\rule{0.241pt}{0.400pt}}
\put(670.0,716.0){\rule[-0.200pt]{0.482pt}{0.400pt}}
\put(685,705.67){\rule{0.241pt}{0.400pt}}
\multiput(685.00,706.17)(0.500,-1.000){2}{\rule{0.120pt}{0.400pt}}
\put(686,704.67){\rule{0.482pt}{0.400pt}}
\multiput(686.00,705.17)(1.000,-1.000){2}{\rule{0.241pt}{0.400pt}}
\put(688,703.67){\rule{0.241pt}{0.400pt}}
\multiput(688.00,704.17)(0.500,-1.000){2}{\rule{0.120pt}{0.400pt}}
\put(689,702.67){\rule{0.241pt}{0.400pt}}
\multiput(689.00,703.17)(0.500,-1.000){2}{\rule{0.120pt}{0.400pt}}
\put(690,701.67){\rule{0.482pt}{0.400pt}}
\multiput(690.00,702.17)(1.000,-1.000){2}{\rule{0.241pt}{0.400pt}}
\put(692,700.67){\rule{0.241pt}{0.400pt}}
\multiput(692.00,701.17)(0.500,-1.000){2}{\rule{0.120pt}{0.400pt}}
\put(693,699.67){\rule{0.241pt}{0.400pt}}
\multiput(693.00,700.17)(0.500,-1.000){2}{\rule{0.120pt}{0.400pt}}
\put(694,698.67){\rule{0.482pt}{0.400pt}}
\multiput(694.00,699.17)(1.000,-1.000){2}{\rule{0.241pt}{0.400pt}}
\put(696,697.67){\rule{0.241pt}{0.400pt}}
\multiput(696.00,698.17)(0.500,-1.000){2}{\rule{0.120pt}{0.400pt}}
\put(684.0,707.0){\usebox{\plotpoint}}
\put(698,696.67){\rule{0.241pt}{0.400pt}}
\multiput(698.00,697.17)(0.500,-1.000){2}{\rule{0.120pt}{0.400pt}}
\put(699,695.67){\rule{0.482pt}{0.400pt}}
\multiput(699.00,696.17)(1.000,-1.000){2}{\rule{0.241pt}{0.400pt}}
\put(701,694.67){\rule{0.241pt}{0.400pt}}
\multiput(701.00,695.17)(0.500,-1.000){2}{\rule{0.120pt}{0.400pt}}
\put(702,693.67){\rule{0.241pt}{0.400pt}}
\multiput(702.00,694.17)(0.500,-1.000){2}{\rule{0.120pt}{0.400pt}}
\put(703,692.67){\rule{0.482pt}{0.400pt}}
\multiput(703.00,693.17)(1.000,-1.000){2}{\rule{0.241pt}{0.400pt}}
\put(705,691.67){\rule{0.241pt}{0.400pt}}
\multiput(705.00,692.17)(0.500,-1.000){2}{\rule{0.120pt}{0.400pt}}
\put(706,690.67){\rule{0.241pt}{0.400pt}}
\multiput(706.00,691.17)(0.500,-1.000){2}{\rule{0.120pt}{0.400pt}}
\put(707,689.67){\rule{0.482pt}{0.400pt}}
\multiput(707.00,690.17)(1.000,-1.000){2}{\rule{0.241pt}{0.400pt}}
\put(709,688.67){\rule{0.241pt}{0.400pt}}
\multiput(709.00,689.17)(0.500,-1.000){2}{\rule{0.120pt}{0.400pt}}
\put(710,687.67){\rule{0.241pt}{0.400pt}}
\multiput(710.00,688.17)(0.500,-1.000){2}{\rule{0.120pt}{0.400pt}}
\put(711,686.67){\rule{0.482pt}{0.400pt}}
\multiput(711.00,687.17)(1.000,-1.000){2}{\rule{0.241pt}{0.400pt}}
\put(713,685.67){\rule{0.241pt}{0.400pt}}
\multiput(713.00,686.17)(0.500,-1.000){2}{\rule{0.120pt}{0.400pt}}
\put(697.0,698.0){\usebox{\plotpoint}}
\put(715,684.67){\rule{0.482pt}{0.400pt}}
\multiput(715.00,685.17)(1.000,-1.000){2}{\rule{0.241pt}{0.400pt}}
\put(717,683.67){\rule{0.241pt}{0.400pt}}
\multiput(717.00,684.17)(0.500,-1.000){2}{\rule{0.120pt}{0.400pt}}
\put(718,682.67){\rule{0.241pt}{0.400pt}}
\multiput(718.00,683.17)(0.500,-1.000){2}{\rule{0.120pt}{0.400pt}}
\put(719,681.67){\rule{0.482pt}{0.400pt}}
\multiput(719.00,682.17)(1.000,-1.000){2}{\rule{0.241pt}{0.400pt}}
\put(721,680.67){\rule{0.241pt}{0.400pt}}
\multiput(721.00,681.17)(0.500,-1.000){2}{\rule{0.120pt}{0.400pt}}
\put(722,679.67){\rule{0.241pt}{0.400pt}}
\multiput(722.00,680.17)(0.500,-1.000){2}{\rule{0.120pt}{0.400pt}}
\put(723,678.67){\rule{0.482pt}{0.400pt}}
\multiput(723.00,679.17)(1.000,-1.000){2}{\rule{0.241pt}{0.400pt}}
\put(725,677.67){\rule{0.241pt}{0.400pt}}
\multiput(725.00,678.17)(0.500,-1.000){2}{\rule{0.120pt}{0.400pt}}
\put(726,676.67){\rule{0.241pt}{0.400pt}}
\multiput(726.00,677.17)(0.500,-1.000){2}{\rule{0.120pt}{0.400pt}}
\put(727,675.67){\rule{0.482pt}{0.400pt}}
\multiput(727.00,676.17)(1.000,-1.000){2}{\rule{0.241pt}{0.400pt}}
\put(729,674.67){\rule{0.241pt}{0.400pt}}
\multiput(729.00,675.17)(0.500,-1.000){2}{\rule{0.120pt}{0.400pt}}
\put(730,673.67){\rule{0.241pt}{0.400pt}}
\multiput(730.00,674.17)(0.500,-1.000){2}{\rule{0.120pt}{0.400pt}}
\put(731,672.67){\rule{0.482pt}{0.400pt}}
\multiput(731.00,673.17)(1.000,-1.000){2}{\rule{0.241pt}{0.400pt}}
\put(733,671.67){\rule{0.241pt}{0.400pt}}
\multiput(733.00,672.17)(0.500,-1.000){2}{\rule{0.120pt}{0.400pt}}
\put(714.0,686.0){\usebox{\plotpoint}}
\put(735,670.67){\rule{0.482pt}{0.400pt}}
\multiput(735.00,671.17)(1.000,-1.000){2}{\rule{0.241pt}{0.400pt}}
\put(737,669.67){\rule{0.241pt}{0.400pt}}
\multiput(737.00,670.17)(0.500,-1.000){2}{\rule{0.120pt}{0.400pt}}
\put(738,668.67){\rule{0.241pt}{0.400pt}}
\multiput(738.00,669.17)(0.500,-1.000){2}{\rule{0.120pt}{0.400pt}}
\put(739,667.67){\rule{0.482pt}{0.400pt}}
\multiput(739.00,668.17)(1.000,-1.000){2}{\rule{0.241pt}{0.400pt}}
\put(741,666.67){\rule{0.241pt}{0.400pt}}
\multiput(741.00,667.17)(0.500,-1.000){2}{\rule{0.120pt}{0.400pt}}
\put(742,665.67){\rule{0.241pt}{0.400pt}}
\multiput(742.00,666.17)(0.500,-1.000){2}{\rule{0.120pt}{0.400pt}}
\put(743,664.67){\rule{0.482pt}{0.400pt}}
\multiput(743.00,665.17)(1.000,-1.000){2}{\rule{0.241pt}{0.400pt}}
\put(745,663.67){\rule{0.241pt}{0.400pt}}
\multiput(745.00,664.17)(0.500,-1.000){2}{\rule{0.120pt}{0.400pt}}
\put(746,662.67){\rule{0.241pt}{0.400pt}}
\multiput(746.00,663.17)(0.500,-1.000){2}{\rule{0.120pt}{0.400pt}}
\put(747,661.67){\rule{0.482pt}{0.400pt}}
\multiput(747.00,662.17)(1.000,-1.000){2}{\rule{0.241pt}{0.400pt}}
\put(749,660.67){\rule{0.241pt}{0.400pt}}
\multiput(749.00,661.17)(0.500,-1.000){2}{\rule{0.120pt}{0.400pt}}
\put(750,659.67){\rule{0.241pt}{0.400pt}}
\multiput(750.00,660.17)(0.500,-1.000){2}{\rule{0.120pt}{0.400pt}}
\put(751,658.67){\rule{0.482pt}{0.400pt}}
\multiput(751.00,659.17)(1.000,-1.000){2}{\rule{0.241pt}{0.400pt}}
\put(753,657.67){\rule{0.241pt}{0.400pt}}
\multiput(753.00,658.17)(0.500,-1.000){2}{\rule{0.120pt}{0.400pt}}
\put(754,656.67){\rule{0.241pt}{0.400pt}}
\multiput(754.00,657.17)(0.500,-1.000){2}{\rule{0.120pt}{0.400pt}}
\put(755,655.67){\rule{0.482pt}{0.400pt}}
\multiput(755.00,656.17)(1.000,-1.000){2}{\rule{0.241pt}{0.400pt}}
\put(757,654.67){\rule{0.241pt}{0.400pt}}
\multiput(757.00,655.17)(0.500,-1.000){2}{\rule{0.120pt}{0.400pt}}
\put(758,653.67){\rule{0.241pt}{0.400pt}}
\multiput(758.00,654.17)(0.500,-1.000){2}{\rule{0.120pt}{0.400pt}}
\put(759,652.67){\rule{0.482pt}{0.400pt}}
\multiput(759.00,653.17)(1.000,-1.000){2}{\rule{0.241pt}{0.400pt}}
\put(761,651.67){\rule{0.241pt}{0.400pt}}
\multiput(761.00,652.17)(0.500,-1.000){2}{\rule{0.120pt}{0.400pt}}
\put(734.0,672.0){\usebox{\plotpoint}}
\put(763,650.67){\rule{0.482pt}{0.400pt}}
\multiput(763.00,651.17)(1.000,-1.000){2}{\rule{0.241pt}{0.400pt}}
\put(765,649.67){\rule{0.241pt}{0.400pt}}
\multiput(765.00,650.17)(0.500,-1.000){2}{\rule{0.120pt}{0.400pt}}
\put(766,648.67){\rule{0.241pt}{0.400pt}}
\multiput(766.00,649.17)(0.500,-1.000){2}{\rule{0.120pt}{0.400pt}}
\put(767,647.67){\rule{0.482pt}{0.400pt}}
\multiput(767.00,648.17)(1.000,-1.000){2}{\rule{0.241pt}{0.400pt}}
\put(769,646.67){\rule{0.241pt}{0.400pt}}
\multiput(769.00,647.17)(0.500,-1.000){2}{\rule{0.120pt}{0.400pt}}
\put(770,645.67){\rule{0.241pt}{0.400pt}}
\multiput(770.00,646.17)(0.500,-1.000){2}{\rule{0.120pt}{0.400pt}}
\put(771,644.67){\rule{0.482pt}{0.400pt}}
\multiput(771.00,645.17)(1.000,-1.000){2}{\rule{0.241pt}{0.400pt}}
\put(773,643.67){\rule{0.241pt}{0.400pt}}
\multiput(773.00,644.17)(0.500,-1.000){2}{\rule{0.120pt}{0.400pt}}
\put(774,642.67){\rule{0.241pt}{0.400pt}}
\multiput(774.00,643.17)(0.500,-1.000){2}{\rule{0.120pt}{0.400pt}}
\put(775,641.67){\rule{0.241pt}{0.400pt}}
\multiput(775.00,642.17)(0.500,-1.000){2}{\rule{0.120pt}{0.400pt}}
\put(776,640.67){\rule{0.482pt}{0.400pt}}
\multiput(776.00,641.17)(1.000,-1.000){2}{\rule{0.241pt}{0.400pt}}
\put(778,639.67){\rule{0.241pt}{0.400pt}}
\multiput(778.00,640.17)(0.500,-1.000){2}{\rule{0.120pt}{0.400pt}}
\put(779,638.67){\rule{0.241pt}{0.400pt}}
\multiput(779.00,639.17)(0.500,-1.000){2}{\rule{0.120pt}{0.400pt}}
\put(780,637.67){\rule{0.482pt}{0.400pt}}
\multiput(780.00,638.17)(1.000,-1.000){2}{\rule{0.241pt}{0.400pt}}
\put(782,636.67){\rule{0.241pt}{0.400pt}}
\multiput(782.00,637.17)(0.500,-1.000){2}{\rule{0.120pt}{0.400pt}}
\put(783,635.67){\rule{0.241pt}{0.400pt}}
\multiput(783.00,636.17)(0.500,-1.000){2}{\rule{0.120pt}{0.400pt}}
\put(784,634.67){\rule{0.482pt}{0.400pt}}
\multiput(784.00,635.17)(1.000,-1.000){2}{\rule{0.241pt}{0.400pt}}
\put(786,633.67){\rule{0.241pt}{0.400pt}}
\multiput(786.00,634.17)(0.500,-1.000){2}{\rule{0.120pt}{0.400pt}}
\put(787,632.67){\rule{0.241pt}{0.400pt}}
\multiput(787.00,633.17)(0.500,-1.000){2}{\rule{0.120pt}{0.400pt}}
\put(788,631.67){\rule{0.482pt}{0.400pt}}
\multiput(788.00,632.17)(1.000,-1.000){2}{\rule{0.241pt}{0.400pt}}
\put(790,630.67){\rule{0.241pt}{0.400pt}}
\multiput(790.00,631.17)(0.500,-1.000){2}{\rule{0.120pt}{0.400pt}}
\put(791,629.67){\rule{0.241pt}{0.400pt}}
\multiput(791.00,630.17)(0.500,-1.000){2}{\rule{0.120pt}{0.400pt}}
\put(792,628.67){\rule{0.482pt}{0.400pt}}
\multiput(792.00,629.17)(1.000,-1.000){2}{\rule{0.241pt}{0.400pt}}
\put(794,627.67){\rule{0.241pt}{0.400pt}}
\multiput(794.00,628.17)(0.500,-1.000){2}{\rule{0.120pt}{0.400pt}}
\put(795,626.67){\rule{0.241pt}{0.400pt}}
\multiput(795.00,627.17)(0.500,-1.000){2}{\rule{0.120pt}{0.400pt}}
\put(796,625.67){\rule{0.482pt}{0.400pt}}
\multiput(796.00,626.17)(1.000,-1.000){2}{\rule{0.241pt}{0.400pt}}
\put(798,624.67){\rule{0.241pt}{0.400pt}}
\multiput(798.00,625.17)(0.500,-1.000){2}{\rule{0.120pt}{0.400pt}}
\put(799,623.67){\rule{0.241pt}{0.400pt}}
\multiput(799.00,624.17)(0.500,-1.000){2}{\rule{0.120pt}{0.400pt}}
\put(800,622.67){\rule{0.482pt}{0.400pt}}
\multiput(800.00,623.17)(1.000,-1.000){2}{\rule{0.241pt}{0.400pt}}
\put(802,621.67){\rule{0.241pt}{0.400pt}}
\multiput(802.00,622.17)(0.500,-1.000){2}{\rule{0.120pt}{0.400pt}}
\put(803,620.67){\rule{0.241pt}{0.400pt}}
\multiput(803.00,621.17)(0.500,-1.000){2}{\rule{0.120pt}{0.400pt}}
\put(804,619.67){\rule{0.482pt}{0.400pt}}
\multiput(804.00,620.17)(1.000,-1.000){2}{\rule{0.241pt}{0.400pt}}
\put(806,618.67){\rule{0.241pt}{0.400pt}}
\multiput(806.00,619.17)(0.500,-1.000){2}{\rule{0.120pt}{0.400pt}}
\put(807,617.67){\rule{0.241pt}{0.400pt}}
\multiput(807.00,618.17)(0.500,-1.000){2}{\rule{0.120pt}{0.400pt}}
\put(808,616.67){\rule{0.482pt}{0.400pt}}
\multiput(808.00,617.17)(1.000,-1.000){2}{\rule{0.241pt}{0.400pt}}
\put(810,615.67){\rule{0.241pt}{0.400pt}}
\multiput(810.00,616.17)(0.500,-1.000){2}{\rule{0.120pt}{0.400pt}}
\put(811,614.67){\rule{0.241pt}{0.400pt}}
\multiput(811.00,615.17)(0.500,-1.000){2}{\rule{0.120pt}{0.400pt}}
\put(812,613.67){\rule{0.482pt}{0.400pt}}
\multiput(812.00,614.17)(1.000,-1.000){2}{\rule{0.241pt}{0.400pt}}
\put(762.0,652.0){\usebox{\plotpoint}}
\put(815,612.67){\rule{0.241pt}{0.400pt}}
\multiput(815.00,613.17)(0.500,-1.000){2}{\rule{0.120pt}{0.400pt}}
\put(816,611.67){\rule{0.482pt}{0.400pt}}
\multiput(816.00,612.17)(1.000,-1.000){2}{\rule{0.241pt}{0.400pt}}
\put(818,610.67){\rule{0.241pt}{0.400pt}}
\multiput(818.00,611.17)(0.500,-1.000){2}{\rule{0.120pt}{0.400pt}}
\put(819,609.67){\rule{0.241pt}{0.400pt}}
\multiput(819.00,610.17)(0.500,-1.000){2}{\rule{0.120pt}{0.400pt}}
\put(820,608.67){\rule{0.482pt}{0.400pt}}
\multiput(820.00,609.17)(1.000,-1.000){2}{\rule{0.241pt}{0.400pt}}
\put(822,607.67){\rule{0.241pt}{0.400pt}}
\multiput(822.00,608.17)(0.500,-1.000){2}{\rule{0.120pt}{0.400pt}}
\put(823,606.67){\rule{0.241pt}{0.400pt}}
\multiput(823.00,607.17)(0.500,-1.000){2}{\rule{0.120pt}{0.400pt}}
\put(824,605.67){\rule{0.482pt}{0.400pt}}
\multiput(824.00,606.17)(1.000,-1.000){2}{\rule{0.241pt}{0.400pt}}
\put(826,604.67){\rule{0.241pt}{0.400pt}}
\multiput(826.00,605.17)(0.500,-1.000){2}{\rule{0.120pt}{0.400pt}}
\put(827,603.67){\rule{0.241pt}{0.400pt}}
\multiput(827.00,604.17)(0.500,-1.000){2}{\rule{0.120pt}{0.400pt}}
\put(828,602.67){\rule{0.482pt}{0.400pt}}
\multiput(828.00,603.17)(1.000,-1.000){2}{\rule{0.241pt}{0.400pt}}
\put(830,601.67){\rule{0.241pt}{0.400pt}}
\multiput(830.00,602.17)(0.500,-1.000){2}{\rule{0.120pt}{0.400pt}}
\put(831,600.67){\rule{0.241pt}{0.400pt}}
\multiput(831.00,601.17)(0.500,-1.000){2}{\rule{0.120pt}{0.400pt}}
\put(832,599.67){\rule{0.482pt}{0.400pt}}
\multiput(832.00,600.17)(1.000,-1.000){2}{\rule{0.241pt}{0.400pt}}
\put(834,598.67){\rule{0.241pt}{0.400pt}}
\multiput(834.00,599.17)(0.500,-1.000){2}{\rule{0.120pt}{0.400pt}}
\put(835,597.67){\rule{0.241pt}{0.400pt}}
\multiput(835.00,598.17)(0.500,-1.000){2}{\rule{0.120pt}{0.400pt}}
\put(836,596.67){\rule{0.482pt}{0.400pt}}
\multiput(836.00,597.17)(1.000,-1.000){2}{\rule{0.241pt}{0.400pt}}
\put(838,595.67){\rule{0.241pt}{0.400pt}}
\multiput(838.00,596.17)(0.500,-1.000){2}{\rule{0.120pt}{0.400pt}}
\put(839,594.67){\rule{0.241pt}{0.400pt}}
\multiput(839.00,595.17)(0.500,-1.000){2}{\rule{0.120pt}{0.400pt}}
\put(840,593.67){\rule{0.482pt}{0.400pt}}
\multiput(840.00,594.17)(1.000,-1.000){2}{\rule{0.241pt}{0.400pt}}
\put(842,592.67){\rule{0.241pt}{0.400pt}}
\multiput(842.00,593.17)(0.500,-1.000){2}{\rule{0.120pt}{0.400pt}}
\put(843,591.67){\rule{0.241pt}{0.400pt}}
\multiput(843.00,592.17)(0.500,-1.000){2}{\rule{0.120pt}{0.400pt}}
\put(844,590.67){\rule{0.482pt}{0.400pt}}
\multiput(844.00,591.17)(1.000,-1.000){2}{\rule{0.241pt}{0.400pt}}
\put(846,589.67){\rule{0.241pt}{0.400pt}}
\multiput(846.00,590.17)(0.500,-1.000){2}{\rule{0.120pt}{0.400pt}}
\put(847,588.67){\rule{0.241pt}{0.400pt}}
\multiput(847.00,589.17)(0.500,-1.000){2}{\rule{0.120pt}{0.400pt}}
\put(848,587.67){\rule{0.482pt}{0.400pt}}
\multiput(848.00,588.17)(1.000,-1.000){2}{\rule{0.241pt}{0.400pt}}
\put(850,586.67){\rule{0.241pt}{0.400pt}}
\multiput(850.00,587.17)(0.500,-1.000){2}{\rule{0.120pt}{0.400pt}}
\put(851,585.67){\rule{0.241pt}{0.400pt}}
\multiput(851.00,586.17)(0.500,-1.000){2}{\rule{0.120pt}{0.400pt}}
\put(852,584.67){\rule{0.241pt}{0.400pt}}
\multiput(852.00,585.17)(0.500,-1.000){2}{\rule{0.120pt}{0.400pt}}
\put(853,583.67){\rule{0.482pt}{0.400pt}}
\multiput(853.00,584.17)(1.000,-1.000){2}{\rule{0.241pt}{0.400pt}}
\put(855,582.67){\rule{0.241pt}{0.400pt}}
\multiput(855.00,583.17)(0.500,-1.000){2}{\rule{0.120pt}{0.400pt}}
\put(856,581.67){\rule{0.241pt}{0.400pt}}
\multiput(856.00,582.17)(0.500,-1.000){2}{\rule{0.120pt}{0.400pt}}
\put(857,580.67){\rule{0.482pt}{0.400pt}}
\multiput(857.00,581.17)(1.000,-1.000){2}{\rule{0.241pt}{0.400pt}}
\put(859,579.67){\rule{0.241pt}{0.400pt}}
\multiput(859.00,580.17)(0.500,-1.000){2}{\rule{0.120pt}{0.400pt}}
\put(860,578.67){\rule{0.241pt}{0.400pt}}
\multiput(860.00,579.17)(0.500,-1.000){2}{\rule{0.120pt}{0.400pt}}
\put(861,577.67){\rule{0.482pt}{0.400pt}}
\multiput(861.00,578.17)(1.000,-1.000){2}{\rule{0.241pt}{0.400pt}}
\put(863,576.67){\rule{0.241pt}{0.400pt}}
\multiput(863.00,577.17)(0.500,-1.000){2}{\rule{0.120pt}{0.400pt}}
\put(864,575.67){\rule{0.241pt}{0.400pt}}
\multiput(864.00,576.17)(0.500,-1.000){2}{\rule{0.120pt}{0.400pt}}
\put(865,574.67){\rule{0.482pt}{0.400pt}}
\multiput(865.00,575.17)(1.000,-1.000){2}{\rule{0.241pt}{0.400pt}}
\put(867,573.67){\rule{0.241pt}{0.400pt}}
\multiput(867.00,574.17)(0.500,-1.000){2}{\rule{0.120pt}{0.400pt}}
\put(868,572.67){\rule{0.241pt}{0.400pt}}
\multiput(868.00,573.17)(0.500,-1.000){2}{\rule{0.120pt}{0.400pt}}
\put(869,571.67){\rule{0.482pt}{0.400pt}}
\multiput(869.00,572.17)(1.000,-1.000){2}{\rule{0.241pt}{0.400pt}}
\put(871,570.67){\rule{0.241pt}{0.400pt}}
\multiput(871.00,571.17)(0.500,-1.000){2}{\rule{0.120pt}{0.400pt}}
\put(872,569.67){\rule{0.241pt}{0.400pt}}
\multiput(872.00,570.17)(0.500,-1.000){2}{\rule{0.120pt}{0.400pt}}
\put(873,568.67){\rule{0.482pt}{0.400pt}}
\multiput(873.00,569.17)(1.000,-1.000){2}{\rule{0.241pt}{0.400pt}}
\put(875,567.67){\rule{0.241pt}{0.400pt}}
\multiput(875.00,568.17)(0.500,-1.000){2}{\rule{0.120pt}{0.400pt}}
\put(876,566.67){\rule{0.241pt}{0.400pt}}
\multiput(876.00,567.17)(0.500,-1.000){2}{\rule{0.120pt}{0.400pt}}
\put(877,565.67){\rule{0.482pt}{0.400pt}}
\multiput(877.00,566.17)(1.000,-1.000){2}{\rule{0.241pt}{0.400pt}}
\put(879,564.67){\rule{0.241pt}{0.400pt}}
\multiput(879.00,565.17)(0.500,-1.000){2}{\rule{0.120pt}{0.400pt}}
\put(880,563.67){\rule{0.241pt}{0.400pt}}
\multiput(880.00,564.17)(0.500,-1.000){2}{\rule{0.120pt}{0.400pt}}
\put(881,562.67){\rule{0.482pt}{0.400pt}}
\multiput(881.00,563.17)(1.000,-1.000){2}{\rule{0.241pt}{0.400pt}}
\put(883,561.67){\rule{0.241pt}{0.400pt}}
\multiput(883.00,562.17)(0.500,-1.000){2}{\rule{0.120pt}{0.400pt}}
\put(884,560.67){\rule{0.241pt}{0.400pt}}
\multiput(884.00,561.17)(0.500,-1.000){2}{\rule{0.120pt}{0.400pt}}
\put(885,559.67){\rule{0.482pt}{0.400pt}}
\multiput(885.00,560.17)(1.000,-1.000){2}{\rule{0.241pt}{0.400pt}}
\put(887,558.67){\rule{0.241pt}{0.400pt}}
\multiput(887.00,559.17)(0.500,-1.000){2}{\rule{0.120pt}{0.400pt}}
\put(888,557.67){\rule{0.241pt}{0.400pt}}
\multiput(888.00,558.17)(0.500,-1.000){2}{\rule{0.120pt}{0.400pt}}
\put(889,556.67){\rule{0.482pt}{0.400pt}}
\multiput(889.00,557.17)(1.000,-1.000){2}{\rule{0.241pt}{0.400pt}}
\put(891,555.67){\rule{0.241pt}{0.400pt}}
\multiput(891.00,556.17)(0.500,-1.000){2}{\rule{0.120pt}{0.400pt}}
\put(892,554.67){\rule{0.241pt}{0.400pt}}
\multiput(892.00,555.17)(0.500,-1.000){2}{\rule{0.120pt}{0.400pt}}
\put(893,553.67){\rule{0.482pt}{0.400pt}}
\multiput(893.00,554.17)(1.000,-1.000){2}{\rule{0.241pt}{0.400pt}}
\put(895,552.67){\rule{0.241pt}{0.400pt}}
\multiput(895.00,553.17)(0.500,-1.000){2}{\rule{0.120pt}{0.400pt}}
\put(896,551.67){\rule{0.241pt}{0.400pt}}
\multiput(896.00,552.17)(0.500,-1.000){2}{\rule{0.120pt}{0.400pt}}
\put(897,550.67){\rule{0.482pt}{0.400pt}}
\multiput(897.00,551.17)(1.000,-1.000){2}{\rule{0.241pt}{0.400pt}}
\put(899,549.67){\rule{0.241pt}{0.400pt}}
\multiput(899.00,550.17)(0.500,-1.000){2}{\rule{0.120pt}{0.400pt}}
\put(900,548.67){\rule{0.241pt}{0.400pt}}
\multiput(900.00,549.17)(0.500,-1.000){2}{\rule{0.120pt}{0.400pt}}
\put(901,547.67){\rule{0.482pt}{0.400pt}}
\multiput(901.00,548.17)(1.000,-1.000){2}{\rule{0.241pt}{0.400pt}}
\put(903,546.67){\rule{0.241pt}{0.400pt}}
\multiput(903.00,547.17)(0.500,-1.000){2}{\rule{0.120pt}{0.400pt}}
\put(903.67,545){\rule{0.400pt}{0.482pt}}
\multiput(903.17,546.00)(1.000,-1.000){2}{\rule{0.400pt}{0.241pt}}
\put(905,543.67){\rule{0.482pt}{0.400pt}}
\multiput(905.00,544.17)(1.000,-1.000){2}{\rule{0.241pt}{0.400pt}}
\put(907,542.67){\rule{0.241pt}{0.400pt}}
\multiput(907.00,543.17)(0.500,-1.000){2}{\rule{0.120pt}{0.400pt}}
\put(908,541.67){\rule{0.241pt}{0.400pt}}
\multiput(908.00,542.17)(0.500,-1.000){2}{\rule{0.120pt}{0.400pt}}
\put(909,540.67){\rule{0.482pt}{0.400pt}}
\multiput(909.00,541.17)(1.000,-1.000){2}{\rule{0.241pt}{0.400pt}}
\put(911,539.67){\rule{0.241pt}{0.400pt}}
\multiput(911.00,540.17)(0.500,-1.000){2}{\rule{0.120pt}{0.400pt}}
\put(912,538.67){\rule{0.241pt}{0.400pt}}
\multiput(912.00,539.17)(0.500,-1.000){2}{\rule{0.120pt}{0.400pt}}
\put(913,537.67){\rule{0.482pt}{0.400pt}}
\multiput(913.00,538.17)(1.000,-1.000){2}{\rule{0.241pt}{0.400pt}}
\put(915,536.67){\rule{0.241pt}{0.400pt}}
\multiput(915.00,537.17)(0.500,-1.000){2}{\rule{0.120pt}{0.400pt}}
\put(916,535.67){\rule{0.241pt}{0.400pt}}
\multiput(916.00,536.17)(0.500,-1.000){2}{\rule{0.120pt}{0.400pt}}
\put(917,534.67){\rule{0.482pt}{0.400pt}}
\multiput(917.00,535.17)(1.000,-1.000){2}{\rule{0.241pt}{0.400pt}}
\put(919,533.67){\rule{0.241pt}{0.400pt}}
\multiput(919.00,534.17)(0.500,-1.000){2}{\rule{0.120pt}{0.400pt}}
\put(920,532.67){\rule{0.241pt}{0.400pt}}
\multiput(920.00,533.17)(0.500,-1.000){2}{\rule{0.120pt}{0.400pt}}
\put(921,531.67){\rule{0.482pt}{0.400pt}}
\multiput(921.00,532.17)(1.000,-1.000){2}{\rule{0.241pt}{0.400pt}}
\put(923,530.67){\rule{0.241pt}{0.400pt}}
\multiput(923.00,531.17)(0.500,-1.000){2}{\rule{0.120pt}{0.400pt}}
\put(924,529.67){\rule{0.241pt}{0.400pt}}
\multiput(924.00,530.17)(0.500,-1.000){2}{\rule{0.120pt}{0.400pt}}
\put(925,528.67){\rule{0.482pt}{0.400pt}}
\multiput(925.00,529.17)(1.000,-1.000){2}{\rule{0.241pt}{0.400pt}}
\put(927,527.67){\rule{0.241pt}{0.400pt}}
\multiput(927.00,528.17)(0.500,-1.000){2}{\rule{0.120pt}{0.400pt}}
\put(928,526.67){\rule{0.241pt}{0.400pt}}
\multiput(928.00,527.17)(0.500,-1.000){2}{\rule{0.120pt}{0.400pt}}
\put(929,525.67){\rule{0.482pt}{0.400pt}}
\multiput(929.00,526.17)(1.000,-1.000){2}{\rule{0.241pt}{0.400pt}}
\put(931,524.67){\rule{0.241pt}{0.400pt}}
\multiput(931.00,525.17)(0.500,-1.000){2}{\rule{0.120pt}{0.400pt}}
\put(932,523.67){\rule{0.241pt}{0.400pt}}
\multiput(932.00,524.17)(0.500,-1.000){2}{\rule{0.120pt}{0.400pt}}
\put(933,522.67){\rule{0.241pt}{0.400pt}}
\multiput(933.00,523.17)(0.500,-1.000){2}{\rule{0.120pt}{0.400pt}}
\put(934,521.67){\rule{0.482pt}{0.400pt}}
\multiput(934.00,522.17)(1.000,-1.000){2}{\rule{0.241pt}{0.400pt}}
\put(936,520.67){\rule{0.241pt}{0.400pt}}
\multiput(936.00,521.17)(0.500,-1.000){2}{\rule{0.120pt}{0.400pt}}
\put(937,519.67){\rule{0.241pt}{0.400pt}}
\multiput(937.00,520.17)(0.500,-1.000){2}{\rule{0.120pt}{0.400pt}}
\put(938,518.67){\rule{0.482pt}{0.400pt}}
\multiput(938.00,519.17)(1.000,-1.000){2}{\rule{0.241pt}{0.400pt}}
\put(940,517.67){\rule{0.241pt}{0.400pt}}
\multiput(940.00,518.17)(0.500,-1.000){2}{\rule{0.120pt}{0.400pt}}
\put(941,516.67){\rule{0.241pt}{0.400pt}}
\multiput(941.00,517.17)(0.500,-1.000){2}{\rule{0.120pt}{0.400pt}}
\put(942,515.67){\rule{0.482pt}{0.400pt}}
\multiput(942.00,516.17)(1.000,-1.000){2}{\rule{0.241pt}{0.400pt}}
\put(944,514.67){\rule{0.241pt}{0.400pt}}
\multiput(944.00,515.17)(0.500,-1.000){2}{\rule{0.120pt}{0.400pt}}
\put(945,513.67){\rule{0.241pt}{0.400pt}}
\multiput(945.00,514.17)(0.500,-1.000){2}{\rule{0.120pt}{0.400pt}}
\put(946,512.67){\rule{0.482pt}{0.400pt}}
\multiput(946.00,513.17)(1.000,-1.000){2}{\rule{0.241pt}{0.400pt}}
\put(948,511.67){\rule{0.241pt}{0.400pt}}
\multiput(948.00,512.17)(0.500,-1.000){2}{\rule{0.120pt}{0.400pt}}
\put(949,510.67){\rule{0.241pt}{0.400pt}}
\multiput(949.00,511.17)(0.500,-1.000){2}{\rule{0.120pt}{0.400pt}}
\put(950,509.67){\rule{0.482pt}{0.400pt}}
\multiput(950.00,510.17)(1.000,-1.000){2}{\rule{0.241pt}{0.400pt}}
\put(952,508.67){\rule{0.241pt}{0.400pt}}
\multiput(952.00,509.17)(0.500,-1.000){2}{\rule{0.120pt}{0.400pt}}
\put(953,507.67){\rule{0.241pt}{0.400pt}}
\multiput(953.00,508.17)(0.500,-1.000){2}{\rule{0.120pt}{0.400pt}}
\put(954,506.67){\rule{0.482pt}{0.400pt}}
\multiput(954.00,507.17)(1.000,-1.000){2}{\rule{0.241pt}{0.400pt}}
\put(956,505.67){\rule{0.241pt}{0.400pt}}
\multiput(956.00,506.17)(0.500,-1.000){2}{\rule{0.120pt}{0.400pt}}
\put(957,504.67){\rule{0.241pt}{0.400pt}}
\multiput(957.00,505.17)(0.500,-1.000){2}{\rule{0.120pt}{0.400pt}}
\put(958,503.67){\rule{0.482pt}{0.400pt}}
\multiput(958.00,504.17)(1.000,-1.000){2}{\rule{0.241pt}{0.400pt}}
\put(960,502.67){\rule{0.241pt}{0.400pt}}
\multiput(960.00,503.17)(0.500,-1.000){2}{\rule{0.120pt}{0.400pt}}
\put(961,501.67){\rule{0.241pt}{0.400pt}}
\multiput(961.00,502.17)(0.500,-1.000){2}{\rule{0.120pt}{0.400pt}}
\put(962,500.67){\rule{0.482pt}{0.400pt}}
\multiput(962.00,501.17)(1.000,-1.000){2}{\rule{0.241pt}{0.400pt}}
\put(964,499.67){\rule{0.241pt}{0.400pt}}
\multiput(964.00,500.17)(0.500,-1.000){2}{\rule{0.120pt}{0.400pt}}
\put(965,498.67){\rule{0.241pt}{0.400pt}}
\multiput(965.00,499.17)(0.500,-1.000){2}{\rule{0.120pt}{0.400pt}}
\put(966,497.67){\rule{0.482pt}{0.400pt}}
\multiput(966.00,498.17)(1.000,-1.000){2}{\rule{0.241pt}{0.400pt}}
\put(968,496.67){\rule{0.241pt}{0.400pt}}
\multiput(968.00,497.17)(0.500,-1.000){2}{\rule{0.120pt}{0.400pt}}
\put(969,495.67){\rule{0.241pt}{0.400pt}}
\multiput(969.00,496.17)(0.500,-1.000){2}{\rule{0.120pt}{0.400pt}}
\put(970,494.67){\rule{0.482pt}{0.400pt}}
\multiput(970.00,495.17)(1.000,-1.000){2}{\rule{0.241pt}{0.400pt}}
\put(972,493.67){\rule{0.241pt}{0.400pt}}
\multiput(972.00,494.17)(0.500,-1.000){2}{\rule{0.120pt}{0.400pt}}
\put(973,492.67){\rule{0.241pt}{0.400pt}}
\multiput(973.00,493.17)(0.500,-1.000){2}{\rule{0.120pt}{0.400pt}}
\put(974,491.67){\rule{0.482pt}{0.400pt}}
\multiput(974.00,492.17)(1.000,-1.000){2}{\rule{0.241pt}{0.400pt}}
\put(976,490.67){\rule{0.241pt}{0.400pt}}
\multiput(976.00,491.17)(0.500,-1.000){2}{\rule{0.120pt}{0.400pt}}
\put(977,489.67){\rule{0.241pt}{0.400pt}}
\multiput(977.00,490.17)(0.500,-1.000){2}{\rule{0.120pt}{0.400pt}}
\put(978,488.67){\rule{0.482pt}{0.400pt}}
\multiput(978.00,489.17)(1.000,-1.000){2}{\rule{0.241pt}{0.400pt}}
\put(979.67,487){\rule{0.400pt}{0.482pt}}
\multiput(979.17,488.00)(1.000,-1.000){2}{\rule{0.400pt}{0.241pt}}
\put(981,485.67){\rule{0.241pt}{0.400pt}}
\multiput(981.00,486.17)(0.500,-1.000){2}{\rule{0.120pt}{0.400pt}}
\put(982,484.67){\rule{0.482pt}{0.400pt}}
\multiput(982.00,485.17)(1.000,-1.000){2}{\rule{0.241pt}{0.400pt}}
\put(984,483.67){\rule{0.241pt}{0.400pt}}
\multiput(984.00,484.17)(0.500,-1.000){2}{\rule{0.120pt}{0.400pt}}
\put(985,482.67){\rule{0.241pt}{0.400pt}}
\multiput(985.00,483.17)(0.500,-1.000){2}{\rule{0.120pt}{0.400pt}}
\put(986,481.67){\rule{0.482pt}{0.400pt}}
\multiput(986.00,482.17)(1.000,-1.000){2}{\rule{0.241pt}{0.400pt}}
\put(988,480.67){\rule{0.241pt}{0.400pt}}
\multiput(988.00,481.17)(0.500,-1.000){2}{\rule{0.120pt}{0.400pt}}
\put(989,479.67){\rule{0.241pt}{0.400pt}}
\multiput(989.00,480.17)(0.500,-1.000){2}{\rule{0.120pt}{0.400pt}}
\put(990,478.67){\rule{0.482pt}{0.400pt}}
\multiput(990.00,479.17)(1.000,-1.000){2}{\rule{0.241pt}{0.400pt}}
\put(992,477.67){\rule{0.241pt}{0.400pt}}
\multiput(992.00,478.17)(0.500,-1.000){2}{\rule{0.120pt}{0.400pt}}
\put(993,476.67){\rule{0.241pt}{0.400pt}}
\multiput(993.00,477.17)(0.500,-1.000){2}{\rule{0.120pt}{0.400pt}}
\put(994,475.67){\rule{0.482pt}{0.400pt}}
\multiput(994.00,476.17)(1.000,-1.000){2}{\rule{0.241pt}{0.400pt}}
\put(996,474.67){\rule{0.241pt}{0.400pt}}
\multiput(996.00,475.17)(0.500,-1.000){2}{\rule{0.120pt}{0.400pt}}
\put(997,473.67){\rule{0.241pt}{0.400pt}}
\multiput(997.00,474.17)(0.500,-1.000){2}{\rule{0.120pt}{0.400pt}}
\put(998,472.67){\rule{0.482pt}{0.400pt}}
\multiput(998.00,473.17)(1.000,-1.000){2}{\rule{0.241pt}{0.400pt}}
\put(1000,471.67){\rule{0.241pt}{0.400pt}}
\multiput(1000.00,472.17)(0.500,-1.000){2}{\rule{0.120pt}{0.400pt}}
\put(1001,470.67){\rule{0.241pt}{0.400pt}}
\multiput(1001.00,471.17)(0.500,-1.000){2}{\rule{0.120pt}{0.400pt}}
\put(1002,469.67){\rule{0.482pt}{0.400pt}}
\multiput(1002.00,470.17)(1.000,-1.000){2}{\rule{0.241pt}{0.400pt}}
\put(1004,468.67){\rule{0.241pt}{0.400pt}}
\multiput(1004.00,469.17)(0.500,-1.000){2}{\rule{0.120pt}{0.400pt}}
\put(1005,467.67){\rule{0.241pt}{0.400pt}}
\multiput(1005.00,468.17)(0.500,-1.000){2}{\rule{0.120pt}{0.400pt}}
\put(1006,466.67){\rule{0.482pt}{0.400pt}}
\multiput(1006.00,467.17)(1.000,-1.000){2}{\rule{0.241pt}{0.400pt}}
\put(1008,465.67){\rule{0.241pt}{0.400pt}}
\multiput(1008.00,466.17)(0.500,-1.000){2}{\rule{0.120pt}{0.400pt}}
\put(1009,464.67){\rule{0.241pt}{0.400pt}}
\multiput(1009.00,465.17)(0.500,-1.000){2}{\rule{0.120pt}{0.400pt}}
\put(1010,463.67){\rule{0.241pt}{0.400pt}}
\multiput(1010.00,464.17)(0.500,-1.000){2}{\rule{0.120pt}{0.400pt}}
\put(1011,462.67){\rule{0.482pt}{0.400pt}}
\multiput(1011.00,463.17)(1.000,-1.000){2}{\rule{0.241pt}{0.400pt}}
\put(1013,461.67){\rule{0.241pt}{0.400pt}}
\multiput(1013.00,462.17)(0.500,-1.000){2}{\rule{0.120pt}{0.400pt}}
\put(1014,460.67){\rule{0.241pt}{0.400pt}}
\multiput(1014.00,461.17)(0.500,-1.000){2}{\rule{0.120pt}{0.400pt}}
\put(1015,459.67){\rule{0.482pt}{0.400pt}}
\multiput(1015.00,460.17)(1.000,-1.000){2}{\rule{0.241pt}{0.400pt}}
\put(1017,458.67){\rule{0.241pt}{0.400pt}}
\multiput(1017.00,459.17)(0.500,-1.000){2}{\rule{0.120pt}{0.400pt}}
\put(1018,457.67){\rule{0.241pt}{0.400pt}}
\multiput(1018.00,458.17)(0.500,-1.000){2}{\rule{0.120pt}{0.400pt}}
\put(1019,456.67){\rule{0.482pt}{0.400pt}}
\multiput(1019.00,457.17)(1.000,-1.000){2}{\rule{0.241pt}{0.400pt}}
\put(1021,455.67){\rule{0.241pt}{0.400pt}}
\multiput(1021.00,456.17)(0.500,-1.000){2}{\rule{0.120pt}{0.400pt}}
\put(1022,454.67){\rule{0.241pt}{0.400pt}}
\multiput(1022.00,455.17)(0.500,-1.000){2}{\rule{0.120pt}{0.400pt}}
\put(1023,453.67){\rule{0.482pt}{0.400pt}}
\multiput(1023.00,454.17)(1.000,-1.000){2}{\rule{0.241pt}{0.400pt}}
\put(1025,452.67){\rule{0.241pt}{0.400pt}}
\multiput(1025.00,453.17)(0.500,-1.000){2}{\rule{0.120pt}{0.400pt}}
\put(1026,451.67){\rule{0.241pt}{0.400pt}}
\multiput(1026.00,452.17)(0.500,-1.000){2}{\rule{0.120pt}{0.400pt}}
\put(1027,450.67){\rule{0.482pt}{0.400pt}}
\multiput(1027.00,451.17)(1.000,-1.000){2}{\rule{0.241pt}{0.400pt}}
\put(1029,449.67){\rule{0.241pt}{0.400pt}}
\multiput(1029.00,450.17)(0.500,-1.000){2}{\rule{0.120pt}{0.400pt}}
\put(1030,448.67){\rule{0.241pt}{0.400pt}}
\multiput(1030.00,449.17)(0.500,-1.000){2}{\rule{0.120pt}{0.400pt}}
\put(1031,447.67){\rule{0.482pt}{0.400pt}}
\multiput(1031.00,448.17)(1.000,-1.000){2}{\rule{0.241pt}{0.400pt}}
\put(1032.67,446){\rule{0.400pt}{0.482pt}}
\multiput(1032.17,447.00)(1.000,-1.000){2}{\rule{0.400pt}{0.241pt}}
\put(1034,444.67){\rule{0.241pt}{0.400pt}}
\multiput(1034.00,445.17)(0.500,-1.000){2}{\rule{0.120pt}{0.400pt}}
\put(1035,443.67){\rule{0.482pt}{0.400pt}}
\multiput(1035.00,444.17)(1.000,-1.000){2}{\rule{0.241pt}{0.400pt}}
\put(1037,442.67){\rule{0.241pt}{0.400pt}}
\multiput(1037.00,443.17)(0.500,-1.000){2}{\rule{0.120pt}{0.400pt}}
\put(1038,441.67){\rule{0.241pt}{0.400pt}}
\multiput(1038.00,442.17)(0.500,-1.000){2}{\rule{0.120pt}{0.400pt}}
\put(1039,440.67){\rule{0.482pt}{0.400pt}}
\multiput(1039.00,441.17)(1.000,-1.000){2}{\rule{0.241pt}{0.400pt}}
\put(1041,439.67){\rule{0.241pt}{0.400pt}}
\multiput(1041.00,440.17)(0.500,-1.000){2}{\rule{0.120pt}{0.400pt}}
\put(1042,438.67){\rule{0.241pt}{0.400pt}}
\multiput(1042.00,439.17)(0.500,-1.000){2}{\rule{0.120pt}{0.400pt}}
\put(1043,437.67){\rule{0.482pt}{0.400pt}}
\multiput(1043.00,438.17)(1.000,-1.000){2}{\rule{0.241pt}{0.400pt}}
\put(1045,436.67){\rule{0.241pt}{0.400pt}}
\multiput(1045.00,437.17)(0.500,-1.000){2}{\rule{0.120pt}{0.400pt}}
\put(1046,435.67){\rule{0.241pt}{0.400pt}}
\multiput(1046.00,436.17)(0.500,-1.000){2}{\rule{0.120pt}{0.400pt}}
\put(1047,434.67){\rule{0.482pt}{0.400pt}}
\multiput(1047.00,435.17)(1.000,-1.000){2}{\rule{0.241pt}{0.400pt}}
\put(1049,433.67){\rule{0.241pt}{0.400pt}}
\multiput(1049.00,434.17)(0.500,-1.000){2}{\rule{0.120pt}{0.400pt}}
\put(1050,432.67){\rule{0.241pt}{0.400pt}}
\multiput(1050.00,433.17)(0.500,-1.000){2}{\rule{0.120pt}{0.400pt}}
\put(1051,431.67){\rule{0.482pt}{0.400pt}}
\multiput(1051.00,432.17)(1.000,-1.000){2}{\rule{0.241pt}{0.400pt}}
\put(1053,430.67){\rule{0.241pt}{0.400pt}}
\multiput(1053.00,431.17)(0.500,-1.000){2}{\rule{0.120pt}{0.400pt}}
\put(1054,429.67){\rule{0.241pt}{0.400pt}}
\multiput(1054.00,430.17)(0.500,-1.000){2}{\rule{0.120pt}{0.400pt}}
\put(1055,428.67){\rule{0.482pt}{0.400pt}}
\multiput(1055.00,429.17)(1.000,-1.000){2}{\rule{0.241pt}{0.400pt}}
\put(1057,427.67){\rule{0.241pt}{0.400pt}}
\multiput(1057.00,428.17)(0.500,-1.000){2}{\rule{0.120pt}{0.400pt}}
\put(1058,426.67){\rule{0.241pt}{0.400pt}}
\multiput(1058.00,427.17)(0.500,-1.000){2}{\rule{0.120pt}{0.400pt}}
\put(1059,425.67){\rule{0.482pt}{0.400pt}}
\multiput(1059.00,426.17)(1.000,-1.000){2}{\rule{0.241pt}{0.400pt}}
\put(1061,424.67){\rule{0.241pt}{0.400pt}}
\multiput(1061.00,425.17)(0.500,-1.000){2}{\rule{0.120pt}{0.400pt}}
\put(1062,423.67){\rule{0.241pt}{0.400pt}}
\multiput(1062.00,424.17)(0.500,-1.000){2}{\rule{0.120pt}{0.400pt}}
\put(1063,422.67){\rule{0.482pt}{0.400pt}}
\multiput(1063.00,423.17)(1.000,-1.000){2}{\rule{0.241pt}{0.400pt}}
\put(1065,421.67){\rule{0.241pt}{0.400pt}}
\multiput(1065.00,422.17)(0.500,-1.000){2}{\rule{0.120pt}{0.400pt}}
\put(1066,420.67){\rule{0.241pt}{0.400pt}}
\multiput(1066.00,421.17)(0.500,-1.000){2}{\rule{0.120pt}{0.400pt}}
\put(1067,419.67){\rule{0.482pt}{0.400pt}}
\multiput(1067.00,420.17)(1.000,-1.000){2}{\rule{0.241pt}{0.400pt}}
\put(1069,418.67){\rule{0.241pt}{0.400pt}}
\multiput(1069.00,419.17)(0.500,-1.000){2}{\rule{0.120pt}{0.400pt}}
\put(1070,417.67){\rule{0.241pt}{0.400pt}}
\multiput(1070.00,418.17)(0.500,-1.000){2}{\rule{0.120pt}{0.400pt}}
\put(1071,416.67){\rule{0.482pt}{0.400pt}}
\multiput(1071.00,417.17)(1.000,-1.000){2}{\rule{0.241pt}{0.400pt}}
\put(1073,415.67){\rule{0.241pt}{0.400pt}}
\multiput(1073.00,416.17)(0.500,-1.000){2}{\rule{0.120pt}{0.400pt}}
\put(1074,414.67){\rule{0.241pt}{0.400pt}}
\multiput(1074.00,415.17)(0.500,-1.000){2}{\rule{0.120pt}{0.400pt}}
\put(1075,413.67){\rule{0.482pt}{0.400pt}}
\multiput(1075.00,414.17)(1.000,-1.000){2}{\rule{0.241pt}{0.400pt}}
\put(1077,412.67){\rule{0.241pt}{0.400pt}}
\multiput(1077.00,413.17)(0.500,-1.000){2}{\rule{0.120pt}{0.400pt}}
\put(1077.67,411){\rule{0.400pt}{0.482pt}}
\multiput(1077.17,412.00)(1.000,-1.000){2}{\rule{0.400pt}{0.241pt}}
\put(1079,409.67){\rule{0.482pt}{0.400pt}}
\multiput(1079.00,410.17)(1.000,-1.000){2}{\rule{0.241pt}{0.400pt}}
\put(1081,408.67){\rule{0.241pt}{0.400pt}}
\multiput(1081.00,409.17)(0.500,-1.000){2}{\rule{0.120pt}{0.400pt}}
\put(1082,407.67){\rule{0.241pt}{0.400pt}}
\multiput(1082.00,408.17)(0.500,-1.000){2}{\rule{0.120pt}{0.400pt}}
\put(1083,406.67){\rule{0.482pt}{0.400pt}}
\multiput(1083.00,407.17)(1.000,-1.000){2}{\rule{0.241pt}{0.400pt}}
\put(1085,405.67){\rule{0.241pt}{0.400pt}}
\multiput(1085.00,406.17)(0.500,-1.000){2}{\rule{0.120pt}{0.400pt}}
\put(1086,404.67){\rule{0.241pt}{0.400pt}}
\multiput(1086.00,405.17)(0.500,-1.000){2}{\rule{0.120pt}{0.400pt}}
\put(1087,403.67){\rule{0.241pt}{0.400pt}}
\multiput(1087.00,404.17)(0.500,-1.000){2}{\rule{0.120pt}{0.400pt}}
\put(1088,402.67){\rule{0.482pt}{0.400pt}}
\multiput(1088.00,403.17)(1.000,-1.000){2}{\rule{0.241pt}{0.400pt}}
\put(1090,401.67){\rule{0.241pt}{0.400pt}}
\multiput(1090.00,402.17)(0.500,-1.000){2}{\rule{0.120pt}{0.400pt}}
\put(1091,400.67){\rule{0.241pt}{0.400pt}}
\multiput(1091.00,401.17)(0.500,-1.000){2}{\rule{0.120pt}{0.400pt}}
\put(1092,399.67){\rule{0.482pt}{0.400pt}}
\multiput(1092.00,400.17)(1.000,-1.000){2}{\rule{0.241pt}{0.400pt}}
\put(1094,398.67){\rule{0.241pt}{0.400pt}}
\multiput(1094.00,399.17)(0.500,-1.000){2}{\rule{0.120pt}{0.400pt}}
\put(1095,397.67){\rule{0.241pt}{0.400pt}}
\multiput(1095.00,398.17)(0.500,-1.000){2}{\rule{0.120pt}{0.400pt}}
\put(1096,396.67){\rule{0.482pt}{0.400pt}}
\multiput(1096.00,397.17)(1.000,-1.000){2}{\rule{0.241pt}{0.400pt}}
\put(1098,395.67){\rule{0.241pt}{0.400pt}}
\multiput(1098.00,396.17)(0.500,-1.000){2}{\rule{0.120pt}{0.400pt}}
\put(1099,394.67){\rule{0.241pt}{0.400pt}}
\multiput(1099.00,395.17)(0.500,-1.000){2}{\rule{0.120pt}{0.400pt}}
\put(1100,393.67){\rule{0.482pt}{0.400pt}}
\multiput(1100.00,394.17)(1.000,-1.000){2}{\rule{0.241pt}{0.400pt}}
\put(1102,392.67){\rule{0.241pt}{0.400pt}}
\multiput(1102.00,393.17)(0.500,-1.000){2}{\rule{0.120pt}{0.400pt}}
\put(1103,391.67){\rule{0.241pt}{0.400pt}}
\multiput(1103.00,392.17)(0.500,-1.000){2}{\rule{0.120pt}{0.400pt}}
\put(1104,390.67){\rule{0.482pt}{0.400pt}}
\multiput(1104.00,391.17)(1.000,-1.000){2}{\rule{0.241pt}{0.400pt}}
\put(1106,389.67){\rule{0.241pt}{0.400pt}}
\multiput(1106.00,390.17)(0.500,-1.000){2}{\rule{0.120pt}{0.400pt}}
\put(1107,388.67){\rule{0.241pt}{0.400pt}}
\multiput(1107.00,389.17)(0.500,-1.000){2}{\rule{0.120pt}{0.400pt}}
\put(1108,387.67){\rule{0.482pt}{0.400pt}}
\multiput(1108.00,388.17)(1.000,-1.000){2}{\rule{0.241pt}{0.400pt}}
\put(1110,386.67){\rule{0.241pt}{0.400pt}}
\multiput(1110.00,387.17)(0.500,-1.000){2}{\rule{0.120pt}{0.400pt}}
\put(1111,385.67){\rule{0.241pt}{0.400pt}}
\multiput(1111.00,386.17)(0.500,-1.000){2}{\rule{0.120pt}{0.400pt}}
\put(1112,384.67){\rule{0.482pt}{0.400pt}}
\multiput(1112.00,385.17)(1.000,-1.000){2}{\rule{0.241pt}{0.400pt}}
\put(1114,383.67){\rule{0.241pt}{0.400pt}}
\multiput(1114.00,384.17)(0.500,-1.000){2}{\rule{0.120pt}{0.400pt}}
\put(1115,382.67){\rule{0.241pt}{0.400pt}}
\multiput(1115.00,383.17)(0.500,-1.000){2}{\rule{0.120pt}{0.400pt}}
\put(1116,381.67){\rule{0.482pt}{0.400pt}}
\multiput(1116.00,382.17)(1.000,-1.000){2}{\rule{0.241pt}{0.400pt}}
\put(1118,380.67){\rule{0.241pt}{0.400pt}}
\multiput(1118.00,381.17)(0.500,-1.000){2}{\rule{0.120pt}{0.400pt}}
\put(1119,379.67){\rule{0.241pt}{0.400pt}}
\multiput(1119.00,380.17)(0.500,-1.000){2}{\rule{0.120pt}{0.400pt}}
\put(1120,378.17){\rule{0.482pt}{0.400pt}}
\multiput(1120.00,379.17)(1.000,-2.000){2}{\rule{0.241pt}{0.400pt}}
\put(1122,376.67){\rule{0.241pt}{0.400pt}}
\multiput(1122.00,377.17)(0.500,-1.000){2}{\rule{0.120pt}{0.400pt}}
\put(1123,375.67){\rule{0.241pt}{0.400pt}}
\multiput(1123.00,376.17)(0.500,-1.000){2}{\rule{0.120pt}{0.400pt}}
\put(1124,374.67){\rule{0.482pt}{0.400pt}}
\multiput(1124.00,375.17)(1.000,-1.000){2}{\rule{0.241pt}{0.400pt}}
\put(1126,373.67){\rule{0.241pt}{0.400pt}}
\multiput(1126.00,374.17)(0.500,-1.000){2}{\rule{0.120pt}{0.400pt}}
\put(1127,372.67){\rule{0.241pt}{0.400pt}}
\multiput(1127.00,373.17)(0.500,-1.000){2}{\rule{0.120pt}{0.400pt}}
\put(1128,371.67){\rule{0.482pt}{0.400pt}}
\multiput(1128.00,372.17)(1.000,-1.000){2}{\rule{0.241pt}{0.400pt}}
\put(1130,370.67){\rule{0.241pt}{0.400pt}}
\multiput(1130.00,371.17)(0.500,-1.000){2}{\rule{0.120pt}{0.400pt}}
\put(1131,369.67){\rule{0.241pt}{0.400pt}}
\multiput(1131.00,370.17)(0.500,-1.000){2}{\rule{0.120pt}{0.400pt}}
\put(1132,368.67){\rule{0.482pt}{0.400pt}}
\multiput(1132.00,369.17)(1.000,-1.000){2}{\rule{0.241pt}{0.400pt}}
\put(1134,367.67){\rule{0.241pt}{0.400pt}}
\multiput(1134.00,368.17)(0.500,-1.000){2}{\rule{0.120pt}{0.400pt}}
\put(1135,366.67){\rule{0.241pt}{0.400pt}}
\multiput(1135.00,367.17)(0.500,-1.000){2}{\rule{0.120pt}{0.400pt}}
\put(1136,365.67){\rule{0.482pt}{0.400pt}}
\multiput(1136.00,366.17)(1.000,-1.000){2}{\rule{0.241pt}{0.400pt}}
\put(1138,364.67){\rule{0.241pt}{0.400pt}}
\multiput(1138.00,365.17)(0.500,-1.000){2}{\rule{0.120pt}{0.400pt}}
\put(1139,363.67){\rule{0.241pt}{0.400pt}}
\multiput(1139.00,364.17)(0.500,-1.000){2}{\rule{0.120pt}{0.400pt}}
\put(1140,362.67){\rule{0.482pt}{0.400pt}}
\multiput(1140.00,363.17)(1.000,-1.000){2}{\rule{0.241pt}{0.400pt}}
\put(1142,361.67){\rule{0.241pt}{0.400pt}}
\multiput(1142.00,362.17)(0.500,-1.000){2}{\rule{0.120pt}{0.400pt}}
\put(1143,360.67){\rule{0.241pt}{0.400pt}}
\multiput(1143.00,361.17)(0.500,-1.000){2}{\rule{0.120pt}{0.400pt}}
\put(1144,359.67){\rule{0.482pt}{0.400pt}}
\multiput(1144.00,360.17)(1.000,-1.000){2}{\rule{0.241pt}{0.400pt}}
\put(1146,358.67){\rule{0.241pt}{0.400pt}}
\multiput(1146.00,359.17)(0.500,-1.000){2}{\rule{0.120pt}{0.400pt}}
\put(1147,357.67){\rule{0.241pt}{0.400pt}}
\multiput(1147.00,358.17)(0.500,-1.000){2}{\rule{0.120pt}{0.400pt}}
\put(1148,356.67){\rule{0.482pt}{0.400pt}}
\multiput(1148.00,357.17)(1.000,-1.000){2}{\rule{0.241pt}{0.400pt}}
\put(1150,355.67){\rule{0.241pt}{0.400pt}}
\multiput(1150.00,356.17)(0.500,-1.000){2}{\rule{0.120pt}{0.400pt}}
\put(1151,354.67){\rule{0.241pt}{0.400pt}}
\multiput(1151.00,355.17)(0.500,-1.000){2}{\rule{0.120pt}{0.400pt}}
\put(1152,353.67){\rule{0.482pt}{0.400pt}}
\multiput(1152.00,354.17)(1.000,-1.000){2}{\rule{0.241pt}{0.400pt}}
\put(1154,352.67){\rule{0.241pt}{0.400pt}}
\multiput(1154.00,353.17)(0.500,-1.000){2}{\rule{0.120pt}{0.400pt}}
\put(1155,351.67){\rule{0.241pt}{0.400pt}}
\multiput(1155.00,352.17)(0.500,-1.000){2}{\rule{0.120pt}{0.400pt}}
\put(1156,350.67){\rule{0.482pt}{0.400pt}}
\multiput(1156.00,351.17)(1.000,-1.000){2}{\rule{0.241pt}{0.400pt}}
\put(1158,349.67){\rule{0.241pt}{0.400pt}}
\multiput(1158.00,350.17)(0.500,-1.000){2}{\rule{0.120pt}{0.400pt}}
\put(1159,348.67){\rule{0.241pt}{0.400pt}}
\multiput(1159.00,349.17)(0.500,-1.000){2}{\rule{0.120pt}{0.400pt}}
\put(1160,347.17){\rule{0.482pt}{0.400pt}}
\multiput(1160.00,348.17)(1.000,-2.000){2}{\rule{0.241pt}{0.400pt}}
\put(1162,345.67){\rule{0.241pt}{0.400pt}}
\multiput(1162.00,346.17)(0.500,-1.000){2}{\rule{0.120pt}{0.400pt}}
\put(1163,344.67){\rule{0.241pt}{0.400pt}}
\multiput(1163.00,345.17)(0.500,-1.000){2}{\rule{0.120pt}{0.400pt}}
\put(1164,343.67){\rule{0.241pt}{0.400pt}}
\multiput(1164.00,344.17)(0.500,-1.000){2}{\rule{0.120pt}{0.400pt}}
\put(1165,342.67){\rule{0.482pt}{0.400pt}}
\multiput(1165.00,343.17)(1.000,-1.000){2}{\rule{0.241pt}{0.400pt}}
\put(1167,341.67){\rule{0.241pt}{0.400pt}}
\multiput(1167.00,342.17)(0.500,-1.000){2}{\rule{0.120pt}{0.400pt}}
\put(1168,340.67){\rule{0.241pt}{0.400pt}}
\multiput(1168.00,341.17)(0.500,-1.000){2}{\rule{0.120pt}{0.400pt}}
\put(1169,339.67){\rule{0.482pt}{0.400pt}}
\multiput(1169.00,340.17)(1.000,-1.000){2}{\rule{0.241pt}{0.400pt}}
\put(1171,338.67){\rule{0.241pt}{0.400pt}}
\multiput(1171.00,339.17)(0.500,-1.000){2}{\rule{0.120pt}{0.400pt}}
\put(1172,337.67){\rule{0.241pt}{0.400pt}}
\multiput(1172.00,338.17)(0.500,-1.000){2}{\rule{0.120pt}{0.400pt}}
\put(1173,336.67){\rule{0.482pt}{0.400pt}}
\multiput(1173.00,337.17)(1.000,-1.000){2}{\rule{0.241pt}{0.400pt}}
\put(1175,335.67){\rule{0.241pt}{0.400pt}}
\multiput(1175.00,336.17)(0.500,-1.000){2}{\rule{0.120pt}{0.400pt}}
\put(1176,334.67){\rule{0.241pt}{0.400pt}}
\multiput(1176.00,335.17)(0.500,-1.000){2}{\rule{0.120pt}{0.400pt}}
\put(1177,333.67){\rule{0.482pt}{0.400pt}}
\multiput(1177.00,334.17)(1.000,-1.000){2}{\rule{0.241pt}{0.400pt}}
\put(1179,332.67){\rule{0.241pt}{0.400pt}}
\multiput(1179.00,333.17)(0.500,-1.000){2}{\rule{0.120pt}{0.400pt}}
\put(1180,331.67){\rule{0.241pt}{0.400pt}}
\multiput(1180.00,332.17)(0.500,-1.000){2}{\rule{0.120pt}{0.400pt}}
\put(1181,330.67){\rule{0.482pt}{0.400pt}}
\multiput(1181.00,331.17)(1.000,-1.000){2}{\rule{0.241pt}{0.400pt}}
\put(1183,329.67){\rule{0.241pt}{0.400pt}}
\multiput(1183.00,330.17)(0.500,-1.000){2}{\rule{0.120pt}{0.400pt}}
\put(1184,328.67){\rule{0.241pt}{0.400pt}}
\multiput(1184.00,329.17)(0.500,-1.000){2}{\rule{0.120pt}{0.400pt}}
\put(1185,327.67){\rule{0.482pt}{0.400pt}}
\multiput(1185.00,328.17)(1.000,-1.000){2}{\rule{0.241pt}{0.400pt}}
\put(1187,326.67){\rule{0.241pt}{0.400pt}}
\multiput(1187.00,327.17)(0.500,-1.000){2}{\rule{0.120pt}{0.400pt}}
\put(1188,325.67){\rule{0.241pt}{0.400pt}}
\multiput(1188.00,326.17)(0.500,-1.000){2}{\rule{0.120pt}{0.400pt}}
\put(1189,324.67){\rule{0.482pt}{0.400pt}}
\multiput(1189.00,325.17)(1.000,-1.000){2}{\rule{0.241pt}{0.400pt}}
\put(1191,323.67){\rule{0.241pt}{0.400pt}}
\multiput(1191.00,324.17)(0.500,-1.000){2}{\rule{0.120pt}{0.400pt}}
\put(1192,322.67){\rule{0.241pt}{0.400pt}}
\multiput(1192.00,323.17)(0.500,-1.000){2}{\rule{0.120pt}{0.400pt}}
\put(1193,321.67){\rule{0.482pt}{0.400pt}}
\multiput(1193.00,322.17)(1.000,-1.000){2}{\rule{0.241pt}{0.400pt}}
\put(1195,320.67){\rule{0.241pt}{0.400pt}}
\multiput(1195.00,321.17)(0.500,-1.000){2}{\rule{0.120pt}{0.400pt}}
\put(1196,319.67){\rule{0.241pt}{0.400pt}}
\multiput(1196.00,320.17)(0.500,-1.000){2}{\rule{0.120pt}{0.400pt}}
\put(1197,318.67){\rule{0.482pt}{0.400pt}}
\multiput(1197.00,319.17)(1.000,-1.000){2}{\rule{0.241pt}{0.400pt}}
\put(1199,317.67){\rule{0.241pt}{0.400pt}}
\multiput(1199.00,318.17)(0.500,-1.000){2}{\rule{0.120pt}{0.400pt}}
\put(1199.67,316){\rule{0.400pt}{0.482pt}}
\multiput(1199.17,317.00)(1.000,-1.000){2}{\rule{0.400pt}{0.241pt}}
\put(1201,314.67){\rule{0.482pt}{0.400pt}}
\multiput(1201.00,315.17)(1.000,-1.000){2}{\rule{0.241pt}{0.400pt}}
\put(1203,313.67){\rule{0.241pt}{0.400pt}}
\multiput(1203.00,314.17)(0.500,-1.000){2}{\rule{0.120pt}{0.400pt}}
\put(1204,312.67){\rule{0.241pt}{0.400pt}}
\multiput(1204.00,313.17)(0.500,-1.000){2}{\rule{0.120pt}{0.400pt}}
\put(1205,311.67){\rule{0.482pt}{0.400pt}}
\multiput(1205.00,312.17)(1.000,-1.000){2}{\rule{0.241pt}{0.400pt}}
\put(1207,310.67){\rule{0.241pt}{0.400pt}}
\multiput(1207.00,311.17)(0.500,-1.000){2}{\rule{0.120pt}{0.400pt}}
\put(1208,309.67){\rule{0.241pt}{0.400pt}}
\multiput(1208.00,310.17)(0.500,-1.000){2}{\rule{0.120pt}{0.400pt}}
\put(1209,308.67){\rule{0.482pt}{0.400pt}}
\multiput(1209.00,309.17)(1.000,-1.000){2}{\rule{0.241pt}{0.400pt}}
\put(1211,307.67){\rule{0.241pt}{0.400pt}}
\multiput(1211.00,308.17)(0.500,-1.000){2}{\rule{0.120pt}{0.400pt}}
\put(1212,306.67){\rule{0.241pt}{0.400pt}}
\multiput(1212.00,307.17)(0.500,-1.000){2}{\rule{0.120pt}{0.400pt}}
\put(1213,305.67){\rule{0.482pt}{0.400pt}}
\multiput(1213.00,306.17)(1.000,-1.000){2}{\rule{0.241pt}{0.400pt}}
\put(1215,304.67){\rule{0.241pt}{0.400pt}}
\multiput(1215.00,305.17)(0.500,-1.000){2}{\rule{0.120pt}{0.400pt}}
\put(1216,303.67){\rule{0.241pt}{0.400pt}}
\multiput(1216.00,304.17)(0.500,-1.000){2}{\rule{0.120pt}{0.400pt}}
\put(1217,302.67){\rule{0.482pt}{0.400pt}}
\multiput(1217.00,303.17)(1.000,-1.000){2}{\rule{0.241pt}{0.400pt}}
\put(1219,301.67){\rule{0.241pt}{0.400pt}}
\multiput(1219.00,302.17)(0.500,-1.000){2}{\rule{0.120pt}{0.400pt}}
\put(1220,300.67){\rule{0.241pt}{0.400pt}}
\multiput(1220.00,301.17)(0.500,-1.000){2}{\rule{0.120pt}{0.400pt}}
\put(1221,299.67){\rule{0.482pt}{0.400pt}}
\multiput(1221.00,300.17)(1.000,-1.000){2}{\rule{0.241pt}{0.400pt}}
\put(1223,298.67){\rule{0.241pt}{0.400pt}}
\multiput(1223.00,299.17)(0.500,-1.000){2}{\rule{0.120pt}{0.400pt}}
\put(1224,297.67){\rule{0.241pt}{0.400pt}}
\multiput(1224.00,298.17)(0.500,-1.000){2}{\rule{0.120pt}{0.400pt}}
\put(1225,296.67){\rule{0.482pt}{0.400pt}}
\multiput(1225.00,297.17)(1.000,-1.000){2}{\rule{0.241pt}{0.400pt}}
\put(1227,295.67){\rule{0.241pt}{0.400pt}}
\multiput(1227.00,296.17)(0.500,-1.000){2}{\rule{0.120pt}{0.400pt}}
\put(1228,294.67){\rule{0.241pt}{0.400pt}}
\multiput(1228.00,295.17)(0.500,-1.000){2}{\rule{0.120pt}{0.400pt}}
\put(1229,293.67){\rule{0.482pt}{0.400pt}}
\multiput(1229.00,294.17)(1.000,-1.000){2}{\rule{0.241pt}{0.400pt}}
\put(1231,292.67){\rule{0.241pt}{0.400pt}}
\multiput(1231.00,293.17)(0.500,-1.000){2}{\rule{0.120pt}{0.400pt}}
\put(1232,291.67){\rule{0.241pt}{0.400pt}}
\multiput(1232.00,292.17)(0.500,-1.000){2}{\rule{0.120pt}{0.400pt}}
\put(1233,290.67){\rule{0.482pt}{0.400pt}}
\multiput(1233.00,291.17)(1.000,-1.000){2}{\rule{0.241pt}{0.400pt}}
\put(1235,289.67){\rule{0.241pt}{0.400pt}}
\multiput(1235.00,290.17)(0.500,-1.000){2}{\rule{0.120pt}{0.400pt}}
\put(1236,288.67){\rule{0.241pt}{0.400pt}}
\multiput(1236.00,289.17)(0.500,-1.000){2}{\rule{0.120pt}{0.400pt}}
\put(1237,287.17){\rule{0.482pt}{0.400pt}}
\multiput(1237.00,288.17)(1.000,-2.000){2}{\rule{0.241pt}{0.400pt}}
\put(1239,285.67){\rule{0.241pt}{0.400pt}}
\multiput(1239.00,286.17)(0.500,-1.000){2}{\rule{0.120pt}{0.400pt}}
\put(1240,284.67){\rule{0.241pt}{0.400pt}}
\multiput(1240.00,285.17)(0.500,-1.000){2}{\rule{0.120pt}{0.400pt}}
\put(1241,283.67){\rule{0.482pt}{0.400pt}}
\multiput(1241.00,284.17)(1.000,-1.000){2}{\rule{0.241pt}{0.400pt}}
\put(1243,282.67){\rule{0.241pt}{0.400pt}}
\multiput(1243.00,283.17)(0.500,-1.000){2}{\rule{0.120pt}{0.400pt}}
\put(1244,281.67){\rule{0.241pt}{0.400pt}}
\multiput(1244.00,282.17)(0.500,-1.000){2}{\rule{0.120pt}{0.400pt}}
\put(1245,280.67){\rule{0.241pt}{0.400pt}}
\multiput(1245.00,281.17)(0.500,-1.000){2}{\rule{0.120pt}{0.400pt}}
\put(1246,279.67){\rule{0.482pt}{0.400pt}}
\multiput(1246.00,280.17)(1.000,-1.000){2}{\rule{0.241pt}{0.400pt}}
\put(1248,278.67){\rule{0.241pt}{0.400pt}}
\multiput(1248.00,279.17)(0.500,-1.000){2}{\rule{0.120pt}{0.400pt}}
\put(1249,277.67){\rule{0.241pt}{0.400pt}}
\multiput(1249.00,278.17)(0.500,-1.000){2}{\rule{0.120pt}{0.400pt}}
\put(1250,276.67){\rule{0.482pt}{0.400pt}}
\multiput(1250.00,277.17)(1.000,-1.000){2}{\rule{0.241pt}{0.400pt}}
\put(1252,275.67){\rule{0.241pt}{0.400pt}}
\multiput(1252.00,276.17)(0.500,-1.000){2}{\rule{0.120pt}{0.400pt}}
\put(1253,274.67){\rule{0.241pt}{0.400pt}}
\multiput(1253.00,275.17)(0.500,-1.000){2}{\rule{0.120pt}{0.400pt}}
\put(1254,273.67){\rule{0.482pt}{0.400pt}}
\multiput(1254.00,274.17)(1.000,-1.000){2}{\rule{0.241pt}{0.400pt}}
\put(1256,272.67){\rule{0.241pt}{0.400pt}}
\multiput(1256.00,273.17)(0.500,-1.000){2}{\rule{0.120pt}{0.400pt}}
\put(1257,271.67){\rule{0.241pt}{0.400pt}}
\multiput(1257.00,272.17)(0.500,-1.000){2}{\rule{0.120pt}{0.400pt}}
\put(1258,270.67){\rule{0.482pt}{0.400pt}}
\multiput(1258.00,271.17)(1.000,-1.000){2}{\rule{0.241pt}{0.400pt}}
\put(1260,269.67){\rule{0.241pt}{0.400pt}}
\multiput(1260.00,270.17)(0.500,-1.000){2}{\rule{0.120pt}{0.400pt}}
\put(1261,268.67){\rule{0.241pt}{0.400pt}}
\multiput(1261.00,269.17)(0.500,-1.000){2}{\rule{0.120pt}{0.400pt}}
\put(1262,267.67){\rule{0.482pt}{0.400pt}}
\multiput(1262.00,268.17)(1.000,-1.000){2}{\rule{0.241pt}{0.400pt}}
\put(1264,266.67){\rule{0.241pt}{0.400pt}}
\multiput(1264.00,267.17)(0.500,-1.000){2}{\rule{0.120pt}{0.400pt}}
\put(1265,265.67){\rule{0.241pt}{0.400pt}}
\multiput(1265.00,266.17)(0.500,-1.000){2}{\rule{0.120pt}{0.400pt}}
\put(1266,264.67){\rule{0.482pt}{0.400pt}}
\multiput(1266.00,265.17)(1.000,-1.000){2}{\rule{0.241pt}{0.400pt}}
\put(1268,263.67){\rule{0.241pt}{0.400pt}}
\multiput(1268.00,264.17)(0.500,-1.000){2}{\rule{0.120pt}{0.400pt}}
\put(1269,262.67){\rule{0.241pt}{0.400pt}}
\multiput(1269.00,263.17)(0.500,-1.000){2}{\rule{0.120pt}{0.400pt}}
\put(1270,261.67){\rule{0.482pt}{0.400pt}}
\multiput(1270.00,262.17)(1.000,-1.000){2}{\rule{0.241pt}{0.400pt}}
\put(1272,260.67){\rule{0.241pt}{0.400pt}}
\multiput(1272.00,261.17)(0.500,-1.000){2}{\rule{0.120pt}{0.400pt}}
\put(1273,259.67){\rule{0.241pt}{0.400pt}}
\multiput(1273.00,260.17)(0.500,-1.000){2}{\rule{0.120pt}{0.400pt}}
\put(1274,258.17){\rule{0.482pt}{0.400pt}}
\multiput(1274.00,259.17)(1.000,-2.000){2}{\rule{0.241pt}{0.400pt}}
\put(1276,256.67){\rule{0.241pt}{0.400pt}}
\multiput(1276.00,257.17)(0.500,-1.000){2}{\rule{0.120pt}{0.400pt}}
\put(1277,255.67){\rule{0.241pt}{0.400pt}}
\multiput(1277.00,256.17)(0.500,-1.000){2}{\rule{0.120pt}{0.400pt}}
\put(1278,254.67){\rule{0.482pt}{0.400pt}}
\multiput(1278.00,255.17)(1.000,-1.000){2}{\rule{0.241pt}{0.400pt}}
\put(1280,253.67){\rule{0.241pt}{0.400pt}}
\multiput(1280.00,254.17)(0.500,-1.000){2}{\rule{0.120pt}{0.400pt}}
\put(1281,252.67){\rule{0.241pt}{0.400pt}}
\multiput(1281.00,253.17)(0.500,-1.000){2}{\rule{0.120pt}{0.400pt}}
\put(1282,251.67){\rule{0.482pt}{0.400pt}}
\multiput(1282.00,252.17)(1.000,-1.000){2}{\rule{0.241pt}{0.400pt}}
\put(1284,250.67){\rule{0.241pt}{0.400pt}}
\multiput(1284.00,251.17)(0.500,-1.000){2}{\rule{0.120pt}{0.400pt}}
\put(1285,249.67){\rule{0.241pt}{0.400pt}}
\multiput(1285.00,250.17)(0.500,-1.000){2}{\rule{0.120pt}{0.400pt}}
\put(1286,248.67){\rule{0.482pt}{0.400pt}}
\multiput(1286.00,249.17)(1.000,-1.000){2}{\rule{0.241pt}{0.400pt}}
\put(1288,247.67){\rule{0.241pt}{0.400pt}}
\multiput(1288.00,248.17)(0.500,-1.000){2}{\rule{0.120pt}{0.400pt}}
\put(1289,246.67){\rule{0.241pt}{0.400pt}}
\multiput(1289.00,247.17)(0.500,-1.000){2}{\rule{0.120pt}{0.400pt}}
\put(1290,245.67){\rule{0.482pt}{0.400pt}}
\multiput(1290.00,246.17)(1.000,-1.000){2}{\rule{0.241pt}{0.400pt}}
\put(1292,244.67){\rule{0.241pt}{0.400pt}}
\multiput(1292.00,245.17)(0.500,-1.000){2}{\rule{0.120pt}{0.400pt}}
\put(1293,243.67){\rule{0.241pt}{0.400pt}}
\multiput(1293.00,244.17)(0.500,-1.000){2}{\rule{0.120pt}{0.400pt}}
\put(1294,242.67){\rule{0.482pt}{0.400pt}}
\multiput(1294.00,243.17)(1.000,-1.000){2}{\rule{0.241pt}{0.400pt}}
\put(1296,241.67){\rule{0.241pt}{0.400pt}}
\multiput(1296.00,242.17)(0.500,-1.000){2}{\rule{0.120pt}{0.400pt}}
\put(1297,240.67){\rule{0.241pt}{0.400pt}}
\multiput(1297.00,241.17)(0.500,-1.000){2}{\rule{0.120pt}{0.400pt}}
\put(1298,239.67){\rule{0.482pt}{0.400pt}}
\multiput(1298.00,240.17)(1.000,-1.000){2}{\rule{0.241pt}{0.400pt}}
\put(1300,238.67){\rule{0.241pt}{0.400pt}}
\multiput(1300.00,239.17)(0.500,-1.000){2}{\rule{0.120pt}{0.400pt}}
\put(1301,237.67){\rule{0.241pt}{0.400pt}}
\multiput(1301.00,238.17)(0.500,-1.000){2}{\rule{0.120pt}{0.400pt}}
\put(1302,236.67){\rule{0.482pt}{0.400pt}}
\multiput(1302.00,237.17)(1.000,-1.000){2}{\rule{0.241pt}{0.400pt}}
\put(1304,235.67){\rule{0.241pt}{0.400pt}}
\multiput(1304.00,236.17)(0.500,-1.000){2}{\rule{0.120pt}{0.400pt}}
\put(1305,234.67){\rule{0.241pt}{0.400pt}}
\multiput(1305.00,235.17)(0.500,-1.000){2}{\rule{0.120pt}{0.400pt}}
\put(1306,233.67){\rule{0.482pt}{0.400pt}}
\multiput(1306.00,234.17)(1.000,-1.000){2}{\rule{0.241pt}{0.400pt}}
\put(1308,232.67){\rule{0.241pt}{0.400pt}}
\multiput(1308.00,233.17)(0.500,-1.000){2}{\rule{0.120pt}{0.400pt}}
\put(1309,231.67){\rule{0.241pt}{0.400pt}}
\multiput(1309.00,232.17)(0.500,-1.000){2}{\rule{0.120pt}{0.400pt}}
\put(1310,230.67){\rule{0.482pt}{0.400pt}}
\multiput(1310.00,231.17)(1.000,-1.000){2}{\rule{0.241pt}{0.400pt}}
\put(1311.67,229){\rule{0.400pt}{0.482pt}}
\multiput(1311.17,230.00)(1.000,-1.000){2}{\rule{0.400pt}{0.241pt}}
\put(1313,227.67){\rule{0.241pt}{0.400pt}}
\multiput(1313.00,228.17)(0.500,-1.000){2}{\rule{0.120pt}{0.400pt}}
\put(1314,226.67){\rule{0.482pt}{0.400pt}}
\multiput(1314.00,227.17)(1.000,-1.000){2}{\rule{0.241pt}{0.400pt}}
\put(1316,225.67){\rule{0.241pt}{0.400pt}}
\multiput(1316.00,226.17)(0.500,-1.000){2}{\rule{0.120pt}{0.400pt}}
\put(1317,224.67){\rule{0.241pt}{0.400pt}}
\multiput(1317.00,225.17)(0.500,-1.000){2}{\rule{0.120pt}{0.400pt}}
\put(1318,223.67){\rule{0.482pt}{0.400pt}}
\multiput(1318.00,224.17)(1.000,-1.000){2}{\rule{0.241pt}{0.400pt}}
\put(1320,222.67){\rule{0.241pt}{0.400pt}}
\multiput(1320.00,223.17)(0.500,-1.000){2}{\rule{0.120pt}{0.400pt}}
\put(1321,221.67){\rule{0.241pt}{0.400pt}}
\multiput(1321.00,222.17)(0.500,-1.000){2}{\rule{0.120pt}{0.400pt}}
\put(1322,220.67){\rule{0.241pt}{0.400pt}}
\multiput(1322.00,221.17)(0.500,-1.000){2}{\rule{0.120pt}{0.400pt}}
\put(1323,219.67){\rule{0.482pt}{0.400pt}}
\multiput(1323.00,220.17)(1.000,-1.000){2}{\rule{0.241pt}{0.400pt}}
\put(1325,218.67){\rule{0.241pt}{0.400pt}}
\multiput(1325.00,219.17)(0.500,-1.000){2}{\rule{0.120pt}{0.400pt}}
\put(1326,217.67){\rule{0.241pt}{0.400pt}}
\multiput(1326.00,218.17)(0.500,-1.000){2}{\rule{0.120pt}{0.400pt}}
\put(1327,216.67){\rule{0.482pt}{0.400pt}}
\multiput(1327.00,217.17)(1.000,-1.000){2}{\rule{0.241pt}{0.400pt}}
\put(1329,215.67){\rule{0.241pt}{0.400pt}}
\multiput(1329.00,216.17)(0.500,-1.000){2}{\rule{0.120pt}{0.400pt}}
\put(1330,214.67){\rule{0.241pt}{0.400pt}}
\multiput(1330.00,215.17)(0.500,-1.000){2}{\rule{0.120pt}{0.400pt}}
\put(1331,213.67){\rule{0.482pt}{0.400pt}}
\multiput(1331.00,214.17)(1.000,-1.000){2}{\rule{0.241pt}{0.400pt}}
\put(1333,212.67){\rule{0.241pt}{0.400pt}}
\multiput(1333.00,213.17)(0.500,-1.000){2}{\rule{0.120pt}{0.400pt}}
\put(1334,211.67){\rule{0.241pt}{0.400pt}}
\multiput(1334.00,212.17)(0.500,-1.000){2}{\rule{0.120pt}{0.400pt}}
\put(1335,210.67){\rule{0.482pt}{0.400pt}}
\multiput(1335.00,211.17)(1.000,-1.000){2}{\rule{0.241pt}{0.400pt}}
\put(1337,209.67){\rule{0.241pt}{0.400pt}}
\multiput(1337.00,210.17)(0.500,-1.000){2}{\rule{0.120pt}{0.400pt}}
\put(1338,208.67){\rule{0.241pt}{0.400pt}}
\multiput(1338.00,209.17)(0.500,-1.000){2}{\rule{0.120pt}{0.400pt}}
\put(1339,207.67){\rule{0.482pt}{0.400pt}}
\multiput(1339.00,208.17)(1.000,-1.000){2}{\rule{0.241pt}{0.400pt}}
\put(1341,206.67){\rule{0.241pt}{0.400pt}}
\multiput(1341.00,207.17)(0.500,-1.000){2}{\rule{0.120pt}{0.400pt}}
\put(1342,205.67){\rule{0.241pt}{0.400pt}}
\multiput(1342.00,206.17)(0.500,-1.000){2}{\rule{0.120pt}{0.400pt}}
\put(1343,204.67){\rule{0.482pt}{0.400pt}}
\multiput(1343.00,205.17)(1.000,-1.000){2}{\rule{0.241pt}{0.400pt}}
\put(1345,203.67){\rule{0.241pt}{0.400pt}}
\multiput(1345.00,204.17)(0.500,-1.000){2}{\rule{0.120pt}{0.400pt}}
\put(1346,202.67){\rule{0.241pt}{0.400pt}}
\multiput(1346.00,203.17)(0.500,-1.000){2}{\rule{0.120pt}{0.400pt}}
\put(1347,201.67){\rule{0.482pt}{0.400pt}}
\multiput(1347.00,202.17)(1.000,-1.000){2}{\rule{0.241pt}{0.400pt}}
\put(1348.67,200){\rule{0.400pt}{0.482pt}}
\multiput(1348.17,201.00)(1.000,-1.000){2}{\rule{0.400pt}{0.241pt}}
\put(1350,198.67){\rule{0.241pt}{0.400pt}}
\multiput(1350.00,199.17)(0.500,-1.000){2}{\rule{0.120pt}{0.400pt}}
\put(1351,197.67){\rule{0.482pt}{0.400pt}}
\multiput(1351.00,198.17)(1.000,-1.000){2}{\rule{0.241pt}{0.400pt}}
\put(1353,196.67){\rule{0.241pt}{0.400pt}}
\multiput(1353.00,197.17)(0.500,-1.000){2}{\rule{0.120pt}{0.400pt}}
\put(1354,195.67){\rule{0.241pt}{0.400pt}}
\multiput(1354.00,196.17)(0.500,-1.000){2}{\rule{0.120pt}{0.400pt}}
\put(1355,194.67){\rule{0.482pt}{0.400pt}}
\multiput(1355.00,195.17)(1.000,-1.000){2}{\rule{0.241pt}{0.400pt}}
\put(1357,193.67){\rule{0.241pt}{0.400pt}}
\multiput(1357.00,194.17)(0.500,-1.000){2}{\rule{0.120pt}{0.400pt}}
\put(1358,192.67){\rule{0.241pt}{0.400pt}}
\multiput(1358.00,193.17)(0.500,-1.000){2}{\rule{0.120pt}{0.400pt}}
\put(1359,191.67){\rule{0.482pt}{0.400pt}}
\multiput(1359.00,192.17)(1.000,-1.000){2}{\rule{0.241pt}{0.400pt}}
\put(1361,190.67){\rule{0.241pt}{0.400pt}}
\multiput(1361.00,191.17)(0.500,-1.000){2}{\rule{0.120pt}{0.400pt}}
\put(1362,189.67){\rule{0.241pt}{0.400pt}}
\multiput(1362.00,190.17)(0.500,-1.000){2}{\rule{0.120pt}{0.400pt}}
\put(1363,188.67){\rule{0.482pt}{0.400pt}}
\multiput(1363.00,189.17)(1.000,-1.000){2}{\rule{0.241pt}{0.400pt}}
\put(1365,187.67){\rule{0.241pt}{0.400pt}}
\multiput(1365.00,188.17)(0.500,-1.000){2}{\rule{0.120pt}{0.400pt}}
\put(1366,186.67){\rule{0.241pt}{0.400pt}}
\multiput(1366.00,187.17)(0.500,-1.000){2}{\rule{0.120pt}{0.400pt}}
\put(1367,185.67){\rule{0.482pt}{0.400pt}}
\multiput(1367.00,186.17)(1.000,-1.000){2}{\rule{0.241pt}{0.400pt}}
\put(1369,184.67){\rule{0.241pt}{0.400pt}}
\multiput(1369.00,185.17)(0.500,-1.000){2}{\rule{0.120pt}{0.400pt}}
\put(1370,183.67){\rule{0.241pt}{0.400pt}}
\multiput(1370.00,184.17)(0.500,-1.000){2}{\rule{0.120pt}{0.400pt}}
\put(1371,182.67){\rule{0.482pt}{0.400pt}}
\multiput(1371.00,183.17)(1.000,-1.000){2}{\rule{0.241pt}{0.400pt}}
\put(1373,181.67){\rule{0.241pt}{0.400pt}}
\multiput(1373.00,182.17)(0.500,-1.000){2}{\rule{0.120pt}{0.400pt}}
\put(1374,180.67){\rule{0.241pt}{0.400pt}}
\multiput(1374.00,181.17)(0.500,-1.000){2}{\rule{0.120pt}{0.400pt}}
\put(1375,179.67){\rule{0.482pt}{0.400pt}}
\multiput(1375.00,180.17)(1.000,-1.000){2}{\rule{0.241pt}{0.400pt}}
\put(1377,178.67){\rule{0.241pt}{0.400pt}}
\multiput(1377.00,179.17)(0.500,-1.000){2}{\rule{0.120pt}{0.400pt}}
\put(1378,177.67){\rule{0.241pt}{0.400pt}}
\multiput(1378.00,178.17)(0.500,-1.000){2}{\rule{0.120pt}{0.400pt}}
\put(1379,176.67){\rule{0.482pt}{0.400pt}}
\multiput(1379.00,177.17)(1.000,-1.000){2}{\rule{0.241pt}{0.400pt}}
\put(1381,175.67){\rule{0.241pt}{0.400pt}}
\multiput(1381.00,176.17)(0.500,-1.000){2}{\rule{0.120pt}{0.400pt}}
\put(1382,174.67){\rule{0.241pt}{0.400pt}}
\multiput(1382.00,175.17)(0.500,-1.000){2}{\rule{0.120pt}{0.400pt}}
\put(1383,173.67){\rule{0.482pt}{0.400pt}}
\multiput(1383.00,174.17)(1.000,-1.000){2}{\rule{0.241pt}{0.400pt}}
\put(1384.67,172){\rule{0.400pt}{0.482pt}}
\multiput(1384.17,173.00)(1.000,-1.000){2}{\rule{0.400pt}{0.241pt}}
\put(1386,170.67){\rule{0.241pt}{0.400pt}}
\multiput(1386.00,171.17)(0.500,-1.000){2}{\rule{0.120pt}{0.400pt}}
\put(1387,169.67){\rule{0.482pt}{0.400pt}}
\multiput(1387.00,170.17)(1.000,-1.000){2}{\rule{0.241pt}{0.400pt}}
\put(1389,168.67){\rule{0.241pt}{0.400pt}}
\multiput(1389.00,169.17)(0.500,-1.000){2}{\rule{0.120pt}{0.400pt}}
\put(1390,167.67){\rule{0.241pt}{0.400pt}}
\multiput(1390.00,168.17)(0.500,-1.000){2}{\rule{0.120pt}{0.400pt}}
\put(1391,166.67){\rule{0.482pt}{0.400pt}}
\multiput(1391.00,167.17)(1.000,-1.000){2}{\rule{0.241pt}{0.400pt}}
\put(1393,165.67){\rule{0.241pt}{0.400pt}}
\multiput(1393.00,166.17)(0.500,-1.000){2}{\rule{0.120pt}{0.400pt}}
\put(1394,164.67){\rule{0.241pt}{0.400pt}}
\multiput(1394.00,165.17)(0.500,-1.000){2}{\rule{0.120pt}{0.400pt}}
\put(1395,163.67){\rule{0.482pt}{0.400pt}}
\multiput(1395.00,164.17)(1.000,-1.000){2}{\rule{0.241pt}{0.400pt}}
\put(1397,162.67){\rule{0.241pt}{0.400pt}}
\multiput(1397.00,163.17)(0.500,-1.000){2}{\rule{0.120pt}{0.400pt}}
\put(1398,161.67){\rule{0.241pt}{0.400pt}}
\multiput(1398.00,162.17)(0.500,-1.000){2}{\rule{0.120pt}{0.400pt}}
\put(1399,160.67){\rule{0.241pt}{0.400pt}}
\multiput(1399.00,161.17)(0.500,-1.000){2}{\rule{0.120pt}{0.400pt}}
\put(1400,159.67){\rule{0.482pt}{0.400pt}}
\multiput(1400.00,160.17)(1.000,-1.000){2}{\rule{0.241pt}{0.400pt}}
\put(1402,158.67){\rule{0.241pt}{0.400pt}}
\multiput(1402.00,159.17)(0.500,-1.000){2}{\rule{0.120pt}{0.400pt}}
\put(1403,157.67){\rule{0.241pt}{0.400pt}}
\multiput(1403.00,158.17)(0.500,-1.000){2}{\rule{0.120pt}{0.400pt}}
\put(1404,156.67){\rule{0.482pt}{0.400pt}}
\multiput(1404.00,157.17)(1.000,-1.000){2}{\rule{0.241pt}{0.400pt}}
\put(1406,155.67){\rule{0.241pt}{0.400pt}}
\multiput(1406.00,156.17)(0.500,-1.000){2}{\rule{0.120pt}{0.400pt}}
\put(1407,154.67){\rule{0.241pt}{0.400pt}}
\multiput(1407.00,155.17)(0.500,-1.000){2}{\rule{0.120pt}{0.400pt}}
\put(1408,153.67){\rule{0.482pt}{0.400pt}}
\multiput(1408.00,154.17)(1.000,-1.000){2}{\rule{0.241pt}{0.400pt}}
\put(1410,152.67){\rule{0.241pt}{0.400pt}}
\multiput(1410.00,153.17)(0.500,-1.000){2}{\rule{0.120pt}{0.400pt}}
\put(1411,151.67){\rule{0.241pt}{0.400pt}}
\multiput(1411.00,152.17)(0.500,-1.000){2}{\rule{0.120pt}{0.400pt}}
\put(1412,150.67){\rule{0.482pt}{0.400pt}}
\multiput(1412.00,151.17)(1.000,-1.000){2}{\rule{0.241pt}{0.400pt}}
\put(1414,149.67){\rule{0.241pt}{0.400pt}}
\multiput(1414.00,150.17)(0.500,-1.000){2}{\rule{0.120pt}{0.400pt}}
\put(1415,148.67){\rule{0.241pt}{0.400pt}}
\multiput(1415.00,149.17)(0.500,-1.000){2}{\rule{0.120pt}{0.400pt}}
\put(1416,147.67){\rule{0.482pt}{0.400pt}}
\multiput(1416.00,148.17)(1.000,-1.000){2}{\rule{0.241pt}{0.400pt}}
\put(1418,146.67){\rule{0.241pt}{0.400pt}}
\multiput(1418.00,147.17)(0.500,-1.000){2}{\rule{0.120pt}{0.400pt}}
\put(1419,145.67){\rule{0.241pt}{0.400pt}}
\multiput(1419.00,146.17)(0.500,-1.000){2}{\rule{0.120pt}{0.400pt}}
\put(1420,144.17){\rule{0.482pt}{0.400pt}}
\multiput(1420.00,145.17)(1.000,-2.000){2}{\rule{0.241pt}{0.400pt}}
\put(1422,142.67){\rule{0.241pt}{0.400pt}}
\multiput(1422.00,143.17)(0.500,-1.000){2}{\rule{0.120pt}{0.400pt}}
\put(1423,141.67){\rule{0.241pt}{0.400pt}}
\multiput(1423.00,142.17)(0.500,-1.000){2}{\rule{0.120pt}{0.400pt}}
\put(1424,140.67){\rule{0.482pt}{0.400pt}}
\multiput(1424.00,141.17)(1.000,-1.000){2}{\rule{0.241pt}{0.400pt}}
\put(1426,139.67){\rule{0.241pt}{0.400pt}}
\multiput(1426.00,140.17)(0.500,-1.000){2}{\rule{0.120pt}{0.400pt}}
\put(1427,138.67){\rule{0.241pt}{0.400pt}}
\multiput(1427.00,139.17)(0.500,-1.000){2}{\rule{0.120pt}{0.400pt}}
\put(1428,137.67){\rule{0.482pt}{0.400pt}}
\multiput(1428.00,138.17)(1.000,-1.000){2}{\rule{0.241pt}{0.400pt}}
\put(1430,136.67){\rule{0.241pt}{0.400pt}}
\multiput(1430.00,137.17)(0.500,-1.000){2}{\rule{0.120pt}{0.400pt}}
\put(1431,135.67){\rule{0.241pt}{0.400pt}}
\multiput(1431.00,136.17)(0.500,-1.000){2}{\rule{0.120pt}{0.400pt}}
\put(1432,134.67){\rule{0.482pt}{0.400pt}}
\multiput(1432.00,135.17)(1.000,-1.000){2}{\rule{0.241pt}{0.400pt}}
\put(1434,133.67){\rule{0.241pt}{0.400pt}}
\multiput(1434.00,134.17)(0.500,-1.000){2}{\rule{0.120pt}{0.400pt}}
\put(1435,132.67){\rule{0.241pt}{0.400pt}}
\multiput(1435.00,133.17)(0.500,-1.000){2}{\rule{0.120pt}{0.400pt}}
\put(1436,131.67){\rule{0.482pt}{0.400pt}}
\multiput(1436.00,132.17)(1.000,-1.000){2}{\rule{0.241pt}{0.400pt}}
\put(814.0,614.0){\usebox{\plotpoint}}
\put(1279,776){\makebox(0,0)[r]{exact solution}}
\multiput(1299,776)(20.756,0.000){5}{\usebox{\plotpoint}}
\put(1399,776){\usebox{\plotpoint}}
\put(111,140){\usebox{\plotpoint}}
\put(111.00,140.00){\usebox{\plotpoint}}
\put(114.16,160.48){\usebox{\plotpoint}}
\put(117.55,180.91){\usebox{\plotpoint}}
\put(120.84,201.36){\usebox{\plotpoint}}
\put(123.65,221.91){\usebox{\plotpoint}}
\put(126.80,242.40){\usebox{\plotpoint}}
\put(130.35,262.81){\usebox{\plotpoint}}
\put(134.03,283.21){\usebox{\plotpoint}}
\put(137.70,303.58){\usebox{\plotpoint}}
\put(141.49,323.94){\usebox{\plotpoint}}
\put(145.33,344.28){\usebox{\plotpoint}}
\put(149.37,364.60){\usebox{\plotpoint}}
\put(153.00,385.00){\usebox{\plotpoint}}
\put(157.40,405.21){\usebox{\plotpoint}}
\put(161.81,425.42){\usebox{\plotpoint}}
\put(166.43,445.56){\usebox{\plotpoint}}
\put(170.95,465.72){\usebox{\plotpoint}}
\put(175.77,485.87){\usebox{\plotpoint}}
\put(181.15,505.86){\usebox{\plotpoint}}
\put(186.56,525.81){\usebox{\plotpoint}}
\put(191.95,545.80){\usebox{\plotpoint}}
\put(198.36,565.43){\usebox{\plotpoint}}
\put(204.13,585.26){\usebox{\plotpoint}}
\put(210.93,604.74){\usebox{\plotpoint}}
\put(218.24,623.97){\usebox{\plotpoint}}
\put(225.64,643.28){\usebox{\plotpoint}}
\put(232.87,662.60){\usebox{\plotpoint}}
\put(241.33,681.33){\usebox{\plotpoint}}
\put(251.08,699.23){\usebox{\plotpoint}}
\put(260.48,717.45){\usebox{\plotpoint}}
\put(271.41,734.82){\usebox{\plotpoint}}
\put(283.27,751.53){\usebox{\plotpoint}}
\put(295.75,767.51){\usebox{\plotpoint}}
\put(309.54,782.54){\usebox{\plotpoint}}
\put(324.73,795.73){\usebox{\plotpoint}}
\put(341.54,806.54){\usebox{\plotpoint}}
\put(359.19,815.60){\usebox{\plotpoint}}
\put(377.54,822.00){\usebox{\plotpoint}}
\put(397.04,825.04){\usebox{\plotpoint}}
\put(416.98,827.00){\usebox{\plotpoint}}
\put(437.09,825.00){\usebox{\plotpoint}}
\put(456.78,822.00){\usebox{\plotpoint}}
\put(475.75,817.25){\usebox{\plotpoint}}
\put(494.36,811.64){\usebox{\plotpoint}}
\put(512.64,804.68){\usebox{\plotpoint}}
\put(530.78,797.22){\usebox{\plotpoint}}
\put(548.35,788.65){\usebox{\plotpoint}}
\put(566.10,779.45){\usebox{\plotpoint}}
\put(583.69,770.31){\usebox{\plotpoint}}
\put(600.92,760.08){\usebox{\plotpoint}}
\put(618.19,749.91){\usebox{\plotpoint}}
\put(635.37,739.31){\usebox{\plotpoint}}
\put(652.44,728.56){\usebox{\plotpoint}}
\put(669.09,716.91){\usebox{\plotpoint}}
\put(685.77,706.23){\usebox{\plotpoint}}
\put(702.42,694.58){\usebox{\plotpoint}}
\put(719.08,682.96){\usebox{\plotpoint}}
\put(735.89,671.55){\usebox{\plotpoint}}
\put(752.34,659.33){\usebox{\plotpoint}}
\put(769.12,647.88){\usebox{\plotpoint}}
\put(785.33,635.33){\usebox{\plotpoint}}
\put(801.78,623.11){\usebox{\plotpoint}}
\put(818.47,611.53){\usebox{\plotpoint}}
\put(834.82,599.18){\usebox{\plotpoint}}
\put(851.17,586.83){\usebox{\plotpoint}}
\put(867.53,574.47){\usebox{\plotpoint}}
\put(883.88,562.12){\usebox{\plotpoint}}
\put(900.23,549.77){\usebox{\plotpoint}}
\put(916.16,536.84){\usebox{\plotpoint}}
\put(932.51,524.49){\usebox{\plotpoint}}
\put(948.87,512.13){\usebox{\plotpoint}}
\put(965.22,499.78){\usebox{\plotpoint}}
\put(980.99,487.01){\usebox{\plotpoint}}
\put(997.34,474.66){\usebox{\plotpoint}}
\put(1013.69,462.31){\usebox{\plotpoint}}
\put(1030.04,449.96){\usebox{\plotpoint}}
\put(1045.81,437.19){\usebox{\plotpoint}}
\put(1062.17,424.83){\usebox{\plotpoint}}
\put(1078.33,412.34){\usebox{\plotpoint}}
\put(1094.29,399.71){\usebox{\plotpoint}}
\put(1110.64,387.36){\usebox{\plotpoint}}
\put(1126.57,374.43){\usebox{\plotpoint}}
\put(1142.93,362.07){\usebox{\plotpoint}}
\put(1159.28,349.72){\usebox{\plotpoint}}
\put(1175.05,336.95){\usebox{\plotpoint}}
\put(1191.40,324.60){\usebox{\plotpoint}}
\put(1207.17,311.83){\usebox{\plotpoint}}
\put(1223.52,299.48){\usebox{\plotpoint}}
\put(1239.45,286.55){\usebox{\plotpoint}}
\put(1255.76,274.12){\usebox{\plotpoint}}
\put(1272.16,261.84){\usebox{\plotpoint}}
\put(1288.09,248.91){\usebox{\plotpoint}}
\put(1304.44,236.56){\usebox{\plotpoint}}
\put(1320.21,223.79){\usebox{\plotpoint}}
\put(1336.45,211.27){\usebox{\plotpoint}}
\put(1352.16,198.42){\usebox{\plotpoint}}
\put(1368.61,186.20){\usebox{\plotpoint}}
\put(1385.03,173.95){\usebox{\plotpoint}}
\put(1400.50,160.75){\usebox{\plotpoint}}
\put(1416.94,148.53){\usebox{\plotpoint}}
\put(1432.86,135.57){\usebox{\plotpoint}}
\put(1438,132){\usebox{\plotpoint}}
\put(110.0,131.0){\rule[-0.200pt]{0.400pt}{175.134pt}}
\put(110.0,131.0){\rule[-0.200pt]{320.156pt}{0.400pt}}
\put(1439.0,131.0){\rule[-0.200pt]{0.400pt}{175.134pt}}
\put(110.0,858.0){\rule[-0.200pt]{320.156pt}{0.400pt}}
\end{picture}

					\end{center}
					\caption{Comparative graph between numerical and exact results for $n=1000$}
			\end{figure}
			
			
	\paragraph{} We can see that for $n\geq 100$ (and even probably before that) the expected and the numerical results are similar.
\\ The smaller the step length is the closer we are to the real plot and the more we have values the more we can be precise but we are still confronted with the numerical lack of precision for large values. (see examples in the results file)
			
			
			\section{Comparisons}
	
			\begin{figure}[htbp]
					\begin{center}
							% GNUPLOT: LaTeX picture
\setlength{\unitlength}{0.240900pt}
\ifx\plotpoint\undefined\newsavebox{\plotpoint}\fi
\begin{picture}(1500,900)(0,0)
\sbox{\plotpoint}{\rule[-0.200pt]{0.400pt}{0.400pt}}%
\put(110.0,131.0){\rule[-0.200pt]{4.818pt}{0.400pt}}
\put(90,131){\makebox(0,0)[r]{$-14$}}
\put(1419.0,131.0){\rule[-0.200pt]{4.818pt}{0.400pt}}
\put(110.0,235.0){\rule[-0.200pt]{4.818pt}{0.400pt}}
\put(90,235){\makebox(0,0)[r]{$-12$}}
\put(1419.0,235.0){\rule[-0.200pt]{4.818pt}{0.400pt}}
\put(110.0,339.0){\rule[-0.200pt]{4.818pt}{0.400pt}}
\put(90,339){\makebox(0,0)[r]{$-10$}}
\put(1419.0,339.0){\rule[-0.200pt]{4.818pt}{0.400pt}}
\put(110.0,443.0){\rule[-0.200pt]{4.818pt}{0.400pt}}
\put(90,443){\makebox(0,0)[r]{$-8$}}
\put(1419.0,443.0){\rule[-0.200pt]{4.818pt}{0.400pt}}
\put(110.0,546.0){\rule[-0.200pt]{4.818pt}{0.400pt}}
\put(90,546){\makebox(0,0)[r]{$-6$}}
\put(1419.0,546.0){\rule[-0.200pt]{4.818pt}{0.400pt}}
\put(110.0,650.0){\rule[-0.200pt]{4.818pt}{0.400pt}}
\put(90,650){\makebox(0,0)[r]{$-4$}}
\put(1419.0,650.0){\rule[-0.200pt]{4.818pt}{0.400pt}}
\put(110.0,754.0){\rule[-0.200pt]{4.818pt}{0.400pt}}
\put(90,754){\makebox(0,0)[r]{$-2$}}
\put(1419.0,754.0){\rule[-0.200pt]{4.818pt}{0.400pt}}
\put(110.0,858.0){\rule[-0.200pt]{4.818pt}{0.400pt}}
\put(90,858){\makebox(0,0)[r]{$0$}}
\put(1419.0,858.0){\rule[-0.200pt]{4.818pt}{0.400pt}}
\put(110.0,131.0){\rule[-0.200pt]{0.400pt}{4.818pt}}
\put(110,90){\makebox(0,0){$10$}}
\put(110.0,838.0){\rule[-0.200pt]{0.400pt}{4.818pt}}
\put(177.0,131.0){\rule[-0.200pt]{0.400pt}{2.409pt}}
\put(177.0,848.0){\rule[-0.200pt]{0.400pt}{2.409pt}}
\put(216.0,131.0){\rule[-0.200pt]{0.400pt}{2.409pt}}
\put(216.0,848.0){\rule[-0.200pt]{0.400pt}{2.409pt}}
\put(243.0,131.0){\rule[-0.200pt]{0.400pt}{2.409pt}}
\put(243.0,848.0){\rule[-0.200pt]{0.400pt}{2.409pt}}
\put(265.0,131.0){\rule[-0.200pt]{0.400pt}{2.409pt}}
\put(265.0,848.0){\rule[-0.200pt]{0.400pt}{2.409pt}}
\put(282.0,131.0){\rule[-0.200pt]{0.400pt}{2.409pt}}
\put(282.0,848.0){\rule[-0.200pt]{0.400pt}{2.409pt}}
\put(297.0,131.0){\rule[-0.200pt]{0.400pt}{2.409pt}}
\put(297.0,848.0){\rule[-0.200pt]{0.400pt}{2.409pt}}
\put(310.0,131.0){\rule[-0.200pt]{0.400pt}{2.409pt}}
\put(310.0,848.0){\rule[-0.200pt]{0.400pt}{2.409pt}}
\put(321.0,131.0){\rule[-0.200pt]{0.400pt}{2.409pt}}
\put(321.0,848.0){\rule[-0.200pt]{0.400pt}{2.409pt}}
\put(332.0,131.0){\rule[-0.200pt]{0.400pt}{4.818pt}}
\put(332,90){\makebox(0,0){$100$}}
\put(332.0,838.0){\rule[-0.200pt]{0.400pt}{4.818pt}}
\put(398.0,131.0){\rule[-0.200pt]{0.400pt}{2.409pt}}
\put(398.0,848.0){\rule[-0.200pt]{0.400pt}{2.409pt}}
\put(437.0,131.0){\rule[-0.200pt]{0.400pt}{2.409pt}}
\put(437.0,848.0){\rule[-0.200pt]{0.400pt}{2.409pt}}
\put(465.0,131.0){\rule[-0.200pt]{0.400pt}{2.409pt}}
\put(465.0,848.0){\rule[-0.200pt]{0.400pt}{2.409pt}}
\put(486.0,131.0){\rule[-0.200pt]{0.400pt}{2.409pt}}
\put(486.0,848.0){\rule[-0.200pt]{0.400pt}{2.409pt}}
\put(504.0,131.0){\rule[-0.200pt]{0.400pt}{2.409pt}}
\put(504.0,848.0){\rule[-0.200pt]{0.400pt}{2.409pt}}
\put(519.0,131.0){\rule[-0.200pt]{0.400pt}{2.409pt}}
\put(519.0,848.0){\rule[-0.200pt]{0.400pt}{2.409pt}}
\put(532.0,131.0){\rule[-0.200pt]{0.400pt}{2.409pt}}
\put(532.0,848.0){\rule[-0.200pt]{0.400pt}{2.409pt}}
\put(543.0,131.0){\rule[-0.200pt]{0.400pt}{2.409pt}}
\put(543.0,848.0){\rule[-0.200pt]{0.400pt}{2.409pt}}
\put(553.0,131.0){\rule[-0.200pt]{0.400pt}{4.818pt}}
\put(553,90){\makebox(0,0){$1000$}}
\put(553.0,838.0){\rule[-0.200pt]{0.400pt}{4.818pt}}
\put(620.0,131.0){\rule[-0.200pt]{0.400pt}{2.409pt}}
\put(620.0,848.0){\rule[-0.200pt]{0.400pt}{2.409pt}}
\put(659.0,131.0){\rule[-0.200pt]{0.400pt}{2.409pt}}
\put(659.0,848.0){\rule[-0.200pt]{0.400pt}{2.409pt}}
\put(686.0,131.0){\rule[-0.200pt]{0.400pt}{2.409pt}}
\put(686.0,848.0){\rule[-0.200pt]{0.400pt}{2.409pt}}
\put(708.0,131.0){\rule[-0.200pt]{0.400pt}{2.409pt}}
\put(708.0,848.0){\rule[-0.200pt]{0.400pt}{2.409pt}}
\put(725.0,131.0){\rule[-0.200pt]{0.400pt}{2.409pt}}
\put(725.0,848.0){\rule[-0.200pt]{0.400pt}{2.409pt}}
\put(740.0,131.0){\rule[-0.200pt]{0.400pt}{2.409pt}}
\put(740.0,848.0){\rule[-0.200pt]{0.400pt}{2.409pt}}
\put(753.0,131.0){\rule[-0.200pt]{0.400pt}{2.409pt}}
\put(753.0,848.0){\rule[-0.200pt]{0.400pt}{2.409pt}}
\put(764.0,131.0){\rule[-0.200pt]{0.400pt}{2.409pt}}
\put(764.0,848.0){\rule[-0.200pt]{0.400pt}{2.409pt}}
\put(775.0,131.0){\rule[-0.200pt]{0.400pt}{4.818pt}}
\put(775,90){\makebox(0,0){$10000$}}
\put(775.0,838.0){\rule[-0.200pt]{0.400pt}{4.818pt}}
\put(841.0,131.0){\rule[-0.200pt]{0.400pt}{2.409pt}}
\put(841.0,848.0){\rule[-0.200pt]{0.400pt}{2.409pt}}
\put(880.0,131.0){\rule[-0.200pt]{0.400pt}{2.409pt}}
\put(880.0,848.0){\rule[-0.200pt]{0.400pt}{2.409pt}}
\put(908.0,131.0){\rule[-0.200pt]{0.400pt}{2.409pt}}
\put(908.0,848.0){\rule[-0.200pt]{0.400pt}{2.409pt}}
\put(929.0,131.0){\rule[-0.200pt]{0.400pt}{2.409pt}}
\put(929.0,848.0){\rule[-0.200pt]{0.400pt}{2.409pt}}
\put(947.0,131.0){\rule[-0.200pt]{0.400pt}{2.409pt}}
\put(947.0,848.0){\rule[-0.200pt]{0.400pt}{2.409pt}}
\put(962.0,131.0){\rule[-0.200pt]{0.400pt}{2.409pt}}
\put(962.0,848.0){\rule[-0.200pt]{0.400pt}{2.409pt}}
\put(975.0,131.0){\rule[-0.200pt]{0.400pt}{2.409pt}}
\put(975.0,848.0){\rule[-0.200pt]{0.400pt}{2.409pt}}
\put(986.0,131.0){\rule[-0.200pt]{0.400pt}{2.409pt}}
\put(986.0,848.0){\rule[-0.200pt]{0.400pt}{2.409pt}}
\put(996.0,131.0){\rule[-0.200pt]{0.400pt}{4.818pt}}
\put(996,90){\makebox(0,0){$100000$}}
\put(996.0,838.0){\rule[-0.200pt]{0.400pt}{4.818pt}}
\put(1063.0,131.0){\rule[-0.200pt]{0.400pt}{2.409pt}}
\put(1063.0,848.0){\rule[-0.200pt]{0.400pt}{2.409pt}}
\put(1102.0,131.0){\rule[-0.200pt]{0.400pt}{2.409pt}}
\put(1102.0,848.0){\rule[-0.200pt]{0.400pt}{2.409pt}}
\put(1129.0,131.0){\rule[-0.200pt]{0.400pt}{2.409pt}}
\put(1129.0,848.0){\rule[-0.200pt]{0.400pt}{2.409pt}}
\put(1151.0,131.0){\rule[-0.200pt]{0.400pt}{2.409pt}}
\put(1151.0,848.0){\rule[-0.200pt]{0.400pt}{2.409pt}}
\put(1168.0,131.0){\rule[-0.200pt]{0.400pt}{2.409pt}}
\put(1168.0,848.0){\rule[-0.200pt]{0.400pt}{2.409pt}}
\put(1183.0,131.0){\rule[-0.200pt]{0.400pt}{2.409pt}}
\put(1183.0,848.0){\rule[-0.200pt]{0.400pt}{2.409pt}}
\put(1196.0,131.0){\rule[-0.200pt]{0.400pt}{2.409pt}}
\put(1196.0,848.0){\rule[-0.200pt]{0.400pt}{2.409pt}}
\put(1207.0,131.0){\rule[-0.200pt]{0.400pt}{2.409pt}}
\put(1207.0,848.0){\rule[-0.200pt]{0.400pt}{2.409pt}}
\put(1217.0,131.0){\rule[-0.200pt]{0.400pt}{4.818pt}}
\put(1217,90){\makebox(0,0){$1\times10^{6}$}}
\put(1217.0,838.0){\rule[-0.200pt]{0.400pt}{4.818pt}}
\put(1284.0,131.0){\rule[-0.200pt]{0.400pt}{2.409pt}}
\put(1284.0,848.0){\rule[-0.200pt]{0.400pt}{2.409pt}}
\put(1323.0,131.0){\rule[-0.200pt]{0.400pt}{2.409pt}}
\put(1323.0,848.0){\rule[-0.200pt]{0.400pt}{2.409pt}}
\put(1351.0,131.0){\rule[-0.200pt]{0.400pt}{2.409pt}}
\put(1351.0,848.0){\rule[-0.200pt]{0.400pt}{2.409pt}}
\put(1372.0,131.0){\rule[-0.200pt]{0.400pt}{2.409pt}}
\put(1372.0,848.0){\rule[-0.200pt]{0.400pt}{2.409pt}}
\put(1390.0,131.0){\rule[-0.200pt]{0.400pt}{2.409pt}}
\put(1390.0,848.0){\rule[-0.200pt]{0.400pt}{2.409pt}}
\put(1405.0,131.0){\rule[-0.200pt]{0.400pt}{2.409pt}}
\put(1405.0,848.0){\rule[-0.200pt]{0.400pt}{2.409pt}}
\put(1418.0,131.0){\rule[-0.200pt]{0.400pt}{2.409pt}}
\put(1418.0,848.0){\rule[-0.200pt]{0.400pt}{2.409pt}}
\put(1429.0,131.0){\rule[-0.200pt]{0.400pt}{2.409pt}}
\put(1429.0,848.0){\rule[-0.200pt]{0.400pt}{2.409pt}}
\put(1439.0,131.0){\rule[-0.200pt]{0.400pt}{4.818pt}}
\put(1439,90){\makebox(0,0){$1\times10^{7}$}}
\put(1439.0,838.0){\rule[-0.200pt]{0.400pt}{4.818pt}}
\put(110.0,131.0){\rule[-0.200pt]{0.400pt}{175.134pt}}
\put(110.0,131.0){\rule[-0.200pt]{320.156pt}{0.400pt}}
\put(1439.0,131.0){\rule[-0.200pt]{0.400pt}{175.134pt}}
\put(110.0,858.0){\rule[-0.200pt]{320.156pt}{0.400pt}}
\put(774,29){\makebox(0,0){n}}
\put(1279,817){\makebox(0,0)[r]{relative error}}
\put(1299.0,817.0){\rule[-0.200pt]{24.090pt}{0.400pt}}
\put(110,798){\usebox{\plotpoint}}
\multiput(110.00,796.92)(1.126,-0.499){135}{\rule{0.999pt}{0.120pt}}
\multiput(110.00,797.17)(152.927,-69.000){2}{\rule{0.499pt}{0.400pt}}
\multiput(265.00,727.92)(1.086,-0.497){59}{\rule{0.965pt}{0.120pt}}
\multiput(265.00,728.17)(64.998,-31.000){2}{\rule{0.482pt}{0.400pt}}
\multiput(332.00,696.92)(1.057,-0.499){143}{\rule{0.944pt}{0.120pt}}
\multiput(332.00,697.17)(152.041,-73.000){2}{\rule{0.472pt}{0.400pt}}
\multiput(486.00,623.92)(1.086,-0.497){59}{\rule{0.965pt}{0.120pt}}
\multiput(486.00,624.17)(64.998,-31.000){2}{\rule{0.482pt}{0.400pt}}
\multiput(553.00,592.92)(1.079,-0.499){141}{\rule{0.961pt}{0.120pt}}
\multiput(553.00,593.17)(153.005,-72.000){2}{\rule{0.481pt}{0.400pt}}
\multiput(708.00,520.92)(1.052,-0.497){61}{\rule{0.938pt}{0.120pt}}
\multiput(708.00,521.17)(65.054,-32.000){2}{\rule{0.469pt}{0.400pt}}
\multiput(775.00,488.92)(1.072,-0.499){141}{\rule{0.956pt}{0.120pt}}
\multiput(775.00,489.17)(152.017,-72.000){2}{\rule{0.478pt}{0.400pt}}
\multiput(929.00,416.92)(1.204,-0.497){53}{\rule{1.057pt}{0.120pt}}
\multiput(929.00,417.17)(64.806,-28.000){2}{\rule{0.529pt}{0.400pt}}
\multiput(996.00,390.58)(1.079,0.499){141}{\rule{0.961pt}{0.120pt}}
\multiput(996.00,389.17)(153.005,72.000){2}{\rule{0.481pt}{0.400pt}}
\multiput(1151.00,462.58)(0.751,0.498){85}{\rule{0.700pt}{0.120pt}}
\multiput(1151.00,461.17)(64.547,44.000){2}{\rule{0.350pt}{0.400pt}}
\multiput(1217.00,506.58)(2.297,0.498){65}{\rule{1.924pt}{0.120pt}}
\multiput(1217.00,505.17)(151.008,34.000){2}{\rule{0.962pt}{0.400pt}}
\multiput(1372.58,530.99)(0.499,-2.600){131}{\rule{0.120pt}{2.172pt}}
\multiput(1371.17,535.49)(67.000,-342.493){2}{\rule{0.400pt}{1.086pt}}
\put(110.0,131.0){\rule[-0.200pt]{0.400pt}{175.134pt}}
\put(110.0,131.0){\rule[-0.200pt]{320.156pt}{0.400pt}}
\put(1439.0,131.0){\rule[-0.200pt]{0.400pt}{175.134pt}}
\put(110.0,858.0){\rule[-0.200pt]{320.156pt}{0.400pt}}
\end{picture}

					\end{center}
					\caption{Relative error}
			\end{figure}
			
		\paragraph{} The relative error is quite log-linear for log(n) until $n=10^5$. In theory, if $n$ tends to $\infty$, our precision gets to be perfect. However, since computers can only work with a finite number of numbers, the precision directly depends on the way we store numbers in the computer. For the float type, the computer uses 32 bits, for the double type, it uses 64 bits. This leads to a precision of ${10}^{-7}$, which implies that the computer considered all numbers $x\leq {10}^{-7}$ to be $0$. This is called the loss of numerical precision and it has been a problem here for the relative errors. In many cases, we asked the computer to compare two values, and the results may not be reliable\footnote{See the data files.}.
			
		\paragraph{}
		\begin{center}
		\begin{tabular}{|*{2}{c|}}
  \hline
  \multicolumn{2}{|c|}{Number of FLOPs} \\
	\hline
	General case & Tridiagonal case\\
	\hline	
	$3n^3 + 2n^2 +2n -5$ & $11n-6$ \\
	\hline
\end{tabular}
\end{center}
	
		\chapter*{Conclusion}	
		
		\paragraph{} This project may be small but it gathers many of the problems a programer has to deal with every day : loss of numerical precision, simplicity of the algorithm, operation time, the ability to use the same code for future projects. We have developed this c++ algorithm to be a good balance of all the criteria. We know that it is not the most precise or the quickest  but it would be enough for most of the easy scientific problems. However we already realize that loss of numerical precision will be a crucial issue for other problems.
		
	
				
\end{document}

















